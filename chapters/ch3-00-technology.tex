\part{Teknologi}
\label{ch:teknologi}
\chapter*{Teknologi}

\begin{chapquote}{Lewis Carroll, \textit{Alice i Eventyrland}}
\enquote{Nu skal jeg klare det bedre denne gang,} sagde hun til sig selv og 
begyndte med at tage den lille gyldne nøgle og låse op for døren, der førte ud 
i haven. 
\end{chapquote}

Gyldne nøgler, ure der kun virker af og til, kapløb for at løse
mærkelige gåder og byggere uden ansigter eller navne. Hvad lyder som
eventyr fra Eventyrland er dagligdag i Bitcoin-verdenen.

Som vi udforskede i kapitel~\ref{ch:økonomi}, er store dele af det nuværende 
finansielle system systematisk ødelagt. Ligesom Alice kan vi kun håbe på at 
klare det bedre denne gang. Men takket være en pseudonym opfinder har vi utrolig
avanceret teknologi til at støtte os denne gang: Bitcoin.

At løse problemer i en radikalt decentraliseret og fjendtlig miljø
kræver unikke løsninger. Hvad der ellers ville være trivielle problemer at løse
er alt andet end det i denne mærkelige verden af noder. Bitcoin er afhængig af 
stærk kryptografi for de fleste løsninger, i det mindste hvis man ser det gennem
teknologiens linse. Præcis hvor stærk denne kryptografi er, vil blive udforsket 
i en af de efterfølgende lektioner.

Kryptografi er det, Bitcoin bruger til at fjerne tillid til myndigheder.
I stedet for at stole på centraliserede institutioner, er systemet afhængigt 
af universets endelige autoritet: fysik. Dog er der stadig nogle korn af tillid 
tilbage. Vi vil undersøge disse korn i den anden lektion af dette kapitel.

~

\begin{samepage}
Del~\ref{ch:teknologi} -- Teknologi:

\begin{enumerate}
  \setcounter{enumi}{14}
  \item Styrke i antal
  \item Refleksioner over \enquote{Stol ikke, verificer}
  \item At fortælle tid kræver arbejde
  \item Bevæg dig langsomt og ødelæg ikke ting
  \item Privatlivet er ikke dødt
  \item Cypherpunks skriver kode
  \item Metaforer for Bitcoins fremtid
\end{enumerate}
\end{samepage}

De sidste par lektioner udforsker ethos inden for teknologisk udvikling i
Bitcoin, hvilket argumenteres for at være lige så vigtigt som teknologien selv. 
Bitcoin er ikke den næste glimrende app på din telefon. Det er grundlaget for 
en ny økonomisk virkelighed, hvorfor Bitcoin bør behandles som finansiel 
software på atomniveau.

Hvor er vi i denne finansielle, samfundsmæssige og teknologiske revolution? 
Netværk og teknologier fra fortiden kan tjene som metaforer for Bitcoins 
fremtid, hvilket udforskes i den sidste lektion af dette kapitel.

Endnu engang, spænd sikkerhedsselen og nyd turen. Som alle eksponentielle 
teknologier, er vi ved at gå parabolisk.