\addpart{Afsluttende tanker}
\pdfbookmark{Konklusion}{konklusion}
\label{ch:konklusion}

\chapter*{Konklusion}

\begin{chapquote}{Lewis Carroll, \textit{Alice i Eventyrland}}
\enquote{Begynd ved begyndelsen,} sagde Kongen meget alvorligt, \enquote{og 
fortsæt, indtil du kommer til enden: så stop.}
\end{chapquote}

Som nævnt i begyndelsen mener jeg, at enhver besvarelse på spørgsmålet 
\textit{“Hvad har du lært af Bitcoin?”} altid vil være ufuldstændig. Symbiosen 
af det, der kan betragtes som flere levende systemer - Bitcoin, teknosfæren og 
økonomi - er for indviklet, emnerne for mange, og ting bevæger sig for hurtigt 
til nogensinde at blive fuldt forstået af én person.

Selv uden fuld forståelse og selv med alle dens særheder og tilsyneladende 
mangler fungerer Bitcoin utvivlsomt. Den fortsætter med at producere blokke 
cirka hvert tiende minut og gør det smukt. Jo længere Bitcoin fortsætter med at 
fungere, desto flere mennesker vil vælge at bruge det.

\begin{quotation}\begin{samepage}
\enquote{Det er sandt, at ting er smukke, når de virker. Kunst er funktion.}
\begin{flushright} -- Giannina Braschi\footnote{Giannina Braschi, \textit{Empire of Dreams} \cite{braschi2011empire}}
\end{flushright}\end{samepage}\end{quotation}

\paragraph{} Bitcoin er et barn af internettet. Det vokser eksponentielt, 
udvisker grænserne mellem discipliner. Det er ikke klart, for eksempel, hvor 
den rene teknologis rige slutter, og hvor en anden sfære begynder. Selvom 
Bitcoin kræver computere for at fungere effektivt, er datalogi ikke 
tilstrækkelig til at forstå det. Bitcoin er ikke kun grænseløs med hensyn til 
dets indre funktioner, men også grænseløs med hensyn til akademiske discipliner.

Økonomi, politik, spilteori, pengehistorie, netværksteori, finans, kryptografi, 
informationsteori, censur, lov og regulering, menneskelig organisation, 
psykologi - alle disse og mere er områder af ekspertise, der måske kan hjælpe 
med at forstå, hvordan Bitcoin fungerer og hvad Bitcoin er.

Ingen enkelt opfindelse er ansvarlig for dens succes. Det er kombinationen af 
flere, tidligere ikke-relaterede dele, limet sammen af spilteoretiske 
incitamenter, der udgør revolutionen, som er Bitcoin. Den smukke blanding af 
mange discipliner er det, der gør Satoshi til en geni.

\paragraph{} Ligegyldigt hvor komplekst et system er, skal Bitcoin træffe 
kompromisser med hensyn til effektivitet, omkostning, sikkerhed og mange andre 
egenskaber. Ligesom der ikke findes en perfekt løsning på at udlede en firkant 
fra en cirkel, vil enhver løsning på de problemer, som Bitcoin forsøger at 
løse, altid være ufuldkommen.

\begin{quotation}\begin{samepage}
\enquote{Jeg tror ikke, vi nogensinde vil have gode penge igen, før vi tager
tingen ud af regeringens hænder, det vil sige, vi kan ikke tage det med
vold ud af regeringens hænder, alt hvad vi kan gøre, er på en snedig omvej
introducere noget, de ikke kan stoppe.}
\begin{flushright} -- Friedrich Hayek\footnote{Friedrich Hayek om pengepolitik, 
    guldfoden, underskud, inflation og John Maynard Keynes 
    \url{https://youtu.be/EYhEDxFwFRU}}
\end{flushright}\end{samepage}\end{quotation}

Bitcoin er den snedige, omvejsfulde måde at genindføre gode penge i verden på. 
Den gør det ved at placere en suveræn individ bag hver node, præcis som Da 
Vinci forsøgte at løse det uløselige problem med at kvadrere en cirkel ved at 
placere Vitruvian Man i dens centrum. Noder fjerner effektivt enhver opfattelse 
af et centrum og skaber et system, der er forbløffende antifragilt og ekstremt 
svært at lukke ned. Bitcoin lever, og dens hjerteslag vil sandsynligvis 
overleve os alle.

Jeg håber, du har nydt disse enogtyve lektioner. Måske er den mest vigtige 
lektion, at Bitcoin bør undersøges holistisk, fra flere vinkler, hvis man gerne 
vil have noget, der nærmer sig et komplet billede. Ligesom at fjerne én del 
fra et komplekst system ødelægger det hele, synes at undersøge dele af Bitcoin 
isoleret at forurene forståelsen af det. Hvis blot én person fjerner 
\enquote{blockchain} fra sit ordforråd og erstatter det med 
\enquote{en kæde af blokke}, vil jeg dø en lykkelig mand.

Under alle omstændigheder fortsætter min rejse. Jeg planlægger at bevæge mig 
dybere ned i dette kaninhul, og jeg inviterer dig til at følge med
på turen.\footnote{\url{https://twitter.com/dergigi}}

% <!-- Twitter -->
% [dergigi]: https://twitter.com/dergigi
%
% <!-- Internal -->
% [sly roundabout way]: https://youtu.be/EYhEDxFwFRU?t=1124
% [Giannina Braschi]: https://en.wikipedia.org/wiki/Braschi%27s_Empire_of_Dreams
