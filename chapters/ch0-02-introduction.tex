\chapter*{Introduktion}
\label{ch:introduktion}

\begin{chapquote}{Lewis Carroll, \textit{Alice i Eventyrland}}
\enquote{Men jeg vil ikke blandt gale mennesker,} bemærkede Alice. 
\enquote{Oh, du kan ikke hjælpe det,} sagde Katten: \enquote{vi er alle gale 
her. Jeg er gal. Du er gal.} \enquote{Hvordan ved du, at jeg er gal?} sagde 
Alice. \enquote{Du må være det,} sagde Katten, \enquote{ellers ville du ikke
være kommet her.}
\end{chapquote}

I oktober 2018 stillede Arjun Balaji det uskyldige spørgsmål,
\textit{Hvad har du lært af Bitcoin?} Efter at have forsøgt at besvare dette
spørgsmål i en kort tweet og fejlet miserabelt, indså jeg, at de ting
jeg har lært, er alt for talrige til at besvare hurtigt, hvis overhovedet.

De ting, jeg har lært, handler selvfølgelig om Bitcoin - eller i det mindste 
er de relateret til det. Dog, mens nogle af de indre arbejdsmetoder i Bitcoin 
bliver forklaret, er de følgende lektioner ikke en forklaring på, hvordan 
Bitcoin fungerer, eller hvad det er, de kan dog hjælpe med at udforske nogle 
af de ting, Bitcoin berører: filosofiske spørgsmål, økonomiske realiteter og 
teknologiske innovationer.

\begin{center}
  \includegraphics[width=7cm]{assets/images/the-tweet.png}
\end{center}

De \textit{21 Lektioner} er struktureret i bundter af syv, resulterende i tre
kapitler. Hvert kapitel ser på Bitcoin gennem en anden linse og udvinder
hvilke lektioner der kan læres ved at undersøge dette mærkelige netværk fra en 
anden vinkel.

\paragraph{Kapitel 1 \hyperref[ch:philosophy]{Philosophy}} 
Udforsker de filosofiske lærdomme fra Bitcoin. Samspillet mellem 
uforanderlighed og forandring, begrebet sand knaphed, Bitcoins pletfri 
undfangelse, problemet med identitet, modsigelsen af replikation og lokalitet, 
ytringsfrihedens magt og vidensgrænserne.

\paragraph{Kapitel 2 \hyperref[ch:economics]{Economics}}
Udforsker de økonomiske lærdomme fra Bitcoin. Lektioner om økonomisk uvidenhed, 
inflation, værdi, penge og pengenes historie, fractional reserve banking og 
hvordan Bitcoin genindfører lydpenge på en snedig, indirekte måde.

\paragraph{Kapitel 3 \hyperref[ch:technology]{Technology}} 
Udforsker nogle af de lektioner, der er lært ved at undersøge teknologien i 
Bitcoin. Hvorfor der er styrke i antal, refleksioner over tillid, hvorfor at 
fortælle tiden kræver arbejde, hvordan at bevæge sig langsomt og ikke ødelægge 
ting er en funktion og ikke en fejl, hvad Bitcoin's skabelse kan fortælle os om
privatliv, hvorfor cypherpunks skriver kode (og ikke love), og hvilke metaforer 
der kan være nyttige at udforske Bitcoins fremtid.

~

Hver lektion indeholder adskillige citater og links i teksten. Hvis en idé er
værd at udforske nærmere, kan du følge links til relaterede værker i
fodnoterne eller i bibliografien.

Selvom noget forhåndsviden om Bitcoin er gavnligt, håber jeg, at disse
lektioner kan fordøjes af enhver nysgerrig læser. Mens nogle relaterer sig til 
hinanden, bør hver lektion være i stand til at stå på egen hånd og kan læses 
uafhængigt. Jeg gjorde mit bedste for at undgå teknisk jargon, selvom noget 
domænespecifikt ordforråd er uundgåeligt.

Jeg håber, at min skrivning tjener som inspiration for andre til at grave under
overfladen og undersøge nogle af de dybere spørgsmål, Bitcoin rejser. Min egen
inspiration kom fra en mangfoldighed af forfattere og indholdsproducenter, til 
hvem jeg er evigt taknemmelig.

Sidst men ikke mindst: mit mål med at skrive dette er ikke at overbevise dig om 
noget. Mit mål er at få dig til at tænke og vise dig, at der er meget mere ved 
Bitcoin end det, der møder øjet. Jeg kan ikke engang fortælle dig, hvad Bitcoin 
er, eller hvad Bitcoin vil lære dig. Det bliver du nødt til at finde ud af selv.

\begin{quotation}\begin{samepage}
\enquote{Efter dette er der ingen vej tilbage. Du tager den blå pille --- 
historien ender, du vågner op i din seng og tror på hvad som helst du vil tro. 
Du tager den røde pille\footnote{den \textit{orange} pille} --- du bliver i 
Wonderland, og jeg viser dig, hvor dybt kaninhullet går.}
\begin{flushright} -- Morpheus
\end{flushright}\end{samepage}\end{quotation}

\begin{center}
  \includegraphics[width=\textwidth]{assets/images/bitcoin-orange-pill.jpg}
  \captionof{figure}*{Husk: Alt, jeg tilbyder, er sandheden. Intet mere.}
  \label{fig:bitcoin-orange-pill}
\end{center}

%
% [Morpheus]: https://en.wikipedia.org/wiki/Red_pill_and_blue_pill#The_Matrix_(1999)
% [this question]: https://twitter.com/arjunblj/status/1050073234719293440
%
% <!-- Internal -->
% [chapter1]: {{ 'bitcoin/lessons/ch1-00-philosophy' | absolute_url }}
% [chapter2]: {{ 'bitcoin/lessons/ch2-00-economics' | absolute_url }}
% [chapter3]: {{ 'bitcoin/lessons/ch3-00-technology' | absolute_url }}
%
% <!-- Wikipedia -->
% [alice]: https://en.wikipedia.org/wiki/Alice%27s_Adventures_in_Wonderland
% [carroll]: https://en.wikipedia.org/wiki/Lewis_Carroll
