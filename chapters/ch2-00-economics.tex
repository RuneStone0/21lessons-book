\part{Økonomi}
\label{ch:økonomi}
\chapter*{Økonomi}

\begin{chapquote}{Lewis Carroll, \textit{Alice i Eventyrland}}
\enquote{Et stort rosetræ stod nær indgangen til haven: roserne på det var 
hvide, men der var tre gartnere ved det, travlt beskæftiget med at male dem 
røde. Dette tænkte Alice var en meget mærkelig ting...}
\end{chapquote}

Penge vokser ikke på træer. At tro det er dumt, og vores forældre sørger for, 
at vi ved det ved at gentage denne sætning som en mantra. Vi opfordres til at 
bruge penge klogt, ikke at bruge dem tankeløst og spare dem i gode tider for at 
hjælpe os gennem de dårlige. Penge vokser trods alt ikke på træer.

Bitcoin lærte mig mere om penge, end jeg nogensinde troede, jeg ville have brug 
for at vide. Gennem det blev jeg tvunget til at udforske historien om penge, 
bankvæsen, forskellige skoler inden for økonomisk tænkning og mange andre ting. 
Jagten på at forstå Bitcoin førte mig ned ad en overflod af stier, nogle af dem 
forsøger jeg at udforske i dette kapitel.

I de første syv lektioner blev nogle af de filosofiske spørgsmål, Bitcoin 
berører, diskuteret. De næste syv lektioner vil se nærmere på penge og økonomi.

~

\begin{samepage}
Del~\ref{ch:økonomi} -- Økonomi:

\begin{enumerate}
  \setcounter{enumi}{7}
  \item Finansiel uvidenhed
  \item Inflation
  \item Værdi
  \item Penge
  \item Historien og faldet af penge
  \item Fractional reserve insanity
  \item Lyd penge
\end{enumerate}
\end{samepage}

Igen vil jeg kun kunne ridse overfladen. Bitcoin er ikke kun ambitiøs, men også 
bred og dyb i omfang, hvilket gør det umuligt at dække alle relevante emner i en 
enkelt lektion, artikel eller bog. Jeg tvivler endda på, om det er muligt 
overhovedet.

Bitcoin er en ny form for penge, hvilket gør læring om økonomi afgørende for at 
forstå det. At beskæftige sig med menneskelig handling og interaktionerne mellem 
økonomiske agenter er sandsynligvis et af de største og mest uklare elementer i 
Bitcoin-puslespillet.

Igen er disse lektioner en udforskning af de forskellige ting, jeg har lært af 
Bitcoin. De er en personlig refleksion over min rejse ned i kaninhullet. Uden 
baggrund inden for økonomi er jeg helt sikkert uden for min komfortzone og 
særlig opmærksom på, at enhver forståelse, jeg måtte have, er ufuldstændig. Jeg 
vil gøre mit bedste for at skitsere, hvad jeg har lært, selvom det indebærer 
risikoen for at gøre mig selv til nar. Trods alt forsøger jeg stadig at besvare 
spørgsmålet: \textit{\enquote{Hvad har du lært af Bitcoin?}}

Efter syv lektioner undersøgt gennem filosofiens linse, lad os bruge økonomiens 
linse til at se på syv mere. Økonomiklasse er alt, hvad jeg kan tilbyde denne 
gang. Endelig destination: \textit{lyd penge}.

% [the question]: https://twitter.com/arjunblj/status/1050073234719293440
