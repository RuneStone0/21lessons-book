\chapter{Sjældenhed}
\label{les:2}

\begin{chapquote}{Alice}
\enquote{Det er nok nu - jeg håber, jeg ikke vokser mere\ldots}
\end{chapquote}

Generelt set ser det ud til, at teknologiens fremskridt gør ting mere 
tilgængelige. Flere og flere mennesker har mulighed for at nyde det, der 
tidligere har været luksusvarer. Snart vil vi alle leve som konger. De fleste 
af os gør det allerede. Som Peter Diamandis skrev i Abundance~\cite{abundance}: 
\enquote{Teknologi er en ressourcebefriende mekanisme. Den kan gøre det, der 
engang var sjældent, nu er let tilgængeligt.}

Bitcoin, er en avanceret teknologi der bryde tendenser og skaber en ny vare 
der er ægte sæjldnehed. Nogle argumenterer endda for, at det er en af 
de sjældneste ting i universet. Antallet af Bitcoins kan ikke udvides, 
uanset hvor meget man end forsøger at skabe flere.

\begin{quotation}\begin{samepage}
\enquote{Kun to ting er ægte sæjldne: tid og bitcoin.}
\begin{flushright} -- Saifedean Ammous\footnote{Præsentation om The Bitcoin 
    Standard~\cite{bitcoinstandard-pres}}
\end{flushright}\end{samepage}\end{quotation}

Paradoksalt nok gør den dette ved hjælp af en mekanisme af kopiering. 
Transaktioner bliver sendt ud, blokke bliver spredt, den distribuerede logbog 
er --- tja, du gættede det --- distribueret. Alle disse er bare fancy ord for 
kopiering. Faktisk kopierer Bitcoin endda sig selv til så mange computere som 
muligt ved at tilbyde enkeltpersoner at køre fulde knudepunkter og udvinde 
nye blokke.

Alt dette duplikations arbejde er en koordineret indsats for at producere 
ægte sæjldnehed.

\paragraph{I en tid med overflod lærte Bitcoin mig, hvad ægte sæjldnehed er.}

% ---
%
% #### Through the Looking-Glass
%
% - [Lesson 14: Sound money][lesson14]
%
% #### Down the Rabbit Hole
%
% - [The Bitcoin Standard: The Decentralized Alternative to Central Banking][bitcoin-standard]
% - [Abundance: The Future Is Better Than You Think][Abundance] by Peter Diamandis
% - [Presentation on The Bitcoin Standard][bitcoin-standard-presentation] by Saifedean Ammous
% - [Modeling Bitcoin's Value with Scarcity][planb-scarcity] by PlanB
% - 🎧 [Misir Mahmudov on the Scarcity of Time & Bitcoin][tftc60] TFTC #60 hosted by Marty Bent
% - 🎧 [PlanB – Modelling Bitcoin's digital scarcity through stock-to-flow techniques][slp67] SLP #67 hosted by Stephan Livera
%
% <!-- Through the Looking-Glass -->
% [lesson14]: {{ 'bitcoin/lessons/ch2-14-sound-money' | absolute_url }}
%
% <!-- Down the Rabbit Hole -->
% [Abundance]: https://www.diamandis.com/abundance
% [bitcoin-standard]: http://amzn.to/2L95bJW
% [bitcoin-standard-presentation]: https://www.bayernlb.de/internet/media/de/ir/downloads_1/bayernlb_research/sonderpublikationen_1/bitcoin_munich_may_28.pdf
% [planb-scarcity]: https://medium.com/@100trillionUSD/modeling-bitcoins-value-with-scarcity-91fa0fc03e25
% [tftc60]: https://anchor.fm/tales-from-the-crypt/episodes/Tales-from-the-Crypt-60-Misir-Mahmudov-e3aibh
% [slp67]: https://stephanlivera.com/episode/67
%
% <!-- Wikipedia -->
% [alice]: https://en.wikipedia.org/wiki/Alice%27s_Adventures_in_Wonderland
% [carroll]: https://en.wikipedia.org/wiki/Lewis_Carroll
