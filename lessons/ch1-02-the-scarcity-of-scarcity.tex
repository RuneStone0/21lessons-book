
\chapter{Knapheden på Knaphed}
\label{les:2}

\begin{chapquote}{Alice}
\enquote{Det er helt nok - jeg håber, jeg ikke vokser mere\ldots}
\end{chapquote}

Generelt set ser det ud til, at teknologiens fremskridt gør ting mere tilgængelige. Flere og flere mennesker har mulighed for at nyde det, der tidligere har været luksusvarer. Snart vil vi alle leve som konger. De fleste af os gør det allerede. Som Peter Diamandis skrev i Abundance~\cite{abundance}: \enquote{Teknologi er en ressourcebefriende mekanisme. Den kan gøre det, der engang var knapt, nu overflod.}

Bitcoin, i sig selv en avanceret teknologi, bryder denne tendens og skaber en ny vare, der virkelig er knap. Nogle argumenterer endda for, at det er en af de sjældneste ting i universet. Udbuddet kan ikke oppustes, uanset hvor meget indsats man vælger at investere i at skabe mere.

\begin{quotation}\begin{samepage}
\enquote{Kun to ting er ægte knappe: tid og bitcoin.}
\begin{flushright} -- Saifedean Ammous\footnote{Præsentation om The Bitcoin Standard~\cite{bitcoinstandard-pres}}
\end{flushright}\end{samepage}\end{quotation}

Paradoksalt nok gør den dette ved hjælp af en mekanisme af kopiering. Transaktioner bliver sendt ud, blokke bliver spredt, den distribuerede hovedbog er --- tja, du gættede det --- distribueret. Alle disse er bare fancy ord for kopiering. Faktisk kopierer Bitcoin endda sig selv til så mange computere som muligt ved at tilskynde enkeltpersoner til at køre fulde knudepunkter og mine nye blokke.

Alt dette duplikation arbejder vidunderligt sammen i en samordnet indsats for at producere knaphed.

\paragraph{I en tid med overflod lærte Bitcoin mig, hvad ægte knaphed er.}

% ---
%
% #### Through the Looking-Glass
%
% - [Lesson 14: Sound money][lesson14]
%
% #### Down the Rabbit Hole
%
% - [The Bitcoin Standard: The Decentralized Alternative to Central Banking][bitcoin-standard]
% - [Abundance: The Future Is Better Than You Think][Abundance] by Peter Diamandis
% - [Presentation on The Bitcoin Standard][bitcoin-standard-presentation] by Saifedean Ammous
% - [Modeling Bitcoin's Value with Scarcity][planb-scarcity] by PlanB
% - 🎧 [Misir Mahmudov on the Scarcity of Time & Bitcoin][tftc60] TFTC #60 hosted by Marty Bent
% - 🎧 [PlanB – Modelling Bitcoin's digital scarcity through stock-to-flow techniques][slp67] SLP #67 hosted by Stephan Livera
%
% <!-- Through the Looking-Glass -->
% [lesson14]: {{ 'bitcoin/lessons/ch2-14-sound-money' | absolute_url }}
%
% <!-- Down the Rabbit Hole -->
% [Abundance]: https://www.diamandis.com/abundance
% [bitcoin-standard]: http://amzn.to/2L95bJW
% [bitcoin-standard-presentation]: https://www.bayernlb.de/internet/media/de/ir/downloads_1/bayernlb_research/sonderpublikationen_1/bitcoin_munich_may_28.pdf
% [planb-scarcity]: https://medium.com/@100trillionUSD/modeling-bitcoins-value-with-scarcity-91fa0fc03e25
% [tftc60]: https://anchor.fm/tales-from-the-crypt/episodes/Tales-from-the-Crypt-60-Misir-Mahmudov-e3aibh
% [slp67]: https://stephanlivera.com/episode/67
%
% <!-- Wikipedia -->
% [alice]: https://en.wikipedia.org/wiki/Alice%27s_Adventures_in_Wonderland
% [carroll]: https://en.wikipedia.org/wiki/Lewis_Carroll
