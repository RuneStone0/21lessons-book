\chapter{Bevæg Dig Langsomt og Ødelæg Ikke Ting}
\label{les:18}

\begin{chapquote}{Lewis Carroll, \textit{Alice i Eventyrland}}
Så båden snoede sig langsomt afsted, under den klare sommerdag, med dens muntre 
besætning og musik af stemmer og latter\ldots
\end{chapquote}

Det kan være en død mantra, men \enquote{bevæg dig hurtigt og ødelæg ting} er 
stadig, hvordan stor del af tech-verdenen opererer. Ideen om, at det ikke
betyder noget, hvis du får tingene rigtige første gang, er en grundlæggende 
søjle i \textit{fejl tidligt, fejl ofte} mentaliteten. Succes måles i vækst, 
så længe du vokser, er alt fint. Hvis noget ikke virker første gang, så
drejer og itererer du bare. Med andre ord: kast nok ting mod
væggen og se, hvad der bliver hængende.

Bitcoin er meget anderledes. Den er anderledes af design. Den er anderledes
af nødvendighed. Som Satoshi påpegede, er e-valuta blevet forsøgt
mange gange før, og alle tidligere forsøg er mislykkedes, fordi der
var et hoved, der kunne skæres af. Nyheden ved Bitcoin er, at det er
et uhyre uden hoveder.

\begin{quotation}\begin{samepage}
\enquote{Mange mennesker afskriver automatisk e-valuta som en tabt sag
på grund af alle virksomhederne, der er fejlet siden 1990'erne. Jeg håber, det 
er åbenlyst, at det kun var den centraliserede karakter af de systemer
der fordømte dem.}
\begin{flushright} -- Satoshi Nakamoto\footnote{Satoshi Nakamoto, i et svar 
    til Sepp Hasslberger \cite{satoshi-centralized-nature}}
\end{flushright}\end{samepage}\end{quotation}

En konsekvens af denne radikale decentralisering er en indbygget modstand 
mod forandring. \enquote{Bevæg dig hurtigt og ødelæg ting} fungerer ikke 
og vil aldrig fungere på Bitcoin's basale lag. Selvom det ville være 
ønskeligt, ville det ikke være muligt uden at overbevise \textit{alle} 
om at ændre deres måder. Det er distribueret konsensus. Det er naturen 
af Bitcoin.

\begin{quotation}\begin{samepage}
\enquote{Bitcoin's natur er sådan, at når version 0.1 blev frigivet, var
kerne designet hugget i sten for resten af dets levetid.}
\begin{flushright} -- Satoshi Nakamoto\footnote{Satoshi Nakamoto, i et svar 
    til Gavin Andresen \cite{satoshi-centralized-nature}}
\end{flushright}\end{samepage}\end{quotation}

Dette er en af de mange paradoksale egenskaber ved Bitcoin. Vi kom alle
til at tro, at alt, der er software, kan ændres let. Men
uhyrets natur gør det vanskeligt at ændre det.

Som Hasu smukt viser i Unpacking Bitcoin's Social
Contract~\cite{social-contract}, er det kun muligt at ændre Bitcoin's regler
ved at \textit{forslå} en ændring og dermed \textit{overtale} alle brugere
af Bitcoin til at acceptere denne ændring. Dette gør Bitcoin meget 
modstandsdygtig over for forandring,
selvom det er software.

Denne modstandsdygtighed er en af de vigtigste egenskaber ved Bitcoin.
Kritiske software-systemer skal være antifragile, hvilket er det
samspil, Bitcoin's sociale lag og dets tekniske lag garanterer.
Pengesystemer er af natur fjendtlige, og som vi har vidst i
tusinder af år, er solide grundlag afgørende i en fjendtlig
miljø.

\begin{quotation}\begin{samepage}
    \enquote{Regnen faldt, oversvømmelserne kom, og vinden blæste og slog på
    det hus; og det faldt ikke, for det var bygget på klippen.}
    \begin{flushright} -- Matthæus 7:24--27
\end{flushright}\end{samepage}\end{quotation}

Argumentabelt set er Bitcoin i denne lignelse om de kloge og tåbelige byggere
ikke huset. Det er klippen. Uforanderlig, uhældig, der giver grundlaget for et 
nyt økonomisk system.

Præcis som geologer, der ved, at klippeformationer altid bevæger sig
og udvikler sig, kan man se, at Bitcoin altid bevæger sig og udvikler sig også.
Du skal bare vide, hvor du skal kigge, og hvordan du skal se på det.

Introduktionen af pay to script hash\footnote{ Pay to script hash (P2SH)
transaktioner blev standardiseret i BIP 16. De tillader, at transaktioner 
sendes til en script hash (adresse der starter med 3) i stedet for en public 
key hash (adresser der starter med 1).~\cite{btcwiki:p2sh}} og segregate
witness\footnote{Segregated Witness (forkortet som SegWit) er en implementeret
protokolopgradering, der har til formål at beskytte mod 
transaktionsmalleabilitet og øge blokkapaciteten. SegWit adskiller 
\textit{vidnet} fra listen af inputs.~\cite{btcwiki:segwit}} er bevis på,
at Bitcoin's regler kan ændres, hvis tilstrækkeligt mange brugere er 
overbeviste om, at vedtagelse af denne ændring er til fordel for netværket. 
Den sidstnævnte muliggjorde udviklingen af lynnetværket 
\footnote{\url{https://lightning.network/}}, som er et af husene, der bliver 
bygget på Bitcoin's solide fundament. Fremtidige opgraderinger som Schnorr
signaturer~\cite{bip:schnorr} vil forbedre effektiviteten og privatlivet, 
såvel som scripts (læs: smarte kontrakter), der vil være uundskillelige fra 
almindelige transaktioner takket være Taproot~\cite{taproot}. Kloge byggere 
bygger virkelig på solide grundlag.

Satoshi var ikke kun en klog bygger teknologisk set. Han forstod også, at det 
ville være nødvendigt at træffe kloge beslutninger ideologisk set.

\begin{quotation}\begin{samepage}
    \enquote{At være åben kilde betyder, at enhver uafhængigt kan gennemgå 
    koden. Hvis den var lukket kilde, ville ingen kunne verificere 
    sikkerheden. Jeg synes, det er essentielt for et program af denne art at 
    være åben kilde.}
    \begin{flushright} -- Satoshi Nakamoto\footnote{Satoshi Nakamoto, i et 
        svar til SmokeTooMuch \cite{satoshi-open-source}}
\end{flushright}\end{samepage}\end{quotation}

Åbenhed er afgørende for sikkerhed og indlejret i åben kilde og
free software-bevægelsen. Som Satoshi påpegede, skal sikre protokoller og
den kode, der implementerer dem, være åbne --- der er ingen sikkerhed
gennem obskuritet. En anden fordel er igen relateret til decentralisering:
kode, der frit kan køres, studeres, ændres, kopieres og distribueres
sikrer, at den spredes vidt og bredt.

Bitcoin's radikalt decentraliserede karakter er det, der får det til at 
bevæge sig langsomt og med overvejelse. Et netværk af noder, hver kørt af 
en suveræn individ, er i sig selv modstandsdygtig over for ændringer --- 
ondsindede eller ej. Uden mulighed for at tvinge opdateringer på brugerne er 
den eneste måde at introducere ændringer på ved langsomt at overbevise hver 
eneste af disse individer om at acceptere en ændring. Denne ikke-centrale 
proces med at introducere og implementere ændringer er det, der gør nettet 
utroligt modstandsdygtigt over for ondsindede ændringer. Det er også det, 
der gør det sværere at rette ødelagte ting end i en centraliseret miljø, 
hvilket er grunden til, at alle prøver ikke at ødelægge ting i første omgang.

\paragraph{Bitcoin lærte mig, at bevæge sig langsomt er en af dets 
funktioner, ikke en fejl.}

% ---
%
% #### Through the Looking-Glass
%
% - [Lesson 1: Immutability and Change][lesson1]
%
% #### Down the Rabbit Hole
%
% - [Unpacking Bitcoin's Social Contract] by Hasu
% - [Schnorr signatures BIP][Schnorr signatures] by Pieter Wuille
% - [Taproot proposal][Taproot] by Gregory Maxwell
% - [P2SH][pay to script hash], [SegWit][segregated witness] on the Bitcoin Wiki
% - [Parable of the Wise and the Foolish Builders][Matthew 7:24--27] on Wikipedia
%
% <!-- Down the Rabbit Hole -->
% [lesson1]: {{ '/bitcoin/lessons/ch1-01-immutability-and-change' | absolute_url }}
%
% [Unpacking Bitcoin's Social Contract]: https://uncommoncore.co/unpacking-bitcoins-social-contract/
% [Matthew 7:24--27]: https://en.wikipedia.org/wiki/Parable_of_the_Wise_and_the_Foolish_Builders
% [pay to script hash]: https://en.bitcoin.it/wiki/Pay_to_script_hash
% [segregated witness]: https://en.bitcoin.it/wiki/Segregated_Witness
% [lightning network]: https://lightning.network/
% [Schnorr signatures]: https://github.com/sipa/bips/blob/bip-schnorr/bip-schnorr.mediawiki#cite_ref-6-0
% [Taproot]: https://lists.linuxfoundation.org/pipermail/bitcoin-dev/2018-January/015614.html
%
% <!-- Wikipedia -->
% [alice]: https://en.wikipedia.org/wiki/Alice%27s_Adventures_in_Wonderland
% [carroll]: https://en.wikipedia.org/wiki/Lewis_Carroll
