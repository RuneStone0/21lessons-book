\chapter{Inflation}
\label{les:9}

\begin{chapquote}{Hjerterdronningen}
\enquote{Kære, her skal vi løbe så hurtigt, som vi kan, bare for at blive på 
stedet. Og hvis du ønsker at komme et sted hen, skal du løbe dobbelt så 
hurtigt som det.}
\end{chapquote}

At forsøge at forstå pengeinflation og hvordan et ikke-inflationært system 
som Bitcoin kunne ændre vores tilgang til tingene, var startpunktet for mit 
dyk ned i økonomien. Jeg vidste, at inflation var satsen, hvormed der blev 
skabt nye penge, men jeg vidste ikke meget ud over det.

Mens nogle økonomer argumenterer for, at inflation er en god ting, hævder 
andre, at \enquote{hårde} penge, der ikke let kan infleres - som vi havde 
det under guldstandarden - er essentielle for en sund økonomi. Bitcoin, der 
har et fast udbud på 21 millioner, er enig med sidstnævnte lejr.

Normalt er virkningerne af inflation ikke umiddelbart åbenlyse. Afhængigt af 
inflationsraten (samt andre faktorer) kan tiden mellem årsag og virkning være 
adskillige år. Ikke kun det, men inflation påvirker forskellige grupper af
mennesker mere end andre. Som Henry Hazlitt påpeger i \textit{Økonomi på ét 
minut}: \enquote{Økonomiens kunst består i ikke kun at se på det øjeblikkelige, 
men på de længerevarende virkninger af enhver handling eller politik; det 
består i at spore konsekvenserne af den politik ikke kun for én gruppe, men 
for alle grupper.}

Et af mine personlige øjeblikke af forståelse var erkendelsen af, at udstedelse 
af ny valuta - at trykke flere penge - er en \textit{fuldstændig} anderledes
økonomisk aktivitet end alle andre økonomiske aktiviteter. Mens reelle varer 
og reelle tjenester producerer reel værdi for virkelige mennesker, gør trykning 
af penge effektivt det modsatte: det tager værdi væk fra alle, der holder den 
inflerede valuta.

\begin{quotation}\begin{samepage}
\enquote{Bare inflation - det vil sige, bare udstedelse af flere penge, med 
konsekvensen af højere lønninger og priser - kan ligne skabelsen af mere 
efterspørgsel. Men i forhold til den faktiske produktion og udveksling af 
virkelige ting er det det ikke.} \begin{flushright} -- Henry Hazlitt
    \footnote{Henry Hazlitt, \textit{Økonomi på ét minut} \cite{hazlitt}}
\end{flushright}\end{samepage}\end{quotation}

Inflationens ødelæggende kraft bliver tydelig, så snart lidt inflation bliver 
til \textit{meget}. Hvis penge hyperinflates, bliver tingene grimme meget 
hurtigt.\footnote{\url{https://en.wikipedia.org/wiki/Hyperinflation}
\cite{wiki:hyperinflation}} Når den inflerende valuta falder fra hinanden, 
vil den ikke være i stand til at bevare værdi over tid, og folk vil skynde 
sig at få fat i varer, der måske kan det.

\paragraph{}
En anden konsekvens af hyperinflation er, at al den opsparing, som folk har 
samlet gennem deres liv, effektivt vil forsvinde. Pengesedlerne i din tegnebog 
vil stadig være der, selvfølgelig. Men det vil være præcis det: værdiløst papir.

\begin{center}
\includegraphics[width=\textwidth]{assets/images/children-playing-with-money.png}
\captionof{figure}{Hyperinflation i Weimar-republikken (1921-1923)}
\label{fig:children-playing-with-money}
\end{center}

\paragraph{}
Penge mister værdi med såkaldt \enquote{mild} inflation også. Det sker bare 
langsomt nok til, at de fleste ikke bemærker den svækkelse af deres købekraft.
Og når trykpresserne kører, kan valuta let infleres, og hvad der plejede at 
være mild inflation, kan blive til en kraftig portion inflation med et tryk på 
en knap. Som Friedrich Hayek påpegede i et af sine essays, fører mild inflation 
normalt til åbenlys inflation.

\begin{quotation}\begin{samepage}
\enquote{'Mild' stabil inflation kan ikke hjælpe - den kan kun føre til åbenlys
inflation.} \begin{flushright} -- Friedrich Hayek\footnote{Friedrich Hayek, 
    \textit{1980'erne Arbejdsløshed og fagforeningerne} \cite{hayek-inflation}}
\end{flushright}\end{samepage}\end{quotation}

Inflation er særlig snedig, fordi den favoriserer dem, der er tættere på 
trykpresserne. Det tager tid, før de nyoprettede penge cirkulerer, og priserne 
justeres, så hvis du kan få fat i flere penge, før alle andres mister værdi, 
er du foran inflationskurven. Det er også derfor, inflation kan ses som en 
skjult skat, fordi regeringerne til sidst tjener på det, mens alle andre ender 
med at betale prisen.

\begin{quotation}\begin{samepage}
\enquote{Jeg mener ikke, det er en overdrivelse at sige, at historien i vid 
udstrækning er en historie om inflation, og normalt om inflationer, der er 
orkestreret af regeringer til fordel for regeringer.}
\begin{flushright} -- Friedrich Hayek\footnote{Friedrich Hayek, \textit{Godt Penge} \cite{hayek-good-money}}
\end{flushright}\end{samepage}\end{quotation}

\newpage

So far, all government-controlled currencies have eventually been
replaced or have collapsed completely. No matter how small the rate of
inflation, \enquote{steady} growth is just another way of saying exponential
growth. In nature as in economics, all systems which grow exponentially
will eventually have to level off or suffer from catastrophic collapse.

\paragraph{}
\enquote{It can't happen in my country,} is what you're probably thinking. 
You don't think that if you are from Venezuela, which is currently suffering 
from hyperinflation. With an inflation rate of over 1 million percent, money is
basically worthless. \cite{wiki:venezuela}

\paragraph{}
It might not happen in the next couple of years, or to the particular currency
used in your country. But a glance at the list of historical
currencies\footnote{See \textit{List of historical currencies} on Wikipedia.
\cite{wiki:historical-currencies}} shows that it will inevitably happen over a
long enough period of time. I remember and used plenty of those listed: the
Austrian schilling, the German mark, the Italian lira, the French franc, the
Irish pound, the Croatian dinar, etc. My grandma even used the Austro-Hungarian
Krone. As time moves on, the currencies currently in use\footnote{See
\textit{List of currencies} on Wikipedia \cite{wiki:list-of-currencies}} will
slowly but surely move to their respective graveyards. They will hyperinflate or
be replaced. They will soon be historical currencies. We will make them
obsolete.

\begin{quotation}\begin{samepage}
\enquote{History has shown that governments will inevitably succumb to the
temptation of inflating the money supply.}
\begin{flushright} -- Saifedean Ammous\footnote{Saifedean Ammous, 
    \textit{The Bitcoin
Standard} \cite{bitcoin-standard}}
\end{flushright}\end{samepage}\end{quotation}

\newpage

Why is Bitcoin different? In contrast to currencies mandated by the government,
monetary goods which are not regulated by governments, but by the laws of
physics\footnote{Gigi, \textit{Bitcoin's Energy Consumption - A shift in
perspective} \cite{gigi:energy}}, tend to survive and even hold their respective
value over time. The best example of this so far is gold, which, as the
aptly-named \textit{Gold-to-Decent-Suit Ratio}\footnote{History shows that the
price of an ounce of gold equals the price of a decent men's suit, according to 
Sionna investment managers \cite{web:gold-to-decent-suite-ratio}} shows, is 
holding its value over hundreds and even thousands of years. It might not be 
perfectly \enquote{stable} --- a questionable concept in the first place --- 
but the value it holds will at least be in the same order of magnitude.

If a monetary good or currency holds its value well over time and space,
it is considered to be \textit{hard}. If it can't hold its value, because it
easily deteriorates or inflates, it is considered a \textit{soft} currency. The
concept of hardness is essential to understand Bitcoin and is worthy of
a more thorough examination. We will return to it in the last economic
lesson: sound money.

\paragraph{}
As more and more countries suffer from
hyperinflation more and more people will have to face the reality
of hard and soft money. If we are lucky, maybe even some central bankers will be
forced to re-evaluate their monetary policies. Whatever might happen, the
insights I have gained thanks to Bitcoin will probably be invaluable, no matter
the outcome.

\paragraph{Bitcoin taught me about the hidden tax of inflation and the 
catastrophe of hyperinflation.}

% ---
%
% #### Down the Rabbit Hole
%
% - [Economics in One Lesson][Henry Hazlitt] by Henry Hazlitt
% - [1980's Unemployment and the Unions][unions] by Friedrich Hayek
% - [Good Money, Part II][good-money]: Volume Six of the Collected Works of F.A. Hayek
% - [The Bitcoin Standard] by Saifedean Ammous
% - [Hyperinflation][hyperinflates], [economic crisis in Venezuela][wiki-venezuela], [list of historical currencies], [list of currencies][currently in use] on Wikipedia
%
% [unions]: https://books.google.com/books/about/1980s_unemployment_and_the_unions.html?id=xM9CAQAAIAAJ
% [good-money]: https://books.google.com/books?id=l_A1vVIaYBYC
%
% [Henry Hazlitt]: https://mises.org/library/economics-one-lesson
% [hyperinflates]: https://en.wikipedia.org/wiki/Hyperinflation
% [inflation cannot help]: https://books.google.com/books?id=zZu3AAAAIAAJ&dq=%22only+while+it+accelerates%22&focus=searchwithinvolume&q=%22steady+inflation+cannot+help%22
% [history of inflation]: https://books.google.com/books?id=l_A1vVIaYBYC&pg=PA142&dq=%22history+is+largely+a+history+of+inflation%22&hl=en&sa=X&ved=0ahUKEwi90NDLrdnfAhUprVkKHUx1CmIQ6AEIKjAA#v=onepage&q=%22history%20is%20largely%20a%20history%20of%20inflation%22&f=false
% [wiki-venezuela]: https://en.wikipedia.org/wiki/Crisis_in_Venezuela#Economic_crisis
% [by the laws of physics]: https://link.medium.com/9fzq2L0J3S
% [\textit{Gold-to-Decent-Suit Ratio}]: https://www.businesswire.com/news/home/20110819005774/en/History-Shows-Price-Ounce-Gold-Equals-Price
% [The Bitcoin Standard]: https://thesaifhouse.wordpress.com/book/
%
% <!-- Wikipedia -->
% [alice]: https://en.wikipedia.org/wiki/Alice%27s_Adventures_in_Wonderland
% [carroll]: https://en.wikipedia.org/wiki/Lewis_Carroll
