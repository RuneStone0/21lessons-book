\chapter{Fractional Reserve Insanity}
\label{les:13}

\begin{chapquote}{Lewis Carroll, \textit{Alice in Wonderland}}
Ak, det var for sent: hun fortsatte med at vokse og vokse, og meget snart var 
der ikke plads til engang dette, og hun prøvede virkningen af at lægge sig 
ned, med den ene albue mod døren og den anden arm krøllet omkring hendes hoved. 
Alligevel fortsatte hun med at vokse, og som en sidste udvej stak hun den ene 
arm ud af vinduet, og det ene ben op i skorstenen, og sagde til sig selv
\enquote{nu kan jeg ikke gøre mere — hvad vil der ske med mig?}
\end{chapquote}

Værdi og penge er ikke trivielle emner, især ikke i dagens tid. Processen med 
pengeoprettelse i vores banksystem er lige så ikke-triviel, og jeg kan ikke 
slippe følelsen af, at det er med vilje. Det, jeg tidligere kun har stødt på 
i akademiske og juridiske tekster, synes også at være almindelig praksis i 
finansverdenen: intet forklares i enkle vilkår, ikke fordi det er virkelig 
komplekst, men fordi sandheden er skjult bag lag og lag af jargon og 
\textit{tilladt} kompleksitet. \enquote{Ekspansiv pengepolitik, kvantitativ 
lempelse, finansiel stimulans til økonomien.} Publikum nikker enig,
hipnotiseret af de prangende ord.

Fractional reserve banking og kvantitativ lempelse er to af disse prangende ord,
der slører, hvad der virkelig sker, ved at maskere det som komplekst og 
svært at forstå. Hvis du skulle forklare dem til en femårig, vil vanviddet af 
begge blive tydeligt hurtigt.

Godfrey Bloom sagde det meget bedre, da han talte til Europa-Parlamentet 
under en fælles debat:

\begin{quotation}\begin{samepage}
\enquote{[...] I forstår ikke rigtig bankvirksomhedens koncept. Alle
bankerne er fallit. Bank Santander, Deutsche Bank, Royal Bank of
Scotland --- de er alle fallit! Og hvorfor er de fallit? Det er ikke en
Guds handling. Det er ikke en slags tsunami. De er fallit, fordi vi
har et system kaldet 'fractional reserve banking', hvilket betyder, at
bankerne kan låne penge ud, som de faktisk ikke har! Det er en kriminel
skandale, og det har stået på for længe. [...]
Vi har forfalskning --- nogle gange kaldet kvantitativ lempelse ---
men forfalskning ved ethvert andet navn. Den kunstige
pengesedelproduktion, som hvis enhver almindelig person gjorde det, ville de 
ende i fængsel i meget lang tid [...] og indtil vi begynder at sende
bankfolk --- og jeg inkluderer centralbankfolk og politikere --- i
fængsel for denne foragtelige handling, vil det fortsætte.}
\begin{flushright} -- Godfrey Bloom\footnote{Fælles debat om
bankunionen~\cite{godfrey-bloom}}
\end{flushright}\end{samepage}\end{quotation}

Lad mig gentage den mest vigtige del: Banker kan låne penge ud, som de faktisk 
ikke har.

Takket være fractional reserve banking skal en bank kun beholde en lille
\textit{fraktion} af hver dollar, den får. Det er et sted mellem $0$ og $10\%$,
normalt i den lavere ende, hvilket gør tingene endnu værre.

Lad os bruge et konkret eksempel for bedre at forstå denne skøre idé: En
fraktion på $10\%$ vil gøre tricket, og vi burde være i stand til at lave alle
beregningerne i vores hoved. Win-win. Så hvis du tager \$100 til en
bank --- fordi du ikke ønsker at opbevare det under din madras --- skal de kun
beholde den aftalte \textit{fraktion} af det. I vores eksempel ville det
være \$10, fordi 10\% af \$100 er \$10. Nemt, ikke sandt?

Så hvad gør banker med resten af pengene? Hvad sker der med dine \$90? De
gør, hvad banker gør, de låner dem ud til andre mennesker. Resultatet er en 
pengemultiplikatoreffekt, der øger pengeforsyningen i økonomien enormt
(Figur~\ref{fig:money-multiplier}). Dit indledende indskud på \$100 vil snart
blive til \$190. Ved at låne en fraktion på 90\% af de nyoprettede \$90 vil der
snart være \$271 i økonomien. Og \$343.90 derefter. Pengeforsyningen er
recursivt stigende, da banker bogstaveligt talt låner penge, de ikke
har~\cite{wiki:money-multiplier}. Uden et eneste Abrakadabra forvandler
banker magisk \$100 til tusind dollars eller mere. Det viser sig, at 10x er 
nemt. Det kræver kun et par udlånsskridt.

\begin{figure}[htbp]
  \centering
  \includegraphics[width=\textwidth]{assets/images/money-multiplier.png}
  \caption{Pengemultiplikatoreffekten}
  \label{fig:money-multiplier}
\end{figure}
  
\paragraph{}
Misforstå mig ikke: Der er ikke noget galt med at låne penge ud. Der er
ikke noget galt med renter. Der er endda ikke noget galt med gode,
gamle almindelige banker for at opbevare din formue et mere sikkert sted end i
din strømpeskuffe.

Centralbanker derimod er en anden størrelse. Grusomheder inden for finansiel
regulering, halvt offentlige halvt private, der leger gud med noget, der
påvirker alle, der er en del af vores globale civilisation, uden en
samvittighed, kun interesseret i den umiddelbare fremtid og tilsyneladende
uden nogen form for ansvarlighed eller auditabilitet (se Figur~\ref{fig:bsg}).

\begin{figure}[htbp]
  \centering
  \includegraphics[width=\textwidth]{assets/images/bsg.jpg}
  \caption{Yellen er stærkt imod revision af Fed, mens Bitcoin Sign Guy 
  stærkt er tilhænger af at købe bitcoin.}
  \label{fig:bsg}
\end{figure}

Mens Bitcoin stadig er inflationsmæssig, vil det ophøre med at være det ret 
snart. Den strengt begrænsede forsyning af 21 millioner bitcoins vil med 
tiden fjerne inflationen fuldstændigt. Vi har nu to monetære verdener: en
inflationær, hvor penge trykkes vilkårligt, og verden af
Bitcoin, hvor den endelige forsyning er fast og let auditabel for alle.
Den ene påtvinges os ved vold, den anden kan tilsluttes af enhver, der ønsker
det. Ingen indgangsbarrierer, ingen at spørge om tilladelse.
Frivillig deltagelse. Det er skønheden ved Bitcoin.

Jeg ville argumentere for, at argumentet mellem keynesianske\footnote{Teorier i
henhold til John Maynard Keynes og hans disciple~\cite{wiki:keynesian}} og
østrigske\footnote{Skole inden for økonomisk tænkning baseret på metodologisk
individualisme~\cite{wiki:austrian}} økonomer ikke længere er rent akademisk.
Satoshi formåede at opbygge et system til værdioverførsel på steroider, hvilket 
skabte den mest lydige valuta, der nogensinde har eksisteret i processen. På 
den ene eller anden måde vil flere og flere mennesker lære om fidusen, som er 
fractional reserve banking. Hvis de når til lignende konklusioner som de fleste 
østrigere og bitcoinere, kunne de tilslutte sig det stadig voksende internet af 
penge. Ingen kan stoppe dem, hvis de vælger at gøre det.

\paragraph{Bitcoin lærte mig, at fractional reserve banking er ren vanvid.}

% ---
%
% #### Down the Rabbit Hole
%
% - [The Creature From Jekyll Island] by G. Edward Griffin
% - [Money Multiplier][money multiplier], [Keynesian Economics][Keynesian], [Austrian School][Austrian] on Wikipedia
%
% [The Creature From Jekyll Island]: https://archive.org/details/pdfy--Pori1NL6fKm2SnY
%
% [joint debate]: https://www.youtube.com/watch?v=hYzX3YZoMrs
% [money multiplier]: https://en.wikipedia.org/wiki/Money_multiplier
% [auditability]: https://i.ytimg.com/vi/ThFGs347MW8/maxresdefault.jpg
% [Keynesian]: https://en.wikipedia.org/wiki/Keynesian_economics
% [Austrian]: https://en.wikipedia.org/wiki/Austrian_School
%
% <!-- Wikipedia -->
% [alice]: https://en.wikipedia.org/wiki/Alice%27s_Adventures_in_Wonderland
% [carroll]: https://en.wikipedia.org/wiki/Lewis_Carroll
