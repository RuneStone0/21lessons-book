\chapter{At Fortælle Tid Kræver Arbejde}
\label{les:17}

\begin{chapquote}{Lewis Carroll, \textit{Alice i Eventyrland}}
\enquote{Kære, kære! Jeg kommer for sent!}
\end{chapquote}

Det bliver ofte sagt, at bitcoins udvindes, fordi tusinder af computere 
arbejder på at løse \textit{meget komplekse} matematiske problemer. Visse 
problemer skal løses, og hvis du beregner det rigtige svar, 
\enquote{producerer} du en bitcoin. Mens denne forenklede opfattelse af 
bitcoin-minedrift måske er lettere at formidle, går den lidt forbi pointen. 
Bitcoins bliver ikke produceret eller skabt, og hele affæren handler ikke 
rigtig om at løse bestemte matematiske problemer. Desuden er matematikken 
ikke særlig kompleks. Det komplekse er at \textit{fortælle tiden} i et 
decentraliseret system.

Som beskrevet i whitepaperet er proof-of-work-systemet (også kaldet minedrift) 
en måde at implementere en distribueret tidsstempelserver.

\begin{center}
  \includegraphics[width=\textwidth]
  {assets/images/bitcoin-whitepaper-timestamp-wide.png}
  \captionof{figure}{Uddrag fra whitepaperet. Sagde nogen tidskæde?}
  \label{fig:bitcoin-whitepaper-timestamp-wide}
\end{center}

Da jeg først lærte, hvordan Bitcoin fungerer, troede jeg også, at 
proof-of-work er ineffektivt og spild af ressourcer. Efter et stykke tid 
begyndte jeg dog at ændre min opfattelse af Bitcoin's 
energiforbrug~\cite{gigi:energy}. Det ser ud til, at proof-of-work stadig 
er bredt misforstået i dag, i året 10 AB (efter Bitcoin).

Da problemerne, der skal løses i proof-of-work, er opfundne, synes mange 
mennesker at tro, at det er \textit{unyttigt} arbejde. Hvis fokus udelukkende 
er på beregningen, er dette en forståelig konklusion. Men Bitcoin handler 
ikke om beregning. Det handler om \textit{uafhængigt at blive enige om 
rækkefølgen af tingene.}

Proof-of-work er et system, hvor alle kan validere, hvad der skete, og i 
hvilken rækkefølge det skete. Denne uafhængige validering er det, der fører 
til konsensus, en individuel enighed mellem flere parter om, hvem der ejer hvad.

I et radikalt decentraliseret miljø har vi ikke luksusen af absolut tid. 
Enhver ur ville introducere en betroet tredjepart, et centralt punkt i 
systemet, som man skulle stole på og kunne angribes. \enquote{Tid er 
rodproblemet,} som Grisha Trubetskoy påpeger~\cite{pow-clock}. Og Satoshi 
løste genialt dette problem ved at implementere et decentraliseret ur via 
en proof-of-work blockchain. Alle er enige på forhånd om, at kæden med det 
største kumulative arbejde er sandhedens kilde. Det er per definition, hvad 
der faktisk skete. Denne enighed er det, der nu er kendt som Nakamoto-konsensus.

\begin{quotation}\begin{samepage}
  \enquote{Netværket tidsstempler transaktioner ved at hashe dem ind i en 
  løbende kæde, som fungerer som bevis på sekvensen af begivenheder, der er 
  vidne til.}
  \begin{flushright} -- Satoshi Nakamoto\footnote{Satoshi Nakamoto, 
    Bitcoin-whitepaperet~\cite{whitepaper}}
\end{flushright}\end{samepage}\end{quotation}
  
Uden en sammenhængende måde at fortælle tiden på er der ingen konsekvent måde at
skelne før fra efter. Pålidelig ordning er umulig. Som nævnt
ovenfor er Nakamoto-konsensus Bitcoins måde at konsekvent fortælle tiden på. 
Systemets incitamentsstruktur producerer et sandsynligt,
decentraliseret ur, ved at udnytte både grådighed og egeninteresse hos
konkurrerende deltagere. Det faktum, at dette ur er upræcist, er
ligegyldigt, fordi begivenhedernes rækkefølge er efterhånden entydig og kan
verificeres af alle.

Takket være proof-of-work er både arbejdet \textit{og} valideringen af arbejdet
radikalt decentraliseret. Alle kan deltage og forlade efter ønske, og
alle kan validere alt til enhver tid. Ikke kun det, men
alle kan validere systemets tilstand \textit{individuelt}, uden
at skulle stole på nogen andre for validering.

At forstå proof-of-work tager tid. Det er ofte modintuitivt,
og selvom reglerne er simple, fører de til ret komplekse fænomener.
For mig hjalp det at ændre min opfattelse af minedrift. Nyttig, ikke unyttig.
Validering, ikke beregning. Tid, ikke blokke.

\paragraph{Bitcoin lærte mig, at at fortælle tiden er svært, især hvis du er
decentraliseret.}

% ---
%
% #### Through the Looking-Glass
%
% - [Bitcoin's Energy Consumption: A shift in perspective][energy]
%
% #### Down the Rabbit Hole
%
% - [Blockchain Proof-of-Work Is a Decentralized Clock][points out] by Gregory Trubetskoy
% - [The Anatomy of Proof-of-Work][pow-anatomy] by Hugo Nguyen
% - [PoW is efficient][pow-efficient] by Dan Held
% - [Mining][bw-mining], [Controlled supply][bw-supply] on the Bitcoin Wiki
%
% [points out]: https://grisha.org/blog/2018/01/23/explaining-proof-of-work/
% [energy]: 
% [whitepaper]: https://bitcoin.org/bitcoin.pdf
%
% [pow-efficient]: https://blog.picks.co/pow-is-efficient-aa3d442754d3
% [pow-anatomy]: https://bitcointechtalk.com/the-anatomy-of-proof-of-work-98c85b6f6667
% [bw-mining]: https://en.bitcoin.it/wiki/Mining
% [bw-supply]: https://en.bitcoin.it/wiki/Controlled_supply
%
% <!-- Wikipedia -->
% [alice]: https://en.wikipedia.org/wiki/Alice%27s_Adventures_in_Wonderland
% [carroll]: https://en.wikipedia.org/wiki/Lewis_Carroll
