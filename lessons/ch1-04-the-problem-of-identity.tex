\chapter{Identitetsproblemet}
\label{les:4}

\begin{chapquote}{Lewis Carroll, \textit{Alice i Eventyrland}}
\enquote{Hvem er du?} sagde larven.
\end{chapquote}

Nic Carter, som en hyldest til Thomas Nagels behandling af det samme spørgsmål 
vedrørende en flagermus, skrev et glimrende stykke, der drøfter følgende 
spørgsmål: Hvad er det at være en bitcoin? Han viser på en glimrende måde, at 
åbne, offentlige blockchains generelt, og Bitcoin i særdeleshed, lider af det 
samme dilemma som Theseus' skib\footnote{I identitetens metafysik er Theseus' 
skib et tankeeksperiment, der rejser spørgsmålet om, hvorvidt et objekt, der 
har haft alle sine komponenter udskiftet, stadig er fundamentalt det samme 
objekt.~\cite{wiki:theseus}}: hvilken Bitcoin er den virkelige Bitcoin?

\begin{quotation}\begin{samepage}
\enquote{Overvej, hvor lidt vedvarende Bitcoin's komponenter har. Hele 
kodebasen er blevet omarbejdet, ændret og udvidet, så den næsten ikke ligner 
sin oprindelige version. [...] Registeret over hvem der ejer hvad, selve 
hovedbogen, er stort set den eneste vedvarende egenskab ved netværket [...]
For at blive betragtet som virkelig lederløs må du opgive den nemme løsning med 
at have en enhed, der kan udpege én kæde som den legitime.}
\begin{flushright} -- Nic Carter\footnote{Nic Carter, \textit{Hvad er det at 
    være en bitcoin?} \cite{bitcoin-identity}}
\end{flushright}\end{samepage}\end{quotation}

Det ser ud til, at teknologiens fremskridt bliver ved med at tvinge os til at 
tage disse filosofiske spørgsmål alvorligt. Før eller senere vil selvkørende 
biler blive konfronteret med virkelige versioner af sporvognsproblemet, der 
tvinger dem til at træffe etiske beslutninger om, hvilke liv der betyder noget, 
og hvilke der ikke gør.

Kryptovalutaer, især siden den første kontroversielle hard fork, tvinger os til 
at tænke over og blive enige om identitetens metafysik. Interessant nok har de 
to største eksempler, vi har hidtil, ført til to forskellige svar. Den 1. 
august 2017 splittes Bitcoin i to lejre. Markedet besluttede, at den uændrede 
kæde er den originale Bitcoin. Et år tidligere, den 25. oktober 2016, splittes 
Ethereum i to lejre. Markedet besluttede, at den \textit{ændrede} kæde er den 
originale Ethereum.

Hvis det er ordentligt decentraliseret, vil spørgsmålene rejst af 
\textit{Theseus' skib} nødt til at blive besvaret i al evighed, så længe disse 
værdioverførselsnetværk eksisterer.

\paragraph{Bitcoin lærte mig, at decentralisering modsiger identitet.}

% ---
%
% #### Down the Rabbit Hole
%
% - [What Is It Like to be a Bat?][in regards to a bat] by Thomas Nagel
% - [What is it like to be a bitcoin?] by Nic Carter
% - [Ship of Theseus], [trolley problem] on Wikipedia
%
% [in regards to a bat]: https://en.wikipedia.org/wiki/What_Is_it_Like_to_Be_a_Bat%3F
% [What is it like to be a bitcoin?]: https://medium.com/s/story/what-is-it-like-to-be-a-bitcoin-56109f3e6753
% [Ship of Theseus]: https://en.wikipedia.org/wiki/Ship_of_Theseus
% [trolley problem]: https://en.wikipedia.org/wiki/Trolley_problem
%
% <!-- Wikipedia -->
% [alice]: https://en.wikipedia.org/wiki/Alice%27s_Adventures_in_Wonderland
% [carroll]: https://en.wikipedia.org/wiki/Lewis_Carroll
