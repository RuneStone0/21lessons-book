\chapter{Metaforer for Bitcoins fremtid}
\label{les:21}

\begin{chapquote}{Lewis Carroll, \textit{Alice i Eventyrland}}
\enquote{Jeg ved, at der er noget interessant, der helt sikkert vil ske\ldots}
\end{chapquote}

I løbet af de sidste par årtier blev det tydeligt, at teknologisk innovation 
ikke følger en lineær trend. Uanset om du tror på den teknologiske singularitet 
eller ej, er det uundgåeligt, at fremskridt er eksponentielt i mange områder. 
Ikke kun det, men hastigheden, hvormed teknologier bliver adopteret, accelererer, 
og før du ved af det, er busken på skolegården væk, og dine børn bruger i stedet 
Snapchat. Eksponentielle kurver har en tendens til at slå dig i ansigtet længe 
før, du ser dem komme.

Bitcoin er en eksponentiel teknologi bygget på eksponentielle teknologier.
\textit{Our World in Data}\footnote{\url{https://ourworldindata.org/}} viser 
smukt den stigende hastighed af teknologisk adoption, der starter i 1903 med 
introduktionen af landlinjer (se figur~\ref{fig:tech-adoption}). Landlinjer, elektricitet, computere, internet, smartphones; følger alle eksponentielle tendenser inden for pris-ydelsesforhold og adoption. Det gør Bitcoin også~\cite{tech-adoption}.

\begin{center}
  \includegraphics[width=\textwidth]{assets/images/tech-adoption.png}
  \captionof{figure}{Bitcoin er bogstaveligt talt uden for diagrammet.}
  \label{fig:tech-adoption}
\end{center}

Bitcoin har ikke kun én, men flere netværkseffekter\footnote{Trace Mayer,
\textit{The Seven Network Effects of Bitcoin}~\cite{7-network-effects}}, alle
som resulterer i eksponentielle vækstmønstre inden for deres respektive områder: pris,
brugere, sikkerhed, udviklere, markedsandel og global adoption som penge.

Efter at have overlevet sin spæde barndom vokser Bitcoin fortsat hver dag på
flere områder end ét. Teknologien er måske ikke nået fuld modenhed endnu.
Den kan være i sin adolescens. Men hvis teknologien er eksponentiel, er vejen
fra obskuritet til udbredelse kort.

\begin{center}
  \includegraphics[width=\textwidth]{assets/images/mobile-phone.png}
  \captionof{figure}{Mobiltelefon, ca. 1965 vs. 2019.}
  \label{fig:mobile-phone}
\end{center}

I sin TED-talk fra 2003 valgte Jeff Bezos at bruge elektricitet som en metafor
for websitets fremtid.\footnote{\url{http://bit.ly/bezos-web}} Alle tre fænomener ---
elektricitet, internettet, Bitcoin --- er \textit{muliggørende} teknologier,
netværk, der gør andre ting mulige. De er infrastruktur at bygge på,
grundlæggende i naturen.

Elektricitet har været til stede i lang tid nu. Vi tager det for givet. Internettet er noget yngre, men de fleste mennesker tager det også for givet. Bitcoin er ti år gammel og er trådt ind i den offentlige bevidsthed under den seneste hype-cyklus. Kun de tidligste adoptører tager det for givet. Jo mere tid der går, desto flere mennesker vil genkende Bitcoin som noget, der bare er.\footnote{Dette er kendt som \textit{Lindy-effekten}. Lindy-effekten er en teori om, at den fremtidige forventede levetid for visse ikke-forgængelige ting som teknologi eller en idé er proportional med deres nuværende alder, således at hver ekstra overlevelsesperiode indebærer en længere forventet levetid.~\cite{wiki:lindy}}

I 1994 var internettet stadig forvirrende og uhåndterligt. At se denne gamle optagelse af \textit{Today Show}\footnote{\url{https://youtu.be/UlJku_CSyNg}} gør det tydeligt, at hvad der føles naturligt og intuitivt nu faktisk ikke var det dengang. Bitcoin er stadig forvirrende og fremmed for de fleste, men ligesom internettet er anden natur for digitale indfødte, vil at bruge og stable sats\footnote{\url{https://twitter.com/hashtag/stackingsats}} være anden natur for fremtidens Bitcoin-indfødte.

\begin{quotation}\begin{samepage}
\enquote{Fremtiden er allerede her --- den er bare ikke særlig ligeligt fordelt.}
\begin{flushright} -- William Gibson\footnote{William Gibson, \textit{Videnskaben i science fiction} \cite{william-gibson}}
\end{flushright}\end{samepage}\end{quotation}

I 1995 brugte omkring $15\%$ af amerikanske voksne internettet. Historiske data fra Pew Research Center~\cite{pew-research} viser, hvordan internettet har indvævet sig i alle vores liv. Ifølge en forbrugerundersøgelse foretaget af Kaspersky Lab~\cite{web:kaspersky} har 13\% af respondenterne brugt Bitcoin og dets kloner til at betale for varer i 2018. Selvom betalinger ikke er den eneste anvendelse af bitcoin, er det en indikation af, hvor vi er i internettets tid: i starten til midten af 90'erne.

I 1997 erklærede Jeff Bezos i et brev til aktionærerne~\cite{bezos-letter}, at \enquote{dette er dag 1 for internettet,} idet han anerkendte det store uudnyttede potentiale for internettet og dermed hans virksomhed. Uanset hvilken dag det er for Bitcoin, er de enorme mængder uudnyttet potentiale tydelige for alle, undtagen den mest tilfældige observatør.

\begin{center}
  \includegraphics[width=\textwidth]{assets/images/internet-evolution-white-dates.png}
  \captionof{figure}{Internettet, 1982 vs. 2005. Kilde: cc-by Merit Network, Inc. og Barrett Lyon, Opte Project}
  \label{fig:internet-evolution-white-dates}
\end{center}

Bitcoin's første node gik online i 2009, efter at Satoshi mined \textit{genesis
blokken}\footnote{Genesisblokken er den første blok i Bitcoin-blokkenkæden.
Moderne versioner af Bitcoin nummererer den som blok $0$, selvom meget tidlige versioner
tællede den som blok $1$. Genesisblokken er normalt hårdkodet i
softwaren i applikationer, der bruger Bitcoin-blokkenkæden. Det er et
særligt tilfælde, fordi det ikke henviser til en tidligere blok og producerer en
udfoldelig subsidie. \textit{coinbase}-parameteren indeholder, sammen med
normal data, følgende tekst: \textit{\enquote{The Times 03/Jan/2009 Chancellor on
brink of second bailout for banks}} \cite{btcwiki:genesis-block}} og udgav
softwaren i det vilde. Hans node var ikke alene i lang tid. Hal Finney var en
af de første til at opfatte ideen og slutte sig til netværket. Ti år
senere, ved skrivelsen af dette, kører mere end
$75.000$\footnote{\url{https://bit.ly/luke-nodecount}} noder Bitcoin.

\begin{center}
  \centering
  \includegraphics[width=8cm]{assets/images/running-bitcoin.png}
  \captionof{figure}{Hal Finney forfattede den første tweet, der nævnte bitcoin i januar 2009.}
  \label{fig:running-bitcoin}
\end{center}

Protokollens basislag er ikke det eneste, der vokser eksponentielt.
Lightning-netværket, en teknologi på andet niveau, vokser endnu
hurtigere.

I januar 2018 havde lynnetværket $40$ noder og $60$
kanaler~\cite{web:lightning-nodes}. I april 2019 voksede netværket til mere
end $4000$ noder og omkring $40.000$ kanaler. Husk, at dette stadig er
eksperimentel teknologi, hvor tab af midler kan og sker. Alligevel er tendensen
tydelig: tusindvis af mennesker er uforsigtige og ivrige efter at bruge det.

\begin{center}
  \includegraphics[width=\textwidth]{assets/images/lnd-growth-lopp-white.png}
  \captionof{figure}{Lightning Network, januar 2018 vs december 2018. Kilde: Jameson Lopp}
  \label{fig:lnd-growth-lopp-white.png}
\end{center}

For mig, der har oplevet den meteoriske stigning af internettet, er parallelle
mellem internettet og Bitcoin åbenlyse. Begge er netværk, begge
er eksponentielle teknologier, og begge muliggør nye muligheder, nye
industrier, nye livsformer. Ligesom elektricitet var den bedste
metafor for at forstå, hvor internettet er på vej hen, kan internettet
være den bedste metafor for at forstå, hvor Bitcoin er på vej hen. Eller, som
Andreas Antonopoulos siger det, er Bitcoin \textit{The Internet of Money}.
Disse metaforer er en fantastisk påmindelse om, at selvom historie ikke gentager
sig, rimer den ofte.

Eksponentielle teknologier er svære at forstå og ofte undervurderede.
Selvom jeg har stor interesse i sådanne teknologier, bliver jeg
konstant overrasket over tempoet af fremskridt og innovation. At se
Bitcoin-økosystemet vokse er som at se internettets opstigning i
hurtigafspilning. Det er opløftende.

Min søgen efter at forsøge at forstå Bitcoin har ført mig ned ad stierne
af historie på mere end én måde. At forstå gamle samfundsstrukturer, tidligere valutaer,
og hvordan kommunikationsnetværk udviklede sig, var alle
en del af rejsen. Fra håndøksen til smartphones har teknologi
utvivlsomt ændret vores verden mange gange. Netværksteknologier
er især transformationelle: skrivning, veje, elektricitet, internettet.
Alle har ændret verden. Bitcoin har ændret min og
vil fortsætte med at ændre sind og hjerter hos dem, der tør bruge
det.

\paragraph{Bitcoin lærte mig, at forståelse af fortiden er afgørende for
at forstå dens fremtid. En fremtid, der kun lige er begyndt\ldots}

% ---
%
% #### Down the Rabbit Hole
%
% - [The Rising Speed of Technological Adoption][the rising speed of technological adoption] by Jeff Desjardins
% - [The 7 Network Effects of Bitcoin][multiple network effects] by Trace Mayer
% - [The Electricity Metaphor for the Web's Future][TED talk] by Jeff Bezos
% - [How the internet has woven itself into American life][data from the Pew Research Center] by Susannah Fox and Lee Rainie
% - [Genesis Block][genesis block] on the Bitcoin Wiki
% - [Lindy Effect][more time] on Wikipedia
%
% [Our World in Data]: https://ourworldindata.org/
% [the rising speed of technological adoption]: https://www.visualcapitalist.com/rising-speed-technological-adoption/
% [multiple network effects]: https://www.thrivenotes.com/the-7-network-effects-of-bitcoin/
% [TED talk]: https://www.ted.com/talks/jeff_bezos_on_the_next_web_innovation
% [recording of the Today Show]: https://www.youtube.com/watch?v=UlJku_CSyNg
% [William Gibson]: https://www.npr.org/2018/10/22/1067220/the-science-in-science-fiction
% [data from the Pew Research Center]: https://www.pewinternet.org/2014/02/27/part-1-how-the-internet-has-woven-itself-into-american-life/
% [consumer survey]: https://www.kaspersky.com/blog/money-report-2018/
% [letter to shareholders]: http://media.corporate-ir.net/media_files/irol/97/97664/reports/Shareholderletter97.pdf
% [running bitcoin]: https://twitter.com/halfin/status/1110302988?lang=en
% [40 nodes]: https://bitcoinist.com/bitcoin-lightning-network-mainnet-nodes/
% [reckless]: https://twitter.com/hashtag/reckless
% [Jameson Lopp]: https://twitter.com/lopp/status/1077200836072296449
% [\textit{The Internet of Money}]: https://theinternetofmoney.info/
% [stacking]: https://twitter.com/hashtag/stackingsats
%
% <!-- Bitcoin Wiki -->
% [genesis block]: https://en.bitcoin.it/wiki/Genesis_block
%
% <!-- Wikipedia -->
% [more time]: https://en.wikipedia.org/wiki/Lindy_effect
% [alice]: https://en.wikipedia.org/wiki/Alice%27s_Adventures_in_Wonderland
% [carroll]: https://en.wikipedia.org/wiki/Lewis_Carroll
