Den vigtigste lektion, jeg har lært af Bitcoin, er, at på lang sigt er hård 
valuta overlegen blød valuta. Hård valuta, også kaldet \textit{sound money} 
(lydige penge), er enhver globalt handlet valuta, der fungerer som en pålidelig 
værdibevarer.

Selvfølgelig er Bitcoin stadig ung og volatil. Kritikere vil sige, at den ikke 
pålideligt bevarer værdi. Argumentet om volatilitet overser dog pointen. 
Volatilitet skal forventes. Markedet vil tage tid om at finde den retfærdige 
pris på denne nye valuta. Derudover påpeges det ofte spøgende, at det er 
baseret på en målefejl. Hvis du tænker i dollars, vil du overse, at en bitcoin 
altid vil være en bitcoin.

\begin{quotation}\begin{samepage}
\enquote{En fast pengemængde eller en mængde, der kun ændres i 
overensstemmelse med objektive og beregnelige kriterier, er en nødvendig 
betingelse for en meningsfuld retfærdig pris på penge.}
\begin{flushright} -- Fr. Bernard W. Dempsey, S.J.\footnote{Perry J. Roets, 
  S.J., \textit{Review of Social Economy} \cite{review-social-economy}}
\end{flushright}\end{samepage}\end{quotation}

\newpage

Som en hurtig tur gennem gravpladsen forglemte valutaer har vist,
vil penge, der kan trykkes, blive trykt. Indtil videre har ingen mennesker i
historien været i stand til at modstå denne fristelse.

Bitcoin gør op med fristelsen til at trykke penge på en genial
måde. Satoshi var opmærksom på vores grådighed og fejlbarlighed --- derfor 
valgte han noget mere pålideligt end menneskelig tilbageholdenhed: matematik.

\begin{figure}[htbp]
  \centering
  \begin{equation}
    \sum_{i=0}^{32} \frac{21000 \lfloor 
    \frac{50 \cdot 10^8}{2^i} \rfloor}{10^8}
  \end{equation}
  \caption{Bitcoin's supply formula}
  \label{fig:supply-formula-white}
\end{figure}

Selvom denne formel er nyttig til at beskrive Bitcoins forsyning, er den faktisk
ingen steder at finde i koden. Udstedelse af nye bitcoins sker på en
algoritmisk kontrolleret måde ved at reducere belønningen, der betales til
minearbejdere hvert fjerde år~\cite{btcwiki:supply}. Formlen ovenfor bruges til
at sammenfatte hurtigt, hvad der sker under motorhjelmen. Hvad der virkelig 
sker, kan bedst ses ved at se på ændringen i blokbelønningen, belønningen 
udbetalt til den, der finder en gyldig blok, hvilket groft sker hvert 10. minut.

\begin{figure}[htbp]
  \centering
  \includegraphics[width=\textwidth]{assets/images/you-are-here.png}
  \caption{Bitcoins kontrollerede forsyning}
  \label{fig:you-are-here}
\end{figure}

Formler, logaritmiske funktioner og eksponentialfunktioner er ikke nøjagtigt
intuitive at forstå. Konceptet \textit{lydighed} kan være lettere at
forstå, hvis det betragtes på en anden måde. Når vi først ved, hvor meget der er
af noget, og når vi ved, hvor svært det er at producere eller
få fat i denne ting, forstår vi straks dens værdi. Hvad der er sandt for
Picassos malerier, Elvis Presleys guitarer og Stradivarius-violiner,
er også sandt for fiatvaluta, guld og bitcoins.

Hårdheden af fiatvaluta afhænger af, hvem der har ansvaret for
de respektive trykkerier. Nogle regeringer er måske mere villige til at
trykke store mængder valuta end andre, hvilket resulterer i en svagere
valuta. Andre regeringer kan være mere restriktive i deres penges
trykning, hvilket resulterer i en hårdere valuta.

\begin{samepage}\begin{quotation}
\enquote{En vigtig aspect af denne nye virkelighed er, at institutioner som
Federal Reserve ikke kan gå konkurs. De kan printe enhver mængde penge, de
måtte have brug for, til næsten ingen omkostninger.}
\begin{flushright} -- Jörg Guido Hülsmann\footnote{Jörg Guido Hülsmann, 
  \textit{The Ethics of Money Production}~\cite{hulsmann2008ethics}}
\end{flushright}\end{quotation}\end{samepage}

Før vi havde fiatvalutaer, blev lydigheden af penge bestemt af
de naturlige egenskaber ved det, vi brugte som penge. Mængden
af guld på jorden er begrænset af fysikkens love. Guld er sjældent, fordi
supernovaer og kollisioner mellem neutronstjerner er sjældne. 
\enquote{Strømmen} af guld er begrænset, fordi udvinding af det er en stor 
indsats. Som et tungt element er det hovedsageligt begravet dybt under jorden.

Ophævelsen af guldfoden åbnede op for en ny virkelighed: tilføjelse af nye penge
kræver blot en dråbe blæk. I vores moderne verden kræver tilføjelse af et par
nuller til saldoen på en bankkonto endnu mindre indsats: at flippe et par bits 
i en bankcomputer er nok.

Princippet beskrevet ovenfor kan udtrykkes mere generelt som
forholdet mellem \enquote{lager} og \enquote{strøm}. Ganske enkelt er 
\textit{lageret} hvor meget af noget der i øjeblikket er til stede. For vores 
formål er lageret en måling af den nuværende pengeforsyning. \textit{Strømmen} 
er, hvor meget der produceres over en periode (f.eks. per år). Nøglen til at 
forstå lydighed af penge ligger i forståelsen af dette lager-til-strøm-forhold.

Beregning af lager-til-strøm-forholdet for fiatvaluta er vanskelig, fordi hvor
mange penge der er, afhænger af, hvordan du ser på det.~\cite{wiki:money-supply} 
Du kunne tælle kun pengesedler og mønter (M0), tilføje rejsechecks og checke
indskud (M1), tilføje opsparingskonti og investeringsfonde og nogle andre ting 
(M2), og endda tilføje indlånscertifikater til alt dette (M3). Desuden 
varierer, hvordan alt dette er defineret og målt, fra land til land, og da den 
amerikanske centralbank stoppede med at offentliggøre \cite{web:fed-m3} tal 
for M3, må vi nøjes med pengemængden M2. Jeg ville gerne verificere disse
tal, men jeg gætter på, at vi må stole på centralbanken for nu.

Guld, en af de sjældneste metaller på jorden, har det højeste lager-til-strøm
forhold. Ifølge US Geological Survey er lidt mere end 190.000 tons blevet 
udvundet. I de sidste få år er der blevet udvundet omkring 3100 tons guld
årligt.~\cite{mineral-commodity-summaries}

Ved hjælp af disse tal kan vi nemt beregne lager-til-strøm-forholdet for
guld (se figur~\ref{fig:stock-to-flow-gold}).

\begin{figure}[htbp]
  \centering
  \begin{equation}
    \frac{190,000 \, t}{3,100 \, t} = 61
  \end{equation}
  \caption{Lager-til-flow-forholdet for guld}
  \label{fig:stock-to-flow-gold}
\end{figure}

Intet har en højere lager-til-flow-forhold end guld. Dette er grunden til, at 
guld indtil nu var den hårdeste, mest lydende valuta, der eksisterede. Det 
bliver ofte sagt, at alt det guld der hidtil er udvundet, ville kunne være i 
to olympiske swimmingpools. Ifølge mine beregninger
\footnote{\url{https://bit.ly/gold-pools}}, ville vi have brug for fire. Så
måske skal dette opdateres, eller olympiske swimmingpools er blevet mindre.

Indtast Bitcoin. Som du sandsynligvis ved, var bitcoin-minedrift alting i
de sidste par år. Dette skyldes, at vi stadig er i de tidlige
faser af det, der kaldes \textit{belønningsæraen}, hvor minedriftsnoder
belønnes med \textit{meget} bitcoin for deres beregningsindsats. Vi er
i øjeblikket i belønningsæra nummer 3, som begyndte i 2016 og slutter i
tidlig 2020, sandsynligvis i maj. Mens bitcoin-udbuddet er forudbestemt,
tillader Bitcoin's indre funktioner kun omtrentlige datoer.
Ikke desto mindre kan vi med sikkerhed forudsige, hvor høj Bitcoins
lager-til-flow-forhold vil være. Advarsel: det vil være højt.

Hvor højt? Nå, det viser sig, at Bitcoin vil blive uendeligt svært (se
Figur~\ref{fig:stock-to-flow-white-cropped}).

\begin{figure}[htbp]
  \centering
  \includegraphics[width=\textwidth]{assets/images/stock-to-flow-white-cropped.png}
  \caption{Visualisering af lager og flow for USD, guld og Bitcoin}
  \label{fig:stock-to-flow-white-cropped}
\end{figure}

\paragraph{}
På grund af en eksponentiel reduktion af minedriftsbelønningen vil 
tilstrømningen af nye bitcoin aftage, hvilket resulterer i en himmelflugt i 
lager-til-flow-forholdet. Det vil indhente guld i 2020, kun for at overgå det 
fire år senere ved at fordoble dets soliditet igen. En sådan fordobling vil 
forekomme 64 gange i alt. Takket være eksponentiernes kraft vil antallet af 
minedriftsbitcoin per år falde under 100 bitcoin om 50 år og under 1 bitcoin om
75 år. Den globale hane, som er blokbelønningen, vil tørre ud omkring år 2140
og stoppe effektivt produktionen af bitcoin. Dette er et langt spil. Hvis du 
læser dette, er du stadig tidligt på den.

\begin{figure}[htbp]
  \centering
  \includegraphics[width=\textwidth]{assets/images/soundness-over-time.png}
  \caption{Stigende lager-til-flow-forhold for bitcoin sammenlignet med guld}
  \label{fig:soundness-over-time}
\end{figure}

Når bitcoin nærmer sig uendeligt lager-til-flow-forhold, vil det være den
lydende valuta, der nogensinde har eksisteret. Uendelig soliditet er svær at 
slå.

Set gennem økonomiens linse er Bitcoin's \textit{sværhedsjustering}
formentlig dens vigtigste komponent. Hvor svært det er at mine bitcoin afhænger
af, hvor hurtigt nye bitcoins mines.\footnote{Det afhænger faktisk af, hvor
hurtigt gyldige blokke findes, men til vores formål er dette det samme som
\enquote{at mine bitcoins} og vil være det i de næste 120 år.} Det er den 
dynamiske justering af netværkets minedifficulty, der gør det muligt for os at 
forudsige dens fremtidige udbud.

Simpliciteten af sværhedsjusteringsalgoritmen kan distrahere fra dens dybde,
men sværhedsjusteringen er virkelig en revolution af Einsteinianiske 
proportioner. Den sikrer, at uanset hvor meget eller hvor lidt indsats der 
lægges i minedrift, vil Bitcoin's kontrollerede udbud ikke blive forstyrret. 
I modsætning til enhver anden ressource, uanset hvor meget energi nogen vil 
investere i at mine bitcoin, vil den samlede belønning ikke stige.

Præcis som $E=mc^2$ dikterer den universelle hastighedsgrænse i vores univers,
dikterer Bitcoin's sværhedsjustering \textbf{den universelle pengelimit}
i Bitcoin.

\paragraph{}
Hvis det ikke var for denne sværhedsjustering, ville alle bitcoins allerede 
være minedriftet. Hvis det ikke var for denne sværhedsjustering, ville Bitcoin 
sandsynligvis ikke have overlevet i sin barndom. Det er det, der sikrer 
netværket i dets belønningsæra. Det er det, der sikrer en stabil og retfærdig 
fordeling\footnote{Dan Held, 
\textit{Bitcoin's Distribution was Fair}~\cite{distribution-was-fair}} af nye
bitcoin. Det er termostaten, der regulerer Bitcoin's pengepolitik.

Einstein viste os noget nyt: uanset hvor hårdt du skubber til et
objekt, vil du på et vist tidspunkt ikke være i stand til at få mere 
hastighed ud af det. Satoshi viste os også noget nyt: uanset hvor hårdt du 
graver efter dette digitale guld, vil du på et vist tidspunkt ikke være i stand 
til at få mere bitcoin ud af det. For første gang i menneskets historie har vi 
en pengemæssig vare, som uanset hvor meget du prøver, ikke vil være i stand til
at producere mere af.

\paragraph{Bitcoin lærte mig, at lydende penge er essentielle.}

% ---
%
% #### Through the Looking-Glass
%
% - [Bitcoin's Energy Consumption: A Shift in Perspective][much energy]
%
% #### Down the Rabbit Hole
%
% - [The Ethics of Money Production][Jörg Guido Hülsmann] by Jörg Guido Hülsmann
% - [Mineral Commodity Summaries 2019][last few years] by the United States Geological Survey
% - [Bitcoin’s Distribution was Fair][fair distribution] by Dan Held
% - [Bitcoin's Controlled Supply][algorithmically controlled] on the Bitcoin Wiki
% - [Money Supply][how much money there is], [Speed of Light][universal speed limit] on Wikipedia
%
% <!-- Internal -->
% [much energy]: 
%
% [Fr. Bernard W. Dempsey, S.J.]: https://www.jstor.org/stable/29769582
% [Jörg Guido Hülsmann]: https://mises.org/sites/default/files/The%20Ethics%20of%20Money%20Production_2.pdf
% [stopped publishing]: https://www.federalreserve.gov/Releases/h6/discm3.htm
% [last few years]: https://minerals.usgs.gov/minerals/pubs/mcs/2018/mcs2018.pdf
% [my calculations]: https://www.wolframalpha.com/input/?i=volume+of+190000+metric+tons+gold+%2F+olympic+swimming+pool+volume
% [fair distribution]: https://blog.picks.co/bitcoins-distribution-was-fair-e2ef7bbbc892
%
% <!-- Bitcoin Wiki -->
% [algorithmically controlled]: https://en.bitcoin.it/wiki/Controlled_supply
%
% <!-- Wikipedia -->
% [how much money there is]: https://en.wikipedia.org/wiki/Money_supply
% [universal speed limit]: https://en.wikipedia.org/wiki/Speed_of_light#Upper_limit_on_speeds
% [alice]: https://en.wikipedia.org/wiki/Alice%27s_Adventures_in_Wonderland
% [carroll]: https://en.wikipedia.org/wiki/Lewis_Carroll
