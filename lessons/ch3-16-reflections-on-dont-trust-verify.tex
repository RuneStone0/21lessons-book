\chapter{Refleksioner om \enquote{Don't Trust, Verify}}
\label{les:16}

\begin{chapquote}{Lewis Carroll, \textit{Alice i Eventyrland}}
\enquote{Nu til beviset,} sagde Kongen, \enquote{og så dommen.}
\end{chapquote}

Bitcoin sigter mod at erstatte, eller i det mindste give et alternativ til,
konventionel valuta. Konventionel valuta er bundet til en centraliseret
myndighed, uanset om vi taler om lovligt betalingsmiddel som den amerikanske
dollar eller moderne monopolpenge som Fortnite's V-Bucks. I begge
eksempler er du bundet til at stole på den centrale myndighed for at udstede, administrere
og cirkulere dine penge. Bitcoin løsner denne binding, og hovedproblemet
som Bitcoin løser, er tillidsproblemet.

\begin{quotation}\begin{samepage}
\enquote{Det grundlæggende problem med konventionel valuta er al den tillid, der er nødvendig for at få det til at fungere. [...] Hvad der er nødvendigt, er et elektronisk betalingssystem baseret på kryptografisk bevis i stedet for tillid.}
\begin{flushright} -- Satoshi Nakamoto\footnote{Satoshi Nakamoto, officiel Bitcoin-annoncering~\cite{bitcoin-announcement} og whitepaper~\cite{whitepaper}}
\end{flushright}\end{samepage}\end{quotation}

Bitcoin løser tillidsproblemet ved at være fuldstændig decentraliseret,
uden central server eller betroede parter. Ikke engang betroede \textit{tredje}
parter, men betroede parter, punktum. Når der ikke er nogen central
myndighed, er der ganske enkelt ingen at stole på. Total decentralisering
er innovationen. Det er roden til Bitcoin's modstandsdygtighed, årsagen
til at det stadig er i live. Decentralisering er også grunden til, at vi har mining,
noder, hardwarewallets, og ja, blockchain. Det eneste, du
skal \enquote{stole på}, er, at vores forståelse af matematik og fysik
ikke er helt ude af kurs, og at flertallet af minearbejdere handler ærligt (hvad de har incitament til at gøre).

Mens den almindelige verden opererer under antagelsen om \textit{\enquote{stol
på, men verificér,}} opererer Bitcoin under antagelsen om \textit{\enquote{stol
ikke på, verificér.}} Satoshi gjorde vigtigheden af at fjerne tillid meget tydelig både i
introduktionen og konklusionen af Bitcoin whitepaper.

\begin{quotation}\begin{samepage}
\enquote{Konklusion: Vi har foreslået et system til elektroniske transaktioner
uden at stole på tillid.}
\begin{flushright} -- Satoshi Nakamoto\footnote{Satoshi Nakamoto, Bitcoin whitepaper~\cite{whitepaper}}
\end{flushright}\end{samepage}\end{quotation}

Bemærk, at \textit{uden at stole på tillid} bruges i en meget specifik kontekst
her. Vi taler om betroede tredjeparter, dvs. andre enheder,
som du stoler på at producere, opbevare og behandle dine penge. Det antages f.eks.,
at du kan stole på din computer.

Som Ken Thompson viste i sin Turing Award-forelæsning, er tillid en
ekstremt vanskelig ting i den beregningsmæssige verden. Når du kører et
program, er du nødt til at stole på al slags software (og hardware), som
i teorien kunne ændre programmet, du forsøger at køre, på en ondsindet
måde. Som Thompson opsummerede i sin \textit{Refleksioner om at stole på tillid}:
\enquote{Moralen er åbenlys. Du kan ikke stole på kode, som du ikke har skabt
helt selv.}~\cite{trusting-trust}

\begin{center}
  \includegraphics[width=\textwidth]{assets/images/ken-thompson-hack.png}
  \captionof{figure}{Uddrag fra Ken Thompsons papir 'Refleksioner om at stole på tillid'}
  \label{fig:ken-thompson-hack}
\end{center}

Thompson viste, at selv hvis du har adgang til kildekoden,
kan din kompilator --- eller enhver anden programbehandlings- eller
hardwareprogram --- være kompromitteret, og det ville være
meget vanskeligt at opdage denne bagdør. Således eksisterer der i praksis ikke
et virkelig \textit{trustless} system. Du ville være nødt til at skabe alt din software \textit{og} alt din
hardware (assemblers, kompilatorer, linkers, osv.) fra bunden,
uden hjælp fra nogen ekstern software eller software-understøttet maskineri.

\begin{quotation}\begin{samepage}
\enquote{Hvis du ønsker at lave en æbletærte fra bunden, skal du først opfinde
universet.}
\begin{flushright} -- Carl Sagan\footnote{Carl Sagan, \textit{Cosmos} \cite{cosmos}}
\end{flushright}\end{samepage}\end{quotation}

Ken Thompson Hack er en særligt genial og svær at opdage bagdør,
så lad os hurtigt se på en svær at opdage bagdør, der fungerer uden at
modificere nogen software. Forskere fandt en måde at kompromittere sikkerhedskritisk
hardware ved at ændre polariteten af siliciumforureninger.~\cite{becker2013stealthy} Ved blot at ændre de fysiske egenskaber
af det materiale, som computerchips er lavet af, lykkedes det dem at kompromittere en
kryptografisk sikker tilfældig talgenerator. Da denne ændring ikke kan ses,
kan bagdøren ikke opdages ved optisk inspektion, hvilket er en af de
vigtigste metoder til at opdage manipulation af chips som disse.

\begin{center}
  \includegraphics[width=\textwidth]{assets/images/stealthy-hardware-trojan.png}
  \captionof{figure}{Stealthy Dopant-Level Hardware Trojans af Becker, Regazzoni, Paar, Burleson}
  \label{fig:stealthy-hardware-trojan}
\end{center}

Lyder det skræmmende? Nå, selv hvis du var i stand til at bygge alt fra
bunden, ville du stadig være nødt til at stole på den underliggende matematik. Du
ville være nødt til at stole på, at \textit{secp256k1} er en elliptisk kurve uden
bagdøre. Ja, ondsindede bagdøre kan indsættes i de matematiske
grundlag af kryptografiske funktioner, og man kan argumentere for, at dette allerede
er sket mindst én gang.~\cite{wiki:Dual_EC_DRBG} Der er gode grunde til at være paranoid, og
det faktum, at alt lige fra din hardware, til din software, til de
elliptiske kurver, der bruges, kan have bagdøre~\cite{wiki:backdoors}, er nogle af dem.

\begin{quotation}\begin{samepage}
  \enquote{Don't trust. Verify.}
  \begin{flushright} -- Bitcoiners everywhere
\end{flushright}\end{samepage}\end{quotation}

De ovenstående eksempler burde illustrere, at \textit{trustless} computing er
utopisk. Bitcoin er sandsynligvis det system, der kommer tættest på denne
utopi, men det er stadig \textit{trust-minimized} --- med det formål at fjerne tillid
hvor det er muligt. Man kan argumentere for, at kæden af tillid aldrig ender, da
du også skal stole på, at beregning kræver energi, at P ikke er lig med NP, og at du rent faktisk er i virkeligheden og ikke
fanget i en simulering af ondsindede aktører.

Udviklere arbejder på værktøjer og procedurer for at minimere al resterende tillid
endnu mere. For eksempel skabte Bitcoin-udviklere
Gitian\footnote{\url{https://gitian.org/}}, som er en metode til softwaredistribution
for at skabe deterministiske builds. Ideen er, at hvis flere udviklere
er i stand til at reproducere identiske binære filer, reduceres risikoen for ondsindet manipulation.
Fancy bagdøre er ikke den eneste angrebsvektor. Enkle trusler om
afpresning er også reelle trusler. Ligesom i hovedprotokollen bruges decentralisering
til at minimere tillid.

Der gøres forskellige bestræbelser for at forbedre på hønen-og-ægget-problemet ved
bootstrapping, som Ken Thompsons hack så brillant påpegede~\cite{web:bootstrapping}. En sådan indsats er
Guix\footnote{\url{https://guix.gnu.org}} (udtales \textit{geeks}), som
bruger funktionelt deklareret pakkehåndtering, hvilket fører til bit-for-bit
reproducerbare builds efter design. Resultatet er, at du ikke længere behøver at stole på nogen
software-leverende servere, da du kan verificere, at den serverede binære fil
ikke er blevet manipuleret ved at genopbygge den fra bunden. For nylig blev der
flet en pull-anmodning for at integrere Guix i Bitcoin-buildprocessen.\footnote{Se PR 15277 af \texttt{bitcoin-core}: \\ \url{https://github.com/bitcoin/bitcoin/pull/15277}}

\begin{center}
  \includegraphics[width=\textwidth]{assets/images/guix-bootstrap-dependencies.png}
  \captionof{figure}{Hvad kom først, hønen eller ægget?}
  \label{fig:guix-bootstrap-dependencies}
\end{center}

 heldigvis er Bitcoin ikke afhængig af en enkelt algoritme eller stykke
hardware. En effekt af Bitcoins radikale decentralisering er en
distribueret sikkerhedsmodel. Selvom bagdøre beskrevet ovenfor ikke skal tages let,
er det usandsynligt, at hver softwarewallet,
hver hardwarewallet, hver kryptografiske bibliotek, hver noderealisering
og hver kompilator af hvert sprog er kompromitteret.
Muligt, men meget usandsynligt.

Bemærk, at du kan generere en privat nøgle uden at stole på nogen beregningsmæssig
hardware eller software. Du kan kaste en mønt~\cite{antonopoulos2014mastering} et
par gange, selvom afhængigt af din mønt og kastestil kan denne kilde
af tilfældighed måske ikke være tilstrækkelig tilfældig. Der er en grund til, at lagringsprotokoller som Glacier\footnote{\url{https://glacierprotocol.org/}} råder til
at bruge terninger af kasino-kvalitet som en af to kilder til entropi.

Bitcoin tvang mig til at reflektere over, hvad det egentlig indebærer ikke at stole på nogen.
Det øgede min bevidsthed om bootstrapping-problemet og den implicitte
kæde af tillid ved udvikling og kørsel af software. Det øgede også min
bevidsthed om de mange måder, hvorpå software og hardware kan blive
kompromitteret.

\paragraph{Bitcoin lærte mig ikke at stole, men at verificere.}

% ---
%
% #### Down the Rabbit Hole
%
% - [The Bitcoin whitepaper][Nakamoto] by Satoshi Nakamoto
% - [Reflections on Trusting Trust][\textit{Reflections on Trusting Trust}] by Ken Thompson
% - [51% Attack][majority] on the Bitcoin Developer Guide
% - [Bootstrapping][bootstrapping], Guix Manual
% - [Secp256k1][secp256k1] on the Bitcoin Wiki
% - [ECC Backdoors][backdoors], [Dual EC DRBG][has already happened] on Wikipedia
%
% [Emmanuel Boutet]: https://commons.wikimedia.org/wiki/User:Emmanuel.boutet
% [\textit{Reflections on Trusting Trust}]: https://www.archive.ece.cmu.edu/~ganger/712.fall02/papers/p761-thompson.pdf
% [found a way]: https://scholar.google.com/scholar?hl=en&as_sdt=0%2C5&q=Stealthy+Dopant-Level+Hardware+Trojans&btnG=
% [Gitian]: https://gitian.org/
% [bootstrapping]: https://www.gnu.org/software/guix/manual/en/html_node/Bootstrapping.html
% [Guix]: https://www.gnu.org/software/guix/
% [pull-request]: https://github.com/bitcoin/bitcoin/pull/15277
% [flip a coin]: https://github.com/bitcoinbook/bitcoinbook/blob/develop/ch04.asciidoc#private-keys
% [Glacier]: https://glacierprotocol.org/
% [secp256k1]: https://en.bitcoin.it/wiki/Secp256k1
% [majority]: https://bitcoin.org/en/developer-guide#term-51-attack
%
% <!-- Wikipedia -->
% [backdoors]: https://en.wikipedia.org/wiki/Elliptic-curve_cryptography#Backdoors
% [has already happened]: https://en.wikipedia.org/wiki/Dual_EC_DRBG
% [Carl Sagan]: https://en.wikipedia.org/wiki/Cosmos_%28Carl_Sagan_book%29
% [alice]: https://en.wikipedia.org/wiki/Alice%27s_Adventures_in_Wonderland
% [carroll]: https://en.wikipedia.org/wiki/Lewis_Carroll
