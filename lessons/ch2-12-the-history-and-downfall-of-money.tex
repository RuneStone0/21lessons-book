\chapter{Historien og Nedgangen af Penge}
\label{les:12}

\begin{chapquote}{Lewis Carroll, \textit{Alice i Eventyrland}}
\enquote{De ville ikke huske de simple regler, deres venner havde givet dem, 
såsom, at hvis du kommer ind i ilden, vil den brænde dig, og at hvis du skærer 
din finger dybt med en kniv, bløder det generelt, og hun havde aldrig glemt, 
at hvis du drikker af en flaske mærket 'gift,' er det næsten sikkert, at du 
er uenig med det, før eller senere.}
\end{chapquote}

Mange mennesker tror, at penge er bakket op af guld, som er låst inde i store 
kasser, beskyttet af tykke mure. Dette holdt op med at være sandt for mange 
årtier siden. Jeg er ikke sikker på, hvad jeg tænkte, da jeg var i meget 
dybere problemer og stort set ingen forståelse af guld, papirpenge eller 
hvorfor det overhovedet skulle bakkes op af noget.

En del af at lære om Bitcoin er at lære om fiat-penge: hvad det betyder, 
hvordan det kom til at være, og hvorfor det måske ikke er den bedste idé, 
vi nogensinde har haft. Så hvad er præcis fiat-penge? Og hvordan endte vi 
med at bruge det?

Hvis noget pålægges ved \textit{fiat}, betyder det simpelthen, at det pålægges 
ved formel tilladelse eller forslag. Derfor er fiat-penge penge simpelthen 
fordi \textit{nogen} siger, at det er penge. Da alle regeringer i dag bruger 
fiat-valuta, er dette nogen \textit{din} regering. Desværre er du ikke 
\textit{fri} til at være uenig i denne værdi. Du vil hurtigt føle, at dette 
forslag er alt andet end ikke-voldeligt. Hvis du nægter at bruge denne 
papirvaluta til at drive forretning og betale skatter, vil de eneste mennesker, 
du vil kunne diskutere økonomi med, være dine cellekammerater.

Værdien af fiat-penge stammer ikke fra dens iboende egenskaber. Hvor god en 
bestemt type fiat-penge er, korrelerer kun med politisk og økonomisk 
(u)stabilitet hos dem, der drømmer det til eksistens. Dens værdi pålægges ved 
dekret, vilkårligt.

\begin{center}
  \centering
  \includegraphics[width=8cm]{assets/images/fiat-definition.png}
  \captionof{figure}{fiat --- 'Lad det ske'}
  \label{fig:fiat-definition}
  \end{center}
  
\paragraph{}
Indtil for nylig blev to typer penge brugt: \textbf{råvaremønt}, lavet
af kostbare \textit{sager}, og \textbf{repræsentativ mønt}, som simpelthen
\textit{repræsenterer} den kostbare ting, mest i skrift.

\paragraph{}
Vi rørte allerede ved råvaremønt ovenfor. Folk brugte specielle knogler,
muslingeskaller og kostbare metaller som penge. Senere blev især mønter lavet af
kostbare metaller som guld og sølv brugt som penge. Den ældste mønt, der hidtil 
er fundet, er lavet af en naturlig blanding af guld og sølv og blev lavet for 
mere end 2700 år siden.\footnote{Ifølge den græske historiker Herodot, der 
skrev i det femte århundrede f.Kr., var lyderne de første, der brugte mønter
af guld og sølv.\cite{coinage-origins}} Hvis der er noget nyt i Bitcoin, er
 møntens koncept det ikke.

\newpage

\begin{center}
  \centering
  \includegraphics[width=5cm]{assets/images/lydian-coin-stater.png}
  \captionof{figure}{Lydian elektromønt. Billede cc-by-sa Classical Numismatic
   Group, Inc.}
  \label{fig:lydian-coin-stater}
\end{center}

Viser sig, at at gemme mønter, eller hodle, for at bruge dagens sprogbrug, er 
næsten lige så gammelt som mønter selv. Den tidligste mønt-hodler var en 
person, der lagde næsten hundrede af disse mønter i en gryde og begravde den 
i fundamentet af et tempel, kun for at blive fundet 2500 år senere. Ret god 
kold opbevaring, hvis du spørger mig.

En af ulemperne ved at bruge mønter af kostbare metaller er, at de kan 
klippes, hvilket effektivt devaluerer møntens værdi. Nye mønter kan præges 
af klipningerne, hvilket inflerer pengeudbuddet over tid og devaluerer hver 
enkelt mønt i processen. Folk skar bogstaveligt talt så meget af som de 
kunne slippe afsted med af deres sølvdollars. Jeg undrer mig over, hvilken 
slags \textit{Dollar Shave Club}-annoncer de havde dengang.

Da regeringer kun er cool med inflation, hvis de er dem, der gør det, blev 
der gjort forsøg på at stoppe denne gerilladebasering. I klassisk 
politi-og-røvere-stil blev møntklippere stadig mere kreative med deres 
teknikker, hvilket tvang \enquote{møntmestrene} til at blive endnu mere 
kreative med deres modforanstaltninger. Isaac Newton, den verdensberømte 
fysiker kendt for \textit{Principia Mathematica}, plejede at være en af 
disse mestre. Han tilskrives tilføjelsen af de små striber på mønternes 
side, som stadig er til stede i dag. Slut var dagene med let møntskæring.

\begin{center}
  \includegraphics[width=\textwidth]{assets/images/clipped-coins.png}
  \captionof{figure}{Klippede sølvmønter af varierende grad.}
  \label{fig:clipped-coins}
\end{center}

Selv med disse metoder til møntdebasering\footnote{Udover klipning var svedning
(ryste mønterne i en pose og indsamle det slidte støv) og propning
(udhule en mønt i midten og banke mønten flad for at lukke hullet)
de mest fremtrædende metoder til møntdebasering. \cite{wiki:coin-debasement}}
holdt i skak lider mønter stadig af andre problemer. De er klodsede og ikke 
særlig praktiske at transportere, især når der skal foretages store
værdioverførsler. Det er ikke særlig praktisk at dukke op med en kæmpe pose 
sølvdollars hver gang du vil købe en Mercedes.

Når vi taler om tyske ting: Hvordan den amerikanske \textit{dollar} fik sit 
navn, er en anden interessant historie. Ordet \enquote{dollar} stammer fra det 
tyske ord \textit{Thaler}, forkortelse for 
\textit{Joachimsthaler}~\cite{wiki:thaler}. En Joachimsthaler var en mønt
præget i byen \textit{Sankt Joachimsthal}. Thaler er simpelthen en forkortelse 
for nogen (eller noget), der kommer fra dalen, og fordi Joachimsthal var 
\textit{dalen} for produktion af sølvmønter, kaldte folk simpelthen disse 
sølvpenge for \textit{Thaler.} Thaler (tysk) udviklede sig til daalders 
(hollandsk) og endelig dollars (engelsk).


\begin{center}
  \centering
  \includegraphics[width=5cm]{assets/images/joachimsthaler.png}
  \captionof{figure}{Den originale 'dollar'. Saint Joachim er afbildet med 
  sin robe og troldmandshat. Billede cc-by-sa Wikipedia-bruger Berlin-George}
  \label{fig:joachimsthaler}
\end{center}

Indførelsen af repræsentativ mønt markerede nedgangen for hårde
penge. Guldcertifikater blev indført i 1863, og cirka femten
år senere begyndte sølvdollaren også langsomt, men sikkert at blive
erstattet af en papirproxy: sølvcertifikatet. \cite{wiki:silver-certificate}

Det tog cirka 50 år fra indførelsen af de første sølvcertifikater, indtil
disse stykker papir udviklede sig til noget, vi i dag ville genkende som én
amerikansk dollar.

\begin{center}
  \centering
  \includegraphics[width=\textwidth]{assets/images/us-silver-dollar-note-smaller.png}
  \captionof{figure}{En amerikansk sølvdollarseddel fra 1928. 'Betalelig til 
  bæreren på forlangende.' Billede cc-by-sa National Numismatic Collection 
  ved Smithsonian Institution}
  \label{fig:us-silver-dollar-note-smaller}
\end{center}

Bemærk, at den amerikanske sølvdollar fra 1928 i
Figur~\ref{fig:us-silver-dollar-note-smaller} stadig går under navnet
\textit{sølvcertifikat}, hvilket indikerer, at dette faktisk bare er et 
dokument, der angiver, at bæreren af dette stykke papir skylder en sølvmønt. 
Det er interessant at se, at teksten, der angiver dette, blev mindre over tid. 
Sporet af \enquote{certifikat} forsvandt fuldstændigt efter et stykke tid og 
blev erstattet af forsikringen om, at disse er føderale reserve sedler.

Som nævnt ovenfor skete det samme med guld. Størstedelen af verden anvendte en
bimetallisk standard~\cite{wiki:bimetallism}, hvilket betyder, at mønter primært
var lavet af guld og sølv. At have certifikater for guld, indløselige i
guld mønter, var nok en teknologisk forbedring. Papir er mere praktisk,
lettere, og da det kan opdeles arbitrært ved blot at trykke et mindre
nummer på det, er det nemmere at opdele i mindre enheder.

For at minde bærerne (brugerne) om, at disse certifikater var
repræsentative for faktisk guld og sølv, blev de farvet derefter
og angav tydeligt dette på selve certifikatet. Du kan glidende læse
teksten fra top til bund:

\begin{quotation}\begin{samepage}
  \enquote{Dette bekræfter, at der er deponeret i skatkammeret for
  Amerikas Forenede Stater hundrede dollars i guld mønt, betalbar til
  bæreren på forlangende.}
\end{samepage}\end{quotation}
  
\begin{center}
  \centering
  \includegraphics[width=\textwidth]{assets/images/us-gold-cert-100-smaller.png}
  \captionof{figure}{En amerikansk 100 dollars guldcertifikat fra 1928. 
  Billede cc-by-sa National Numismatic Collection, National Museum of American
  History.}
  \label{fig:us-gold-cert-100-smaller}
\end{center}

I 1963 blev ordene \enquote{BETALBAR TIL BÆREREN PÅ FORLANGENDE} fjernet fra
alle nyudstedte sedler. Fem år senere blev indløsningen af papirsedler
til guld og sølv afsluttet.

Ordene, der antydede oprindelsen og ideen bag papirpenge, blev
fjernet. Den gyldne farve forsvandt. Alt, hvad der var tilbage, var papiret
og med det evnen for regeringen til at trykke så meget af det, den ønsker.

Med afskaffelsen af guldfoden i 1971 var denne århundredlange
illusion fuldendt. Penge blev den illusion, vi alle deler den dag i dag: 
fiat-penge. Det er værdifuldt, fordi nogen, der kommanderer en hær og driver 
fængsler, siger, at det er værdifuldt. Som det tydeligt kan læses på hver 
dollar seddel i omløb i dag, \enquote{DENNE SEDDEL ER LEGALT BETALINGSMIDDEL}. 
Med andre ord: Den har værdi, fordi sedlen siger det.

\begin{center}
  \centering
  \includegraphics[width=\textwidth]{assets/images/us-dollar-2004.jpg}
  \captionof{figure}{En 2004-serie af amerikanske tyve dollarsedler, der 
  bruges i dag. 'DENNE SEDDEL ER LEGALT BETALINGSMIDDEL'}
  \label{fig:us-dollar-2004}
\end{center}
  
I øvrigt er der en anden interessant lektion på dagens sedler,
skjult lige for øjnene af os. Den anden linje siger, at dette er legalt 
betalingsmiddel \enquote{TIL ALLE GÆLD, OFFENTLIG OG PRIVAT}. Hvad der måske 
er åbenlyst for økonomer, var overraskende for mig: Alle penge er gæld. Mit 
hoved gør stadig ondt på grund af det, og jeg vil lade udforskningen af 
forholdet mellem penge og gæld være en øvelse for læseren.

\paragraph{}
Som vi har set, blev guld og sølv brugt som penge i årtusinder. Over tid
blev mønter lavet af guld og sølv erstattet af papir. Papir
blev langsomt accepteret som betaling. Denne accept skabte en
illusion --- illusionen om, at papiret selv har værdi. Det endelige
skridt var at fuldstændigt bryde forbindelsen mellem repræsentationen og
det faktiske: at afskaffe guldfoden og overbevise alle om, at
papiret i sig selv er kostbart.

\paragraph{Bitcoin lærte mig om historien om penge og det største tryllenummer
i økonomisk historie: fiat-valuta.}

% ---
%
% #### Down the Rabbit Hole
%
% - [Shelling Out: The Origins of Money] by Nick Szabo
% - [Methods of Coin Debasement][coin debasement], [Thaler], [U.S. Silver Certificate][silver certificates], [Bimetallism][bimetallic standard] on Wikipedia
%
% [oldest coin]: https://www.britishmuseum.org/explore/themes/money/the_origins_of_coinage.aspx
% [coin debasement]: https://en.wikipedia.org/wiki/Methods_of_coin_debasement
% [Thaler]: https://en.wikipedia.org/wiki/Thaler
% [Berlin-George]: https://en.wikipedia.org/wiki/File:Bohemia,_Joachimsthaler_1525_Electrotype_Copy._VF._Obverse..jpg
% [silver certificates]: https://en.wikipedia.org/wiki/Silver_certificate_%28United_States%29
% [bimetallic standard]: https://en.wikipedia.org/wiki/Bimetallism
% [Shelling Out: The Origins of Money]: https://nakamotoinstitute.org/shelling-out/
%
% <!-- Wikipedia -->
% [alice]: https://en.wikipedia.org/wiki/Alice%27s_Adventures_in_Wonderland
% [carroll]: https://en.wikipedia.org/wiki/Lewis_Carroll
