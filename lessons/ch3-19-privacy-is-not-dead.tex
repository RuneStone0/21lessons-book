\chapter{Privatliv er Ikke Dødt}
\label{les:19}

\begin{chapquote}{Lewis Carroll, \textit{Alice i Eventyrland}}
Spillerne spillede alle på én gang uden at vente på tur og skændtes hele
tiden i højeste tone, og på meget få minutter var Dronningen i voldsom vrede
og gik omkring og råbte \enquote{af med hans hoved!} eller \enquote{af med hendes
hoved!} cirka en gang i minuttet.
\end{chapquote}

Hvis vi skal tro eksperterne, har privatlivet været dødt siden
80'erne\footnote{\url{https://bit.ly/privacy-is-dead}}. Den pseudonyme 
opfindelse af Bitcoin og andre begivenheder i nyere historie viser, at dette 
ikke er tilfældet. Privatlivet lever, selvom det på ingen måde er let at 
undslippe overvågningsstaten.

Satoshi gik langt for at skjule sine spor og skjule
sin identitet. Ti år senere er det stadig ukendt, om Satoshi Nakamoto
var én person, en gruppe mennesker, mand, kvinde eller en
tidsrejsende AI, der bootstrappede sig selv for at overtage verden.
Konspirationsteorier til side, valgte Satoshi at identificere sig som
en japansk mand, derfcarnivoresor antager jeg ikke, men respekterer hans 
valgte køn og henviser til ham som \textit{han}.

\begin{center}
  \includegraphics[width=\textwidth]{assets/images/nope.png}
  \captionof{figure}{Jeg er ikke Dorian Nakamoto.}
  \label{fig:nope}
\end{center}

Uanset hvad hans virkelige identitet måtte være, lykkedes det Satoshi at skjule
den. Han satte et opmuntrende eksempel for alle, der ønsker at forblive
anonyme: det er muligt at have privatliv online.

\begin{quotation}\begin{samepage}
\enquote{Kryptering virker. Ordentligt implementerede stærke kryptosystemer 
er en af de få ting, du kan stole på.}
\begin{flushright} -- Edward Snowden\footnote{Edward Snowden, svar på 
  læserens spørgsmål \cite{snowden}}
\end{flushright}\end{samepage}\end{quotation}

Satoshi var ikke den første pseudonyme eller anonyme opfinder, og han vil 
ikke være den sidste. Nogle har direkte efterlignet denne pseudonyme 
publiceringsstil, som Tom Elvis Yedusor fra 
MimbleWimble~\cite{mimblewimble-origin} berømmelse, mens andre har
offentliggjort avancerede matematiske beviser og forblevet helt
anonyme~\cite{4chan-math}.

Det er en mærkelig ny verden, vi lever i. En verden, hvor identitet er
valgfri, bidrag accepteres baseret på fortjeneste, og folk kan
samarbejde og handle frit. Det vil tage lidt tilvænning at blive
komfortabel med disse nye paradigmer, men jeg tror stærkt på, at alt dette 
har potentiale til at ændre verden til det bedre.

Vi bør alle huske på, at privatliv er en grundlæggende menneskeret
\footnote{Universal Declaration of Human Rights, 
\textit{Artikel 12}.~\cite{article12}}. Og så længe
mennesker udøver og forsvarer disse rettigheder, er kampen for privatlivet 
langt fra ovre.

\paragraph{Bitcoin lærte mig, at privatlivet ikke er dødt.}

% ---
%
% #### Down the Rabbit Hole
%
% - [Universal Declaration of Human Rights][fundamental human right] by the United Nations
% - [A lower bound on the length of the shortest superpattern][anonymous] by Anonymous 4chan Poster, Robin Houston, Jay Pantone, and Vince Vatter
%
% [since the 80ies]: https://books.google.com/ngrams/graph?content=privacy+is+dead&year_start=1970&year_end=2019&corpus=15&smoothing=3&share=&direct_url=t1%3B%2Cprivacy%20is%20dead%3B%2Cc0
% [time-traveling AI]: https://blockchain24-7.com/is-crypto-creator-a-time-travelling-ai/
% ["I am not Dorian Nakamoto."]: http://p2pfoundation.ning.com/forum/topics/bitcoin-open-source?commentId=2003008%3AComment%3A52186
% [MimbleWimble]: https://github.com/mimblewimble/docs/wiki/MimbleWimble-Origin
% [anonymous]: https://oeis.org/A180632/a180632.pdf
% [fundamental human right]: http://www.un.org/en/universal-declaration-human-rights/
%
% <!-- Wikipedia -->
% [alice]: https://en.wikipedia.org/wiki/Alice%27s_Adventures_in_Wonderland
% [carroll]: https://en.wikipedia.org/wiki/Lewis_Carroll
