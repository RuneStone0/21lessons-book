\chapter{Replikation og Lokalitet}
\label{les:3}

\begin{chapquote}{Lewis Carroll, \textit{Alice i Eventyrland}}
Derefter kom en vred stemme -- kaninen's -- \enquote{Pat, Pat! hvor er du?}
\end{chapquote}

Ignorer kvantemekanik. Lokalitet er ikke et problem i den fysiske verden. 
Spørgsmålet \textit{\enquote{Hvor er X?}} kan let besvares, uanset om 
X er en person eller en genstand. I den digitale verden er spørgsmålet om 
\textit{hvor} allerede et svært spørgsmål, men ikke umuligt at besvare. Hvor er 
dine e-mails, egentlig? Et dårligt svar ville være \enquote{skyen}, hvilket 
bare er en anden persons computer. Alligevel, hvis du ønskede at spore hver 
lagringsenhed, der har dine e-mails, kunne du teoretisk set finde dem.

Med bitcoin er spørgsmålet om \enquote{hvor} \textit{rigtig} svært. Hvor er 
dine bitcoins, præcist?

\begin{quotation}\begin{samepage}
\enquote{Jeg åbnede mine øjne, kiggede rundt og stillede det uundgåelige, det 
traditionelle, det beklageligt kliché postoperative spørgsmål: `Hvor er jeg?'}
\begin{flushright} -- Daniel Dennett\footnote{Daniel Dennett, 
    \textit{Hvor Er Jeg?}~\cite{where-am-i}}
\end{flushright}\end{samepage}\end{quotation}

Problemet er tofoldigt: For det første distribueres den distribuerede logbog 
ved fuld replikation, hvilket betyder, at logbogen er overalt. For det andet 
er der ingen bitcoins. Ikke kun fysisk, men også \textit{teknisk}.

Bitcoin holder styr på en række ubrugte transaktioner uden nogensinde at 
skulle henvise til en enhed, der repræsenterer en bitcoin. Existensen af en 
bitcoin bestemmes ved at se på antallet af ubrugte transaktioner og ved at
gennemgå alle tidligere overførseler.

\begin{quotation}\begin{samepage}
\enquote{Hvor er den på dette tidspunkt, er den under vejs? [...] For det første 
er der ingen bitcoins. De er der bare ikke. De eksisterer ikke. Der er kun 
bogføringsposter i en fælles logbog [...] De eksisterer ikke fysisk nogen steder. 
Logbogen eksisterer over alt i mange fysiske lokationer. Geografi giver 
ikke mening her --- det vil ikke hjælpe dig med at forstå dette.}
\begin{flushright} -- Peter Van Valkenburgh\footnote{Peter Van Valkenburgh på 
    \textit{What Bitcoin Did}-podcast, episode 49 \cite{wbd049}}
\end{flushright}\end{samepage}\end{quotation}

Så hvad ejer du egentlig, når du siger \textit{\enquote{Jeg har en bitcoin}}, 
hvis der ikke er nogen bitcoins? Kan du husje alle disse mærkelige ord, som du
blev tvunget til at skrive ned? Det viser sig, at disse 
magiske ord er det, du ejer: en magisk formular\footnote{Cryptography's Magic 
Dust: How digital information is changing our society \cite{gigi:magic-spell}}, 
som kan bruges til at tilføje nogle poster til den offentlige logbog --- 
nøglerne til at \enquote{flytte} nogle bitcoins. Derfor er dine private nøgler 
\textit{for alle formål} dine bitcoins. Hvis du tror, det er noget jeg har 
fundet på for sjov, så er du velkommen til at sende mig dine private nøgler.

\paragraph{Bitcoin lærte mig, at lokalitet er en vanskelig sag.}

% ---
%
% #### Through the Looking-Glass
%
% - [The Magic Dust of Cryptography: How digital information is changing our society][a magic spell]
%
% #### Down the Rabbit Hole
%
% - [Where Am I?][Daniel Dennett] by Daniel Dennett
% - 🎧 [Peter Van Valkenburg on Preserving the Freedom to Innovate with Public Blockchains][wbd049] WBD #49 hosted by Peter McCormack
%
% <!-- Through the Looking-Glass -->
% [a magic spell]: 
%
% <!-- Down the Rabbit Hole -->
% [Daniel Dennett]: https://www.lehigh.edu/~mhb0/Dennett-WhereAmI.pdf
% [1st Amendment]: https://en.wikipedia.org/wiki/First_Amendment_to_the_United_States_Constitution
% [wbd049]: https://www.whatbitcoindid.com/podcast/coin-centers-peter-van-valkenburg-on-preserving-the-freedom-to-innovate-with-public-blockchains
%
% <!-- Wikipedia -->
% [alice]: https://en.wikipedia.org/wiki/Alice%27s_Adventures_in_Wonderland
% [carroll]: https://en.wikipedia.org/wiki/Lewis_Carroll
