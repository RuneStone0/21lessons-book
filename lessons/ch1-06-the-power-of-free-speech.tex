
\chapter{Ytringsfrihedens Magt}
\label{les:6}

\begin{chapquote}{Lewis Carroll, \textit{Alice i Eventyrland}}
\enquote{Undskyld mig?} sagde musen, rynkende på panden, men meget høfligt, 
\enquote{talte du?}
\end{chapquote}

Bitcoin er en idé. En idé, som i sin nuværende form er manifestationen af en 
mekanisme, der udelukkende er drevet af tekst. Hver eneste aspekt af Bitcoin 
er tekst: Det videnskabelige notat ("whitepaper") er tekst. Softwaren, der køres
af dens brugere, er tekst. Logbogen er tekst. Transaktionerne er tekst. 
Offentlige og private nøgler er tekst. Hver eneste aspekt af Bitcoin er tekst 
og dermed lig med ytringsfrihed.

\begin{quotation}\begin{samepage}
\enquote{Kongressen må ikke vedtage love, der vedrører en etablering af religion
eller forhindrer den frie udøvelse heraf; begrænse ytringsfriheden 
eller pressefriheden; eller rettigheden for folk at samles fredeligt og
fremsætte anmodninger til regeringen om at rette op på uretfærdigheder.}
\begin{flushright} -- Første tillæg til den amerikanske forfatning
\end{flushright}\end{samepage}\end{quotation}

Selvom den sidste kamp i Crypto Wars\footnote{\textit{Crypto Wars} er en 
uformel betegnelse for de amerikanske og allierede regeringers forsøg på at 
underminere kryptering.\cite{eff-cryptowars}\cite{wiki:cryptowars}} ikke er 
blevet udkæmpet endnu, vil det være meget svært at kriminalisere en idé, ikke 
mindst en idé, der er baseret på udvekslingen af tekstbeskeder. Hver gang en 
regering forsøger at forbyde tekst eller tale, glider vi ned ad en sti af 
absurditet, der uundgåeligt fører til grusomheder som ulovlige tal\footnote{Et 
ulovligt tal er et tal, der repræsenterer information, som det er ulovligt at 
besidde, udtale, sprede eller på anden måde transmittere i nogle retslige 
jurisdiktioner.\cite{wiki:illegal-number}} og ulovlige primtal\footnote{Et 
ulovligt primtal er et primtal, der repræsenterer information, hvis besiddelse 
eller distribution er forbudt i nogle retslige jurisdiktioner. Et af de første 
ulovlige primtal blev fundet i 2001. Når det fortolkes på en bestemt måde, 
beskriver det et computerprogram, der omgår den digitale retsstyring, der 
anvendes på DVD'er. Distribution af et sådant program i USA er ulovligt under 
Digital Millennium Copyright Act. Et ulovligt primtal er en form for ulovligt 
tal.\cite{wiki:illegal-prime}}.

Så længe der er en del af verden, hvor ytringsfrihed er fri som i 
\textit{frihed}, er Bitcoin ustoppelig.

\begin{quotation}\begin{samepage}
\enquote{Der er intet tidspunkt i nogen Bitcoin-transaktion, hvor Bitcoin
ophører med at eksistere. Det er \textit{alt tekst}, hele tiden.
[...] Bitcoin er \textit{tekst}. Bitcoin er \textit{ytring}. Det kan ikke 
reguleres i et frit land som USA med garanterede ukrænkelige rettigheder (First 
Amendment), der specifikt forhindre regeringer i at regulere pressefrihed.}
\begin{flushright} -- Beautyon\footnote{Beautyon, \textit{Hvorfor Amerika ikke 
    kan regulere Bitcoin} \cite{america-regulate-bitcoin}}
\end{flushright}\end{samepage}\end{quotation}

\paragraph{Bitcoin lærte mig, at i et frit samfund er fri ytring og fri 
software ustoppelige.}

% ---
%
% #### Through the Looking-Glass
%
% - [The Magic Dust of Cryptography: How digital information is changing our society][a magic spell]
%
% #### Down the Rabbit Hole
%
% - [Why America can't regulate Bitcoin][Beautyon] by Beautyon
% - [First Amendment to the United States Constitution][1st Amendment], [Crypto Wars], [illegal numbers], [illegal primes] on Wikipedia
%
% <!-- Through the Looking-Glass -->
% [a magic spell]: 
%
% <!-- Down the Rabbit Hole -->
% [1st Amendment]: https://en.wikipedia.org/wiki/First_Amendment_to_the_United_States_Constitution
% [Crypto Wars]: https://en.wikipedia.org/wiki/Crypto_Wars
% [illegal numbers]: https://en.wikipedia.org/wiki/Illegal_number
% [illegal primes]: https://en.wikipedia.org/wiki/Illegal_prime
% [Beautyon]: https://hackernoon.com/why-america-cant-regulate-bitcoin-8c77cee8d794
%
% <!-- Wikipedia -->
% [alice]: https://en.wikipedia.org/wiki/Alice%27s_Adventures_in_Wonderland
% [carroll]: https://en.wikipedia.org/wiki/Lewis_Carroll
