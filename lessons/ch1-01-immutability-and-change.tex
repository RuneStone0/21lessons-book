\chapter{Uforanderlighed og Forandring}
\label{les:1}

\begin{chapquote}{Alice}
\enquote{Jeg undrer mig over, om jeg er blevet forandret i løbet af natten. Lad mig tænke.
Var jeg den samme, da jeg stod op i morges? Jeg tror næsten, jeg kan huske at have følt mig
en smule anderledes. Men hvis jeg ikke er den samme, er det næste spørgsmål: 'Hvem i verden er jeg?' Ah,
det er den store gåde!}
\end{chapquote}

Bitcoin er i sin natur svær at beskrive. Det er en \textit{ny ting}, og enhver
forsøg på at lave en sammenligning med tidligere begreber - om det så er ved at kalde
det digitalt guld eller pengenes internet - vil uundgåeligt falde kort
af helheden. Uanset hvad din foretrukne analogi måtte være, er to aspekter af
Bitcoin helt essentielle: decentralisering og uforanderlighed.

\paragraph{}
En måde at tænke på Bitcoin er som en automatiseret social kontrakt\footnote{Hasu,
Unpacking Bitcoin's Social Contract~\cite{social-contract}}. Softwaren er
bare ét element i puslespillet, og håbet om at ændre Bitcoin ved at ændre
softwaren er en øvelse i futility. Man skulle overbevise resten af
netværket om at vedtage ændringerne, hvilket er mere en psykologisk indsats end en
softwareteknisk.

\paragraph{}
Det følgende kan lyde absurd ved første øjekast, ligesom så mange andre ting i
dette rum, men jeg tror alligevel dybt på, at det er sandt: Du
vil ikke ændre Bitcoin, men Bitcoin vil ændre dig.

\begin{quotation}\begin{samepage}
\enquote{Bitcoin vil ændre os mere, end vi vil ændre det.}
\begin{flushright} -- Marty Bent\footnote{Tales From the Crypt~\cite{tftc21}}
\end{flushright}\end{samepage}\end{quotation}

Det tog mig lang tid at indse dybden af dette. Da Bitcoin
bare er software, og alting er open source, kan du bare ændre
ting efter behov, ikke sandt? Forkert. \textit{Meget} forkert. Ikke overraskende vidste
Bitcoin's skaber alt for godt.

\begin{quotation}\begin{samepage}
\enquote{Bitcoin's natur er sådan, at når version 0.1 blev frigivet, var kernen
design fastlagt for resten af dens levetid.}
\begin{flushright} -- Satoshi Nakamoto\footnote{BitcoinTalk forumindlæg: 'Re:
Transactions and Scripts\ldots'~\cite{satoshi-set-in-stone}}
\end{flushright}\end{samepage}\end{quotation}

Mange har forsøgt at ændre Bitcoin's natur. Indtil videre er alle
fejlet. Mens der er et uendeligt hav af forgreninger og alternative mønter,
gør Bitcoin-netværket stadig sit arbejde, ligesom det gjorde, da den første
node gik online. Alternative mønter betyder ikke noget i det lange løb. Forgreningerne
vil til sidst sulte ihjel. Det er Bitcoin, der betyder noget. Så længe vores
grundlæggende forståelse af matematik og/eller fysik ikke ændres,
vil Bitcoin-honninggrævlingen fortsætte med ikke at bekymre sig.

\begin{quotation}\begin{samepage}
\enquote{Bitcoin er det første eksempel på en ny livsform. Den lever og ånder
på internettet. Den lever, fordi den kan betale folk for at holde
den i live. [\ldots] Den kan ikke ændres. Den kan ikke diskuteres. Den
kan ikke manipuleres. Den kan ikke korrumperes. Den kan ikke stoppes.
[\ldots] Hvis en atomkrig ødelagde halvdelen af vores planet, ville den fortsætte
med at leve, ukorrumperet.}
\begin{flushright} -- Ralph Merkle\footnote{DAOs, Democracy and
Governance,~\cite{merkle-dao}}
\end{flushright}\end{samepage}\end{quotation}

Bitcoin-netværkets hjerteslag vil overleve alle vores.

~

At indse det ovenstående ændrede mig meget mere end de seneste blokke i Bitcoin's
blockchain nogensinde vil gøre. Det ændrede min tidspræference, min forståelse af
økonomi, mine politiske synspunkter og meget mere. Faktisk ændrer det endda
menneskers kostvaner\footnote{Inside the World of the Bitcoin
Carnivores,~\cite{carnivores}}. Hvis alt dette lyder vanvittigt for dig, er du i
godt selskab. Alt dette er vanvittigt, og alligevel sker det.

~

\paragraph{Bitcoin lærte mig, at det ikke vil ændre sig. Jeg vil.}

% ---
%
% #### Through the Looking-Glass
%
% - [Bitcoin's Gravity: How idea-value feedback loops are pulling people in][gravity]
% - [Lesson 18: Move slowly and don't break things][lesson18]
%
% #### Down the Rabbit Hole
%
% - [Unpacking Bitcoin's Social Contract][automated social contract]: A framework for skeptics by Hasu
% - [DAOs, Democracy and Governance][Ralph Merkle] by Ralph C. Merkle
% - [Marty's Bent][bent]: A daily newsletter highlighting signal in Bitcoin by Marty Bent
% - [Technical Discussion on Bitcoin's Transactions and Scripts][Satoshi Nakamoto] by Satoshi Nakamoto, Gavin Andresen, and others
% - [Inside the World of the Bitcoin Carnivores][carnivores]: Why a small community of Bitcoin users is eating meat exclusively by Jordan Pearson
% - [Tales From the Crypt][tftc] hosted by Marty Bent
%
% <!-- Internal -->
% [gravity]: 
% [lesson18]: {{ 'bitcoin/lessons/ch3-18-move-slowly-and-dont-break-things' | absolute_url }}
%
% <!-- Further Reading -->
% [automated social contract]: https://medium.com/@hasufly/bitcoins-social-contract-1f8b05ee24a9
% [carnivores]: https://motherboard.vice.com/en_us/article/ne74nw/inside-the-world-of-the-bitcoin-carnivores
% [tftc]: https://tftc.io/tales-from-the-crypt/
% [bent]: https://tftc.io/martys-bent/
%
% <!-- Quotes -->
% [Ralph Merkle]: http://merkle.com/papers/DAOdemocracyDraft.pdf
% [Satoshi Nakamoto]: https://bitcointalk.org/index.php?topic=195.msg1611#msg1611
%
% <!-- Twitter People -->
% [Marty Bent]: https://twitter.com/martybent
%
% <!-- Wikipedia -->
% [alice]: https://en.wikipedia.org/wiki/Alice%27s_Adventures_in_Wonderland
% [carroll]: https://en.wikipedia.org/wiki/Lewis_Carroll
