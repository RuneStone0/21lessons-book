\chapter{Cypherpunks Skriver Kode}
\label{les:20}

\begin{chapquote}{Lewis Carroll, \textit{Alice i Eventyrland}}
\enquote{Jeg ser, du prøver at opfinde noget.}
\end{chapquote}

Som mange fantastiske idéer kom Bitcoin ikke ud af ingenting. Det blev
muliggjort ved at anvende og kombinere mange innovationer og opdagelser inden
for matematik, fysik, datalogi og andre områder. Mens
uundgåeligt en geni, ville Satoshi ikke have kunnet opfinde Bitcoin
uden de kæmper, hvis skuldre han stod på.

\begin{quotation}\begin{samepage}
\enquote{Den, der kun ønsker og håber, blander sig ikke aktivt i 
begivenhedernes gang og i formningen af sin egen skæbne.}
\begin{flushright} -- Ludwig von Mises\footnote{Ludwig von Mises, 
  \textit{Human Action} \cite{human-action}}
\end{flushright}\end{samepage}\end{quotation}
% > <cite>[Ludwig Von Mises]</cite>

En af disse kæmper er Eric Hughes, en af grundlæggerne af 
cypherpunk-bevægelsen og forfatter til \textit{A Cypherpunk's Manifesto}. 
Det er svært at forestille sig, at Satoshi ikke blev påvirket af dette 
manifest. Det taler om mange ting, som Bitcoin muliggør og udnytter, 
såsom direkte og private transaktioner, elektroniske penge og kontanter, 
anonyme systemer og forsvar af privatlivet med kryptografi og digitale 
signature.

\begin{quotation}\begin{samepage}
\enquote{Privatliv er nødvendigt for et åbent samfund i den elektroniske 
tidsalder. [...] Da vi ønsker privatliv, skal vi sikre, at hver part i en
transaktion kun har kendskab til det, der er direkte nødvendigt for den
transaktion. [...] Derfor kræver privatliv i et åbent samfund anonyme 
transaktionssystemer. Indtil nu har kontanter været det primære sådanne 
system. Et anonymt transaktionssystem er ikke et hemmeligt transaktionssystem.
[...]
Vi cypherpunks dedikerer os til at opbygge anonyme systemer. Vi forsvarer 
vores privatliv med kryptografi, med anonyme mail
videresendelsessystemer, med digitale signaturer og med elektroniske
penge.
Cypherpunks skriver kode.}
\begin{flushright} -- Eric Hughes\footnote{Eric Hughes, A Cypherpunk's 
  Manifesto \cite{cypherpunk-manifesto}}
\end{flushright}\end{samepage}\end{quotation}

Cypherpunks finder ikke trøst i håb og ønsker. De blander sig aktivt
i begivenhedernes gang og former deres egen skæbne.
Cypherpunks skriver kode.

Således, i sand cypherpunk-stil, satte Satoshi sig ned og begyndte at skrive
kode. Kode, der tog en abstrakt idé og beviste for verden, at det rent 
faktisk fungerede. Kode, der såede frøet til en ny økonomisk virkelighed.
Takket være denne kode kan alle verificere, at dette nye system rent faktisk
fungerer, og cirka hvert 10. minut beviser Bitcoin for verden, at det stadig 
er i live.

\begin{figure}[htbp]
  \centering
  \includegraphics[width=\textwidth]{assets/images/bitcoin-code-white.png}
  \caption{Kodeuddrag fra Bitcoin version 0.1}
  \label{fig:bitcoin-code-white}
\end{figure}

For at sikre, at hans innovation transcenderer fantasi og bliver virkelighed,
skrev Satoshi kode for at implementere sin idé, før han skrev whitepaperet. 
Han sørgede også for ikke at udsætte\footnote{\enquote{Vi bør ikke udsætte 
det for evigt, indtil enhver mulig funktion er færdig.} -- 
Satoshi Nakamoto~\cite{satoshi-delay}} enhver udgivelse for evigt.
Alt i alt, \enquote{der vil altid være én mere ting at gøre.}

\begin{quotation}\begin{samepage}
\enquote{Jeg måtte skrive al koden, før jeg kunne overbevise mig selv om, at jeg
kunne løse hvert problem, derefter skrev jeg papiret.}
\begin{flushright} -- Satoshi Nakamoto\footnote{Satoshi Nakamoto, Sv: 
  Bitcoin P2P e-cash paper \cite{satoshi-mail-code-first}}
\end{flushright}\end{samepage}\end{quotation}

I nutidens verden af endeløse løfter og tvivlsom udførelse var der desperat 
behov for en øvelse i dedikeret opbygning. Vær omhyggelig, overbevis
dig selv om, at du rent faktisk kan løse problemerne, og implementer
løsningerne. Vi bør alle stræbe efter at være lidt mere cypherpunk.

\paragraph{Bitcoin lærte mig, at cypherpunks skriver kode.}

% ---
%
% #### Down the Rabbit Hole
%
% - [Bitcoin version 0.1.0 announcement][version 0.1.0] by Satoshi Nakamoto
% - [Bitcoin P2P e-cash paper announcement][mail-announcement] by Satoshi Nakamoto
%
% [mail-announcement]: http://www.metzdowd.com/pipermail/cryptography/2008-October/014810.html
% [Ludwig Von Mises]: https://mises.org/library/human-action-0/html/pp/613
% [version 0.1.0]: https://bitcointalk.org/index.php?topic=68121.0
% [not to delay]: https://bitcointalk.org/index.php?topic=199.msg1670#msg1670
% [6]: http://www.metzdowd.com/pipermail/cryptography/2008-November/014832.html
%
% <!-- Wikipedia -->
% [alice]: https://en.wikipedia.org/wiki/Alice%27s_Adventures_in_Wonderland
% [carroll]: https://en.wikipedia.org/wiki/Lewis_Carroll
