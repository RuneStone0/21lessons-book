\chapter{Økonomisk Uvidenhed}
\label{les:8}

\begin{chapquote}{Lewis Carroll, \textit{Alice i Eventyrland}}
\enquote{Og hvilken uvidende lille pige, hun vil tro, jeg er, for at spørge! 
Nej, det holder aldrig: måske ser jeg det skrevet et eller andet sted.}
\end{chapquote}

En af de mest overraskende ting for mig var mængden af finans, økonomi og
psykologi, der kræves for at få greb om det, der ved første øjekast synes at 
være et rent \textit{teknisk} system --- et computer netværk. For at 
parafrasere en lille fyr med behårede fødder: \enquote{Det er en farlig 
affære, Frodo, at træde ind i Bitcoin. Du læser hvidbogen, og hvis du ikke 
passer på dine skridt, er der ingen måde at vide, hvor du kan blive skyllet 
hen.}

For at forstå et nyt monetært system skal du stifte bekendtskab med det gamle. 
Jeg begyndte meget hurtigt at indse, at mængden af økonomisk uddannelse, jeg 
nød i uddannelsessystemet, var essentielt \textit{nul}.

\paragraph{}
Som en femårig begyndte jeg at stille mig selv en masse spørgsmål: Hvordan 
fungerer bankvæsenet? Hvordan fungerer aktiemarkedet? Hvad er fiatpenge? Hvad 
er \textit{almindelige} penge? Hvorfor er der så meget gæld?
\footnote{\url{https://www.usdebtclock.org/}} Hvor mange penge bliver faktisk
printet, og hvem beslutter det?

\newpage

Efter en mild panik over omfanget af min uvidenhed fandt jeg trøst i 
erkendelsen af, at jeg var i godt selskab.

\begin{quotation}\begin{samepage}
\enquote{Er det ikke ironisk, at Bitcoin har lært mig mere om penge end alle 
de år, jeg har brugt på at arbejde for finansielle institutioner? \ldots 
inklusive starten af min karriere ved en centralbank.}
\begin{flushright} -- Aaron\footnote{Aaron (\texttt{@aarontaycc}, 
    \texttt{@fiatminimalist}), tweet fra 12. dec. 2018~\cite{aarontaycc-tweet}}
\end{flushright}\end{samepage}\end{quotation}

\begin{quotation}\begin{samepage}
\enquote{Jeg har lært mere om økonomi, finans, teknologi, kryptografi, 
menneskelig psykologi, politik, spilteori, lovgivning og mig selv de sidste 
tre måneder af krypto end de sidste tre og et halvt år på college.}
\begin{flushright} -- Dunny\footnote{Dunny (\texttt{@BitcoinDunny}), tweet fra 
    28. nov. 2017~\cite{bitcoindunny-tweet}}
\end{flushright}\end{samepage}\end{quotation}

Dette er blot to af mange bekendelser over hele Twitter.\footnote{Se 
\url{http://bit.ly/btc-learned} for flere bekendelser på Twitter.} Bitcoin, 
som blev udforsket i Lektion \ref{les:1}, er en levende ting. Mises hævdede, 
at økonomi også er en levende ting. Og som vi alle ved fra personlig erfaring, 
er levende ting intrinsisk svære at forstå.

\begin{quotation}\begin{samepage}
\enquote{Et videnskabeligt system er kun en station i en uendelig progressiv
søgen efter viden. Det er nødvendigvis påvirket af utilstrækkeligheden, der er 
iboende i enhver menneskelig bestræbelse. Men at erkende disse kendsgerninger
betyder ikke, at nutidens økonomi er tilbagestående. Det betyder blot, at
økonomi er en levende ting --- og at leve indebærer både ufuldkommenhed
og forandring.}
\begin{flushright} -- Ludwig von Mises\footnote{Ludwig von Mises, 
    \textit{Human Action}
\cite{human-action}}
\end{flushright}\end{samepage}\end{quotation}

\newpage

Vi læser alle om forskellige finanskriser i nyhederne, undrer os over, hvordan 
disse store redningsaktioner fungerer, og undrer os over, at ingen nogensinde 
synes at blive holdt ansvarlig for skader, der er i billionklassen. Jeg er 
stadig forundret, men i det mindste begynder jeg at få et glimt af, hvad der 
foregår i finansverdenen.

Nogle mennesker går endda så langt som at tilskrive den generelle uvidenhed om 
disse emner til systematisk, vilfuld uvidenhed. Mens historie, fysik, biologi, 
matematik og sprog alle er en del af vores uddannelse, udforskes verdenen af 
penge og finans overraskende kun overfladisk, hvis overhovedet. Jeg spekulerer 
på, om folk stadig ville være villige til at pådrage sig så meget gæld, som de 
gør i øjeblikket, hvis alle blev uddannet i personlig økonomi og penge- og 
gældsforhold. Så spekulerer jeg på, hvor mange lag aluminium der skal til for 
at lave en effektiv tinfoil-hat. Sandsynligvis tre.

\begin{quotation}\begin{samepage}
\enquote{Disse kriser, disse redningsaktioner, er ikke tilfældige. Og det er 
heller ikke en tilfældighed, at der ikke er nogen økonomisk uddannelse i skolen.
 [...] Det er forudoverlagt. Ligesom det var ulovligt at uddanne en slave før 
 borgerkrigen, må vi ikke lære om penge i skolen.}
\begin{flushright} -- Robert Kiyosaki\footnote{Robert Kiyosaki, 
    \textit{Why the Rich are Getting Richer}\cite{robert-kiyosaki}}
\end{flushright}\end{samepage}\end{quotation}

Som i Troldmanden fra Oz bliver vi bedt om ikke at lægge mærke til manden 
bag gardinet. I modsætning til i Troldmanden fra Oz har vi nu reel
trolddom\footnote{\url{http://bit.ly/btc-wizardry}}: et censurresistent, 
åbent, grænseløst netværk for værdioverførsel. Der er ingen forhæng, og 
magien er synlig for alle.\footnote{\url{https://github.com/bitcoin/bitcoin}}

\paragraph{Bitcoin lærte mig at se bag gardinet og konfrontere min økonomiske 
uvidenhed.}

% ---
%
% #### Down the Rabbit Hole
%
% - [Human Action][Ludwig von Mises] by Ludwig von Mises
% - [Why the Rich are Getting Richer][Robert Kiyosaki] by Robert Kiyosaki
%
% [real wizardry]: https://external-preview.redd.it/8d03MWWOf2HIyKrT8ThBGO4WFv-u25JaYqhbEO9b1Sk.jpg?width=683&auto=webp&s=dc5922d84717c6a94527bafc0189fd4ca02a24bb
% [visible to anyone]: https://github.com/bitcoin/bitcoin
%
% <!-- Wikipedia -->
% [alice]: https://en.wikipedia.org/wiki/Alice%27s_Adventures_in_Wonderland
% [carroll]: https://en.wikipedia.org/wiki/Lewis_Carroll
