\chapter{Værdi}
\label{les:10}

\begin{chapquote}{Lewis Carroll, \textit{Alice i Eventyrland}}
\enquote{Det var den hvide kanin, der langsomt travede tilbage og så sig 
nervøst omkring, som om den havde mistet noget\ldots}
\end{chapquote}

Værdi er på en måde paradoksal, og der er flere teorier\footnote{Se 
\textit{Teori om værdi (økonomi)} på Wikipedia \cite{wiki:theory-of-value}}, 
der forsøger at forklare, hvorfor vi værdsætter visse ting mere end andre.
Mennesker har været opmærksomme på dette paradoks i tusinder af år. Som Plato 
skrev i sin dialog med Euthydemus, værdsætter vi nogle ting, fordi de er 
sjældne, og ikke kun baseret på deres nødvendighed for vores overlevelse.

\begin{quotation}\begin{samepage}
\enquote{Og hvis du er klog, vil du give denne samme rådgivning til dine 
elever også --- at de aldrig skal tale med nogen undtagen dig og hinanden. 
For det er det sjældne, Euthydemus, der er dyrebart, mens vand er billigst, 
selvom det er det bedste, som Pindar sagde.}
\begin{flushright} -- Plato\footnote{Plato, 
    \textit{Euthydemus} \cite{euthydemus}}
\end{flushright}\end{samepage}\end{quotation}

Denne værdiparadoks\footnote{Se \textit{Paradoks om værdi} på Wikipedia
\cite{wiki:paradox-of-value}} viser noget interessant om os mennesker: vi synes 
at værdsætte ting på en subjektiv\footnote{Se \textit{Subjektiv værditeori} på 
Wikipedia \cite{wiki:subjective-theory-of-value}} basis, men gør det med visse 
ikke-arbitrære kriterier. Noget kan være \textit{dyrebart} for os af 
forskellige årsager, men ting, vi værdsætter, deler visse karakteristika. Hvis 
vi kan kopiere noget meget let, eller hvis det er naturligt rigeligt, 
værdsætter vi det ikke.

Det ser ud til, at vi værdsætter noget, fordi det er sjældent (guld, 
diamanter, tid), svært eller arbejdskrævende at producere, ikke kan 
erstattes (et gammelt billede af en elsket person), er nyttigt på en måde, der 
gør det muligt for os at gøre ting, som vi ellers ikke kunne, eller en 
kombination af disse, såsom store kunstværker.

Bitcoin er alt dette: den er ekstremt sjælden (21 millioner), bliver stadig 
sværere at producere (halvering af belønningen), kan ikke erstattes (en mistet 
privat nøgle er tabt for evigt) og gør det muligt for os at gøre nogle meget 
nyttige ting. Det er sandsynligvis det bedste redskab til værdioverførsel over 
grænser, næsten immun over for censur og beslaglæggelse i processen, og det 
er desuden en selvstændig værdilager, der giver enkeltpersoner mulighed for at 
opbevare deres formue uafhængigt af banker og regeringer, bare for at nævne to.

\paragraph{Bitcoin lærte mig, at værdi er subjektiv, men ikke arbitrær.}

% ---
%
% #### Down the Rabbit Hole
%
% - [Euthydemus] by Plato
% - [Theory of Value][multiple theories], [Paradox of Value][paradox of value], [Subjective Theory of Value][subjective] on Wikipedia
%
% [Euthydemus]: http://www.perseus.tufts.edu/hopper/text?doc=Perseus:text:1999.01.0178:text=Euthyd.
% [Plato]: http://www.perseus.tufts.edu/hopper/text?doc=plat.+euthyd.+304b
%
% <!-- Wikipedia -->
% [multiple theories]: https://en.wikipedia.org/wiki/Theory_of_value_%28economics%29
% [paradox of value]: https://en.wikipedia.org/wiki/Paradox_of_value
% [subjective]: https://en.wikipedia.org/wiki/Subjective_theory_of_value
% [alice]: https://en.wikipedia.org/wiki/Alice%27s_Adventures_in_Wonderland
% [carroll]: https://en.wikipedia.org/wiki/Lewis_Carroll
