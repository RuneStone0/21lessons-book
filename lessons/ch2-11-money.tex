\chapter{Penge}
\label{les:11}

\begin{chapquote}{Vismanden}
\enquote{I min ungdom, \ldots \
Holdt jeg alle mine lemmer meget smidige, \
Ved brugen af denne salve, \
fem shilling pr. boks -- \
Lad mig sælge dig et par.}
\end{chapquote}

Hvad er penge? Vi bruger det hver dag, og alligevel er dette spørgsmål overraskende svært at besvare. Vi er afhængige af det på store og små måder, og hvis vi har for lidt af det, bliver vores liv meget svært. Alligevel tænker vi sjældent på det, der angiveligt får verden til at dreje rundt. Bitcoin tvang mig til at besvare dette spørgsmål igen og igen: Hvad er pokker penge?

I vores \enquote{moderne} verden tænker de fleste sandsynligvis på stykker papir, når de taler om penge, selvom det meste af vores penge blot er et tal på en bankkonto. Vi bruger allerede nuller og ettaller som vores penge, så hvordan adskiller Bitcoin sig? Bitcoin er anderledes, fordi det grundlæggende set er en meget anderledes \textit{type} penge end de penge, vi bruger i øjeblikket. For at forstå dette, bliver vi nødt til at se nærmere på, hvad penge er, hvordan det opstod, og hvorfor guld og sølv blev brugt i det meste af handelshistorien.

\paragraph{}
Skaller, guld, sølv, papir, bitcoin. På et tidspunkt er \textbf{penge det, som folk bruger som penge}, uanset dets form og formål, eller mangel på samme.

Penge, som en opfindelse, er genial. En verden uden penge er vanvittigt kompliceret: Hvor mange fisk vil købe mig nye sko? Hvor mange køer vil købe mig et hus? Hvad nu hvis jeg ikke har brug for noget lige nu, men jeg skal af med mine snart rådne æbler? Du behøver ikke meget fantasi for at indse, at en bytteøkonomi er vanvittigt ineffektiv.

Det fantastiske ved penge er, at det kan byttes til \textit{hvad som helst andet} --- det er en temmelig fantastisk opfindelse! Som Nick Szabo\footnote{\url{http://unenumerated.blogspot.com/}} brilliant opsummerer i \textit{Shelling Out: The Origins of Money} \cite{shelling-out}, har mennesker brugt alle mulige ting som penge: perler lavet af sjældne materialer som elfenben, skaller eller specielle knogler, forskellige former for smykker og senere sjældne metaller som sølv og guld.

\begin{quotation}\begin{samepage}
\enquote{Set på denne måde er det mere typisk for et ædelmetal. I stedet for at ændre udbuddet for at holde værdien den samme, er udbuddet forudbestemt, og værdien ændres.}
\begin{flushright} -- Satoshi Nakamoto\footnote{Satoshi Nakamoto, i et svar til Sepp
Hasslberger \cite{satoshi-precious-metal}}
\end{flushright}\end{samepage}\end{quotation}

At være de dovne skabninger, vi er, tænker vi ikke for meget over ting, der bare fungerer. Penge fungerer for det meste fint for de fleste af os. Ligesom med vores biler eller vores computere bliver de fleste af os kun nødt til at tænke på tingenes indre virkemåde, hvis de går i stykker. Mennesker, der så deres livsbesparelser forsvinde på grund af hyperinflation, kender værdien af hård valuta, ligesom mennesker, der så deres venner og familie forsvinde på grund af grusomhederne begået af Nazi-Tyskland eller Sovjetunionen, kender værdien af privatliv.

Det interessante ved penge er, at det er altomfattende. Penge er halvdelen af enhver transaktion, hvilket giver dem, der er ansvarlige for at skabe penge, enorm magt.

\begin{quotation}\begin{samepage}
\enquote{Da penge udgør den ene halvdel af enhver kommerciel transaktion, og hele civilisationer bogstaveligt talt stiger og falder baseret på kvaliteten af deres penge, taler vi om en imponerende magt, en der flyver under dække af natten. Det er magten til at væve illusioner, der ser ægte ud, så længe de varer. Det er selve kernen af ​​Fed's magt.}
\begin{flushright} -- Ron Paul\footnote{Ron Paul, \textit{End the Fed} \cite{end-the-fed}}
\end{flushright}\end{samepage}\end{quotation}

Bitcoin fjerner fredeligt denne magt, da det gør op med pengeoprettelse, og det gør det uden brug af magt.

Penge gennemgik adskillige iterationer. De fleste iterationer var gode. De forbedrede vores penge på den ene eller anden måde. For nylig blev indre funktioner af vores penge imidlertid korrumperet. I dag skabes næsten alle vores penge simpelthen \textit{ud af ingenting} af magthaverne. For at forstå, hvordan dette kom til at være, måtte jeg lære om historien og efterfølgende nedgangen af ​​penge.

Om det vil kræve en række katastrofer eller blot en monumentalt uddannelsesmæssig indsats at rette denne korruption, forbliver at se. Jeg beder til lydighedens guder, at det vil være det sidste.

\paragraph{Bitcoin lærte mig, hvad penge er.}\end{samepage}\end{quotation}

% ---
%
% #### Down the Rabbit Hole
%
% - [End the Fed][Ron Paul] by Ron Paul
% - [Money, blockchains, and social scalability][social-scalability] by Nick Szabo
%
% [social-scalability]: https://unenumerated.blogspot.co.at/2017/02/money-blockchains-and-social-scalability.html
%
