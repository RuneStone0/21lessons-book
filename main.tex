% !TEX root = main.tex
% \documentclass{tufte-book}%[a4paper,twoside]
% See https://github.com/Tufte-LaTeX/tufte-latex/blob/master/sample-book.tex for details

% --- AMAZON BEGIN ---
% WITHOUT BLEED
% US Trade => 6x9
\documentclass[paper=6in:9in,pagesize=pdftex,
               headinclude=on,footinclude=on,12pt]{scrbook}
%
% Paper width
% W = 6in
% Paper height
% H = 9in
% Paper gutter
% BCOR = 0.5in
% Margin (0.5in imposed on lulu, recommended on createspace)
% m = 0.5in
% Text height
% h = H - 2m = 8in
% Text width
% w = W - 2m - BCOR = 4.5in
\areaset[0.50in]{4.5in}{8in}
% --- AMAZON END ---

% Copyright with title BEGIN
\usepackage{fancyhdr}
\def\secondpage{\clearpage\null\vfill
\pagestyle{empty}
\begin{minipage}[b]{0.9\textwidth}
\normalsize 21 Lektioner \newline
\footnotesize Hvad jeg har lært af at falde ned i Bitcoin-kaninhullet \par

2. Udgave. Version 0.3.12-e, git commit \texttt{3be46a1}.

\footnotesize\raggedright
\setlength{\parskip}{0.5\baselineskip}
Copyright \copyright 2018--\the\year\ Gigi / \href{https://twitter.com/dergigi}{@dergigi} / \href{https://dergigi.com}{dergigi.com} \par

Oversættelse of Udgivet af Rune Kristensen / \href{https://twitter.com/52544B}{@52544B}

\includegraphics[width=2cm]{assets/images/cc-by-sa.pdf}

Denne bog og dens online version distribueres under betingelserne i
Creative Commons Attribution-ShareAlike 4.0-licens. En referencekopi af denne
licens kan findes på den officielle creative commons
side.\footnote{\url{https://creativecommons.org/licenses/by-sa/4.0}}

\end{minipage}
\vspace*{2\baselineskip}
\cleardoublepage
\rfoot{\thepage}}

\makeatletter
\g@addto@macro{\maketitle}{\secondpage}
\makeatother
% Copyright with title END

% Use serif font for chapters and parts
\setkomafont{disposition}{\bfseries}
\KOMAoptions{headings=small}

% Packages
\usepackage{setspace}
\usepackage{booktabs}
\usepackage{graphicx}
\setkeys{Gin}{width=\linewidth,totalheight=\textheight,keepaspectratio}
\graphicspath{{graphics/}}

%%
% For Quotes
\usepackage{csquotes}
\renewcommand\mkbegdispquote[2]{\makebox[0pt][r]{\textquotedblleft\,}}
\renewcommand\mkenddispquote[2]{\,\textquotedblright#2}

%%
% Just some sample text
\usepackage{lipsum}

%%
% For nicely typeset tabular material
\usepackage{booktabs}

%%
% Bibliography stuff: Biber, BibTex, BibLatex
%\usepackage[autostyle]{csquotes}
% \usepackage[
    % backend=biber,
    % style=authoryear-icomp,
    % sortlocale=de_DE,
    % natbib=true,
    % url=false,
    % doi=true,
    % eprint=false
% ]{biblatex}
\usepackage[backend=biber]{biblatex}
\usepackage{url}
\addbibresource{main.bib}
% \usepackage{natbib}
% \bibliographystyle{plain}

%%
% Hyperlinks
\usepackage[hidelinks]{hyperref}

%%
% For graphics / images
\usepackage{caption}
\usepackage{graphicx}
\setkeys{Gin}{width=\linewidth,totalheight=\textheight,keepaspectratio}
\graphicspath{{graphics/}}

% The fancyvrb package lets us customize the formatting of verbatim
% environments.  We use a slightly smaller font.
\usepackage{fancyvrb}
\fvset{fontsize=\normalsize}

%%
% Prints argument within hanging parentheses (i.e., parentheses that take
% up no horizontal space).  Useful in tabular environments.
\newcommand{\hangp}[1]{\makebox[0pt][r]{(}#1\makebox[0pt][l]{)}}

%%
% Prints an asterisk that takes up no horizontal space.
% Useful in tabular environments.
\newcommand{\hangstar}{\makebox[0pt][l]{*}}

%%
% Prints a trailing space in a smart way.
\usepackage{xspace}

% Prints the month name (e.g., January) and the year (e.g., 2008)
\newcommand{\monthyear}{%
  \ifcase\month\or January\or February\or March\or April\or May\or June\or
  July\or August\or September\or October\or November\or
  December\fi\space\number\year
}


% Prints an epigraph and speaker in sans serif, all-caps type.
\newcommand{\openepigraph}[2]{%
  %\sffamily\fontsize{14}{16}\selectfont
  \begin{fullwidth}
  \sffamily\large
  \begin{doublespace}
  \noindent\allcaps{#1}\\% epigraph
  \noindent\allcaps{#2}% author
  \end{doublespace}
  \end{fullwidth}
}

% Inserts a blank page
\newcommand{\blankpage}{\newpage\hbox{}\thispagestyle{empty}\newpage}

\usepackage{units}

% Typesets the font size, leading, and measure in the form of 10/12x26 pc.
\newcommand{\measure}[3]{#1/#2$\times$\unit[#3]{pc}}

% Macros for typesetting the documentation
\newcommand{\hlred}[1]{\textcolor{Maroon}{#1}}% prints in red
\newcommand{\hangleft}[1]{\makebox[0pt][r]{#1}}
\newcommand{\hairsp}{\hspace{1pt}}% hair space
\newcommand{\hquad}{\hskip0.5em\relax}% half quad space
\newcommand{\TODO}{\textcolor{red}{\bf TODO!}\xspace}
\newcommand{\na}{\quad--}% used in tables for N/A cells
\providecommand{\XeLaTeX}{X\lower.5ex\hbox{\kern-0.15em\reflectbox{E}}\kern-0.1em\LaTeX}
\newcommand{\tXeLaTeX}{\XeLaTeX\index{XeLaTeX@\protect\XeLaTeX}}
% \index{\texttt{\textbackslash xyz}@\hangleft{\texttt{\textbackslash}}\texttt{xyz}}
\newcommand{\tuftebs}{\symbol{'134}}% a backslash in tt type in OT1/T1
\newcommand{\doccmdnoindex}[2][]{\texttt{\tuftebs#2}}% command name -- adds backslash automatically (and doesn't add cmd to the index)
\newcommand{\doccmddef}[2][]{%
  \hlred{\texttt{\tuftebs#2}}\label{cmd:#2}%
  \ifthenelse{\isempty{#1}}%
    {% add the command to the index
      \index{#2 command@\protect\hangleft{\texttt{\tuftebs}}\texttt{#2}}% command name
    }%
    {% add the command and package to the index
      \index{#2 command@\protect\hangleft{\texttt{\tuftebs}}\texttt{#2} (\texttt{#1} package)}% command name
      \index{#1 package@\texttt{#1} package}\index{packages!#1@\texttt{#1}}% package name
    }%
}% command name -- adds backslash automatically
\newcommand{\doccmd}[2][]{%
  \texttt{\tuftebs#2}%
  \ifthenelse{\isempty{#1}}%
    {% add the command to the index
      \index{#2 command@\protect\hangleft{\texttt{\tuftebs}}\texttt{#2}}% command name
    }%
    {% add the command and package to the index
      \index{#2 command@\protect\hangleft{\texttt{\tuftebs}}\texttt{#2} (\texttt{#1} package)}% command name
      \index{#1 package@\texttt{#1} package}\index{packages!#1@\texttt{#1}}% package name
    }%
}% command name -- adds backslash automatically
\newcommand{\docopt}[1]{\ensuremath{\langle}\textrm{\textit{#1}}\ensuremath{\rangle}}% optional command argument
\newcommand{\docarg}[1]{\textrm{\textit{#1}}}% (required) command argument
\newenvironment{docspec}{\begin{quotation}\begin{samepage}\ttfamily\parskip0pt\parindent0pt\ignorespaces}{\end{flushright}\end{samepage}\end{quotation}}% command specification environment
\newcommand{\docenv}[1]{\texttt{#1}\index{#1 environment@\texttt{#1} environment}\index{environments!#1@\texttt{#1}}}% environment name
\newcommand{\docenvdef}[1]{\hlred{\texttt{#1}}\label{env:#1}\index{#1 environment@\texttt{#1} environment}\index{environments!#1@\texttt{#1}}}% environment name
\newcommand{\docpkg}[1]{\texttt{#1}\index{#1 package@\texttt{#1} package}\index{packages!#1@\texttt{#1}}}% package name
\newcommand{\doccls}[1]{\texttt{#1}}% document class name
\newcommand{\docclsopt}[1]{\texttt{#1}\index{#1 class option@\texttt{#1} class option}\index{class options!#1@\texttt{#1}}}% document class option name
\newcommand{\docclsoptdef}[1]{\hlred{\texttt{#1}}\label{clsopt:#1}\index{#1 class option@\texttt{#1} class option}\index{class options!#1@\texttt{#1}}}% document class option name defined
\newcommand{\docmsg}[2]{\bigskip\begin{fullwidth}\noindent\ttfamily#1\end{fullwidth}\medskip\par\noindent#2}
\newcommand{\docfilehook}[2]{\texttt{#1}\index{file hooks!#2}\index{#1@\texttt{#1}}}
\newcommand{\doccounter}[1]{\texttt{#1}\index{#1 counter@\texttt{#1} counter}}

% Generates the index
\usepackage{makeidx}
\makeindex

%%
% Chapter/Lesson Quotes
\makeatletter
\renewcommand{\@chapapp}{}% Not necessary...
\newenvironment{chapquote}[2][4em]
  {\setlength{\@tempdima}{#1}%
   \def\chapquote@author{#2}%
   \parshape 1 \@tempdima \dimexpr\textwidth-2\@tempdima\relax%
   \itshape}
  {\par\normalfont\hfill--\ \chapquote@author\hspace*{\@tempdima}\par\bigskip}
\makeatother

%%%%%%%%%%%%%%%%%%%%%%%%%%%%%%%%%%%%%%%%%%%%%%%%%%%%%%%%%%%%%%%%%%%%%%%%%%%%%%%%
%                             eBook / Kindle / MOBi
%%%%%%%%%%%%%%%%%%%%%%%%%%%%%%%%%%%%%%%%%%%%%%%%%%%%%%%%%%%%%%%%%%%%%%%%%%%%%%%%
% See https://www.lode.de/blog/how-to-create-a-kindle-ebook-with-latex/

\usepackage{tex4ebook}

% --- Rewrite commands ---
\ifxetex

\else
    \usepackage{endnotes}
    \let\footcite\citep
    \ifx\HCode\undefined
        \def\myrule{\hrule}
        \newcommand{\emdash}[1][]{\hspace{0pt}---\hspace{0pt}}%
    \else
        \def\myrule{\HCode{<hr/>}}
            \def\semicolon{\detokenize{;}}
            \def\emdash{\HCode{&\#8212;}}%
            % \renewcommand\newpage[1][]{\HCode{<mbp:pagebreak />}}
    \fi
    \let\footnote=\endnote
\fi


%%%%%%%%%%%%%%%%%%%%%%%%%%%%%%%%%%%%%%%%%%%%%%%%%%%%%%%%%%%%%%%%%%%%%%%%%%%%%%%%
%                                   DOCUMENT
%%%%%%%%%%%%%%%%%%%%%%%%%%%%%%%%%%%%%%%%%%%%%%%%%%%%%%%%%%%%%%%%%%%%%%%%%%%%%%%%

\begin{document}

\coverimage{assets/images/ebook-cover.png}

\frontmatter

\title{21 Lektioner}
\subtitle{Hvad jeg har lært af at falde ned i Bitcoin-kaninhullet}
\author{Gigi}
\date{}

\maketitle

\cleardoublepage


\newpage \vspace*{8cm}
% Sets a PDF bookmark for the dedication
\pdfbookmark{Dedication}{dedication}
\thispagestyle{empty}
\begin{center}
  \Large \emph{
    Dedikeret til min kone, mit barn og alle denne verdens børn. Må bitcoin 
    tjene dig og giver en vision for en fremtid, der er værd at kæmpe for.
  }
\end{center}

\chapter*{Forord}
\pdfbookmark{Forord}{forord}

Nogle kalder det en religiøs oplevelse. Andre kalder det Bitcoin.

Jeg mødte først Gigi i et af mine åndelige hjem - Riga, Letland - hjemmet for
\textit{The Baltic Honeybadger} konferencen, hvor de mest dedikerede af
Bitcoin-troende foretager en årlig pilgrimsrejse. Efter en dyb frokostsamtale
var båndet, Gigi og jeg skabte, lige så fast som en Bitcoin-transaktion, der
blev behandlet, da vi for få timer siden gav hinanden hånden.

Mit andet åndelige hjem, Christ Church, Oxford, hvor jeg havde privilegiet at
studere til min MBA, var stedet, hvor jeg oplevede mit \enquote{Rabbit Hole}-
øjeblik. Ligesom Gigi transcenderede jeg de økonomiske, tekniske og sociale 
sfærer og blev åndeligt indhyllet af Bitcoin. Efter at have \enquote{købt højt} 
under boblen i november 2013 var der flere ekstremt hårdt lærte lektioner i den 
nådesløst knusende 3-årige lange nedtur i aktiemarkedet. 
Disse 21 lektioner ville virkelig have tjent mig godt på det tidspunkt. Mange af 
disse lektioner er simpelthen naturlige sandheder, der for den uindviede er 
sløret af en uigennemsigtig, skrøbelig film. Ved slutningen af denne bog vil 
facaden dog smuldre totalt.

På en krystalklar nat i Oxford i slutningen af august 2016, bare få uger efter
at kniven drejede i mit hjerte igen, da Bitfinex Exchange blev hacket, sad jeg
i stille overvejelse i Christ Church's Master's Garden. Tiderne var hårde, og
jeg var ved mit mentale og emotionelle bristepunkt efter hvad der syntes at være 
en livstids tortur; ikke på grund af økonomisk tab, men af de knusende åndelige
tab. Jeg følte mig isoleret i min verdensopfattelse. Hvis der bare var
ressourcer som denne på det tidspunkt for at se, at jeg ikke var alene. Master's
Garden er et meget specielt sted for mig og mange før mig gennem
århundrederne. Det var der, hvor en Charles Dodgson, en matematiklærer på Christ
Church, observerede en af sine unge elever, Alice Liddell, datteren af
dekanen på Christ Church. Dodgson, bedre kendt under sit kælenavn, Lewis 
Carroll, brugte Alice og haven som sin inspiration, og i magien af den hellige
jord stirrede jeg dybt ind i krypto-kløften, og den stirrede flammende tilbage,
tilintetgørende min arrogance og klaskede min selvrespekt lige i ansigtet. Jeg 
fandt endelig ro.

21 Lektioner tager dig med på en sand Bitcoin-rejse; ikke bare en rejse inden 
for filosofi, teknologi og økonomi, men også ind i sjælen.

Når du dykker dybere ned i filosofien, som kort er beskrevet i 7 af de 21 
lektioner, kan man næsten forstå oprindelsen af alle væsener med tilstrækkelig 
tid og eftertanke. Hans 7 lektioner om økonomi fanger, i enkle vendinger, 
hvordan vi er underlagt et finansielt greb, og hvordan det har formået at sætte 
skyklapper på vores sind, hjerter og sjæle. De 7 lektioner om teknologi udlægger 
skønheden og teknologisk perfektion af Bitcoin. Som en ikke-teknisk 
Bitcoin-entusiast giver lektionerne en gennemgang af den underliggende 
teknologiske karakter af Bitcoin og faktisk også teknologiens natur selv.

I denne flygtige oplevelse, vi kalder livet, lever, elsker og lærer vi. Men hvad 
er livet andet end en tidsstemplede række af begivenheder?

At erobre Bitcoin-bjerget er ikke let. Falske tinder er talrige, klipperne er 
rå, og revner og sprækker ligger overalt og lurer på at sluge dig. Efter at have 
læst denne bog vil du se, at Gigi er den ultimative Bitcoin guide, og jeg vil 
værdsætte ham for evigt.

\begin{flushright}
  Hass McCook \\
  29. november 2019
\end{flushright}

\newpage \vspace*{4cm}
\thispagestyle{empty}
\begin{quotation}
\begin{center}
  \large
  \enquote{Vil du fortælle mig, hvor jeg burde gå herfra?} \\~\\
  \enquote{Det afhænger meget af, hvor du gerne vil hen.} \\~\\
  \enquote{Det betyder ikke så meget for mig hvor --} \\~\\
  \enquote{Så betyder det ikke noget, hvilken vej du går.}
\end{center}
\begin{flushright} -- Lewis Carroll, \textit{Alice i Eventyrland}\end{flushright}
\end{quotation}
\tableofcontents

```latex
\def\bitcoinB{\leavevmode
  {\setbox0=\hbox{\textsf{B}}%
    \dimen0\ht0 \advance\dimen0 0.2ex
    \ooalign{\hfil \box0\hfil\cr
      \hfil\vrule height \dimen0 depth.2ex\hfil\cr
    }%
  }%
}

\chapter*{Om Denne Bog \\ (... og Om Forfatteren)}
\pdfbookmark{Om Denne Bog (... og Om Forfatteren)}{about}

Dette er en lidt usædvanlig bog. Men hey, Bitcoin er også en lidt usædvanlig
teknologi, så en usædvanlig bog om Bitcoin kan være passende. Jeg er ikke sikker
på, om jeg er en usædvanlig fyr (jeg kan lide at tænke på mig selv som en
\textit{almindelig} fyr), men historien om, hvordan denne bog kom til, og 
hvordan jeg blev forfatter, er værd at fortælle.

For det første er jeg ikke en forfatter. Jeg er ingeniør. Jeg studerede ikke
skrivning. Jeg studerede kode og programmering. For det andet havde jeg aldrig
til hensigt at skrive en bog, og slet ikke en bog om Bitcoin. For pokker, 
jeg er ikke engang en indfødt engelsktalende.\footnote{Årsagen til, at jeg
skriver disse ord, er, at min hjerne arbejder på mystiske måder. Når der opstår
noget teknisk, skifter den til engelsk tilstand.} Jeg er bare en fyr, der blev
bidt af Bitcoin. Hårdt.

Hvem er \textit{jeg} til at skrive en bog om Bitcoin? Det er et godt spørgsmål.
Den korte svar er let: Jeg er Gigi, og jeg er en bitcoiner.

Den lange svar er lidt mere nuanceret.

\paragraph{}
Min baggrund er inden for datalogi og softwareudvikling. I et
tidligere liv var jeg en del af en forskningsgruppe, der forsøgte at få
computere til at tænke og forstå, blandt andet. I endnu et tidligere liv skrev
jeg software til automatisk pasbehandling og relaterede ting, hvilket er endnu
mere skræmmende. Jeg ved lidt om computere og netværk, så jeg gætter på,
at jeg har en smule forspring for at forstå den tekniske side af Bitcoin.
Dog, som jeg forsøger at beskrive i denne bog, er den tekniske side af tingene
kun en lille del af udyret, som er Bitcoin. Og hver eneste lille dele er vigtig.

Denne bog kom til på grund af et simpelt spørgsmål: \textit{\enquote{Hvad har du
lært af Bitcoin?}} Jeg forsøgte at besvare dette spørgsmål i et enkelt tweet.
Derefter blev tweetet til en tweetstorm. Tweetstormen blev til en artikel.
Artiklen blev til tre artikler. Tre artikler blev til 21 Lektioner. Og 21
Lektioner blev til denne bog. Så jeg gætter på, at jeg bare er virkelig dårlig
til at kondensere mine tanker ned til et enkelt tweet.

\paragraph{}
\textit{\enquote{Hvorfor skrive denne bog?}}, spørger du måske. Igen er der et
kort og et langt svar. Det korte svar er, at jeg simpelthen måtte.
Jeg var (og er stadig) \textit{besat} af Bitcoin. Jeg finder det uendeligt
fascinerende. Jeg kan tilsyneladende ikke stoppe med at tænke på det og de
konsekvenser, det vil have for vores globale samfund. Det lange svar er, at jeg
tror, at Bitcoin den vigtige opfindelse i vores tid, og
flere mennesker skal forstå karakteren af denne opfindelse. Bitcoin er
stadig en af de mest misforståede fænomener i vores moderne verden, og det tog
mig år at indse alvoren af denne fremmedartede teknologi. At indse,
hvad Bitcoin er, og hvordan det vil transformere vores samfund, er en dyb
oplevelse. Jeg håber at så frøene, som måske fører til denne erkendelse
i dit hoved.

Selvom denne sektion hedder \enquote{\textit{Om Denne Bog (... og Om 
Forfatteren)}}, betyder det faktisk ikke noget, hvem jeg er eller 
hvad jeg gjorde. Jeg er bare en del af netværket, både 
bogstaveligt \textit{og} billedligt. Desuden bør du alligevel ikke stole på, 
hvad jeg siger. Som vi bitcoiners kan lide at sige: undersøg selv, og 
vigtigst af alt: stol ikke på noget, men verificer.

Jeg gjorde mit bedste for at lave mine lektier og give masser af kilder til dig,
kære læser, at dykke ned i. Ud over fodnoterne og citaterne i denne bog forsøger
jeg at holde en opdateret liste over ressourcer på
\href{https://21lessons.com/rabbithole}{21lessons.com/rabbithole} og på
\href{https://bitcoin-resources.com}{bitcoin-resources.com}, som også opregner
masser af andre udvalgte ressourcer, bøger og podcasts, der vil hjælpe dig med 
at forstå, hvad Bitcoin er.

\paragraph{}
Kort sagt er dette blot en bog om Bitcoin, skrevet af en bitcoiner.
Bitcoin har ikke brug for denne bog, og du har sandsynligvis ikke brug for denne
bog for at forstå Bitcoin. Jeg tror, at Bitcoin vil blive forstået af dig,
så snart \textit{du} er klar, og jeg tror også, at de første brøkdele af en
bitcoin vil finde dig, så snart du er klar til at modtage dem. I essensen
vil alle få \bitcoinB{}itcoin på præcis det rigtige tidspunkt. I mellemtiden er 
Bitcoin simpelthen til stede, og det er nok. \footnote{Beautyon, 
\textit{Bitcoin er. Og det er nok.}~\cite{bitcoin-is}}
\chapter*{Forord}

At falde ned i Bitcoin-kaninhullet er en mærkelig oplevelse. Ligesom mange andre
føler jeg, at jeg har lært mere de sidste par år, hvor jeg har studeret Bitcoin,
end jeg har gjort i to årtier med formel uddannelse.

De følgende lektioner er en samling af, hvad jeg har lært. Først udgivet
som en artikelserie med titlen \textit{"Hvad jeg har lært af Bitcoin,"} kan det
der følger, ses som en tredje udgave af den originale serie.

Ligesom Bitcoin er disse lektioner ikke en statisk ting. Jeg planlægger at 
arbejde på dem periodisk og udgive opdaterede versioner samt yderligere 
materiale i fremtiden.

I modsætning til Bitcoin behøver fremtidige versioner af dette projekt ikke at
være bagudkompatible. Nogle lektioner kan udvides, andre kan omarbejdes eller
erstattes.

Bitcoin er en uudtømmelig mentor, hvilket er grunden til, at jeg ikke hævder, at
disse lektioner er altomfattende eller komplette. De er en refleksion af min
personlige rejse ned i kaninhullet. Der er mange flere lektioner at lære, og
hver person vil lære noget forskelligt ved at træde ind i Bitcoin-verdenen.

Jeg håber, at du vil finde disse lektioner nyttige, og at processen med at lære
ikke vil være lige så besværlig og smertefuld hvis du skulle have gjort det 
alene.

% <!-- Internal -->
% [I]: 
%
% <!-- Twitter -->
% [dergigi]: https://twitter.com/dergigi
%
% <!-- Wikipedia -->
% [alice]: https://en.wikipedia.org/wiki/Alice%27s_Adventures_in_Wonderland
% [carroll]: https://en.wikipedia.org/wiki/Lewis_Carroll

%%
% Start the main matter (normal chapters)
\mainmatter

\part*{21 Lessons}
\newpage \vspace*{8cm}
\thispagestyle{empty}
\begin{quotation}
\begin{center}
  \large
  \enquote{Åh, du tåbelige Alice!} sagde hun igen, \enquote{hvordan kan du lære 
  lektioner her? Der er næsten ikke plads til dig, og slet ingen plads til 
  lektiebøger!}
\end{center}
\begin{flushright} -- Lewis Carroll, 
  \textit{Alice i Eventyrland}\end{flushright}
\end{quotation}

\chapter*{Introduktion}
\label{ch:introduktion}

\begin{chapquote}{Lewis Carroll, \textit{Alice i Eventyrland}}
\enquote{Men jeg vil ikke blandt gale mennesker,} bemærkede Alice. 
\enquote{Oh, du kan ikke hjælpe det,} sagde Katten: \enquote{vi er alle gale 
her. Jeg er gal. Du er gal.} \enquote{Hvordan ved du, at jeg er gal?} sagde 
Alice. \enquote{Du må være det,} sagde Katten, \enquote{ellers ville du ikke
være kommet her.}
\end{chapquote}

I oktober 2018 stillede Arjun Balaji det uskyldige spørgsmål,
\textit{Hvad har du lært af Bitcoin?} Efter at have forsøgt at besvare dette
spørgsmål i en kort tweet og fejlet miserabelt, indså jeg, at de ting
jeg har lært, er alt for talrige til at besvare hurtigt, hvis overhovedet.

De ting, jeg har lært, handler selvfølgelig om Bitcoin - eller i det mindste 
er de relateret til det. Dog, mens nogle af de indre arbejdsmetoder i Bitcoin 
bliver forklaret, er de følgende lektioner ikke en forklaring på, hvordan 
Bitcoin fungerer, eller hvad det er, de kan dog hjælpe med at udforske nogle 
af de ting, Bitcoin berører: filosofiske spørgsmål, økonomiske realiteter og 
teknologiske innovationer.

\begin{center}
  \includegraphics[width=7cm]{assets/images/the-tweet.png}
\end{center}

De \textit{21 Lektioner} er struktureret i bundter af syv, resulterende i tre
kapitler. Hvert kapitel ser på Bitcoin gennem en anden linse og udvinder
hvilke lektioner der kan læres ved at undersøge dette mærkelige netværk fra en 
anden vinkel.

\paragraph{\hyperref[ch:philosophy]{Kapitel 1}}{ udforsker de filosofiske
lærdomme fra Bitcoin. Samspillet mellem uforanderlighed og forandring, begrebet
sand knaphed, Bitcoins pletfri undfangelse, problemet med identitet,
modsigelsen af replikation og lokalitet, ytringsfrihedens magt og
vidensgrænserne.}

\paragraph{\hyperref[ch:economics]{Kapitel 2}}{ udforsker de økonomiske lærdomme
fra Bitcoin. Lektioner om økonomisk uvidenhed, inflation, værdi, penge og
pengenes historie, fractional reserve banking og hvordan Bitcoin genindfører
lydpenge på en snedig, indirekte måde.}

\paragraph{\hyperref[ch:technology]{Kapitel 3}}{ udforsker nogle af de 
lektioner, der er lært ved at undersøge teknologien i Bitcoin. Hvorfor der er 
styrke i antal, refleksioner over tillid, hvorfor at fortælle tiden kræver 
arbejde, hvordan at bevæge sig langsomt og ikke ødelægge ting er en funktion og 
ikke en fejl, hvad Bitcoin's skabelse kan fortælle os om privatliv, hvorfor 
cypherpunks skriver kode (og ikke love), og hvilke metaforer der kan være 
nyttige at udforske Bitcoins fremtid.}

~

Hver lektion indeholder adskillige citater og links i teksten. Hvis en idé er
værd at udforske nærmere, kan du følge links til relaterede værker i
fodnoterne eller i bibliografien.

Selvom noget forhåndsviden om Bitcoin er gavnligt, håber jeg, at disse
lektioner kan fordøjes af enhver nysgerrig læser. Mens nogle relaterer sig til 
hinanden, bør hver lektion være i stand til at stå på egen hånd og kan læses 
uafhængigt. Jeg gjorde mit bedste for at undgå teknisk jargon, selvom noget 
domænespecifikt ordforråd er uundgåeligt.

Jeg håber, at min skrivning tjener som inspiration for andre til at grave under
overfladen og undersøge nogle af de dybere spørgsmål, Bitcoin rejser. Min egen
inspiration kom fra en mangfoldighed af forfattere og indholdsproducenter, til 
hvem jeg er evigt taknemmelig.

Sidst men ikke mindst: mit mål med at skrive dette er ikke at overbevise dig om 
noget. Mit mål er at få dig til at tænke og vise dig, at der er meget mere ved 
Bitcoin end det, der møder øjet. Jeg kan ikke engang fortælle dig, hvad Bitcoin 
er, eller hvad Bitcoin vil lære dig. Det bliver du nødt til at finde ud af selv.

\begin{quotation}\begin{samepage}
\enquote{Efter dette er der ingen vej tilbage. Du tager den blå pille --- 
historien ender, du vågner op i din seng og tror på hvad som helst du vil tro. 
Du tager den røde pille\footnote{den \textit{orange} pille} --- du bliver i 
Wonderland, og jeg viser dig, hvor dybt kaninhullet går.}
\begin{flushright} -- Morpheus
\end{flushright}\end{samepage}\end{quotation}

\begin{center}
  \includegraphics[width=\textwidth]{assets/images/bitcoin-orange-pill.jpg}
  \captionof{figure}*{Husk: Alt, jeg tilbyder, er sandheden. Intet mere.}
  \label{fig:bitcoin-orange-pill}
\end{center}

%
% [Morpheus]: https://en.wikipedia.org/wiki/Red_pill_and_blue_pill#The_Matrix_(1999)
% [this question]: https://twitter.com/arjunblj/status/1050073234719293440
%
% <!-- Internal -->
% [chapter1]: {{ 'bitcoin/lessons/ch1-00-philosophy' | absolute_url }}
% [chapter2]: {{ 'bitcoin/lessons/ch2-00-economics' | absolute_url }}
% [chapter3]: {{ 'bitcoin/lessons/ch3-00-technology' | absolute_url }}
%
% <!-- Wikipedia -->
% [alice]: https://en.wikipedia.org/wiki/Alice%27s_Adventures_in_Wonderland
% [carroll]: https://en.wikipedia.org/wiki/Lewis_Carroll

\part{Filosofi}
\label{ch:filosofi}
\chapter*{Filosofi}

\begin{chapquote}{Lewis Carroll, \textit{Alice i Eventyrland}}
Musen betragtede hende ret nysgerrigt og syntes at blinke med det ene af dens 
små øjne, men den sagde intet.
\end{chapquote}

Ved første øjekast kunne man konkludere, at Bitcoin er langsom, spild af 
ressourcer, unødvendigt redundant og overdrevent paranoid. Men ved en dybere 
forståelse af Bitcoin vil man finde ud af, at tingene ikke er, som de ser ud 
ved første øjekast.

Bitcoin har en måde at tage dine antagelser og vende dem på hovedet. Efter et 
stykke tid, lige når du var ved at blive komfortabel igen, vil Bitcoin smadre 
igennem væggen som en tyr i en kinabutik og knuse dine antagelser endnu engang.

\begin{center}
  \includegraphics[width=\textwidth]{assets/images/blind-monks.jpg}
  \captionof{figure}{Blinde munke undersøger Bitcoin-elefant}
  \label{fig:blind-monks}
\end{center}

Bitcoin er et barn af mange discipliner. Ligesom blinde munke, der undersøger en 
elefant, vil enhver der nærmer sig denne nye teknologi, se den fra en ny 
vinkel. Og alle vil drage forskellige konklusioner om væsenets natur.

De følgende lektioner handler om nogle af mine antagelser, som Bitcoin knuste, 
og de konklusioner, jeg nåede frem til. Filosofiske spørgsmål om 
uforanderlighed, sjældenhed, lokalitet og identitet udforskes i de første fire 
lektioner. Hver del består af syv lektioner.

~

\begin{samepage}
Del~\ref{ch:filosofi} -- Filosofi:

\begin{enumerate}
  \item Uforanderlighed og forandring
  \item Sjældenhen af sjældenhed
  \item Reproduktion og lokalitet
  \item Problemet med identitet
  \item En uspoleret oprindelse
  \item Ytringsfrihedens magt
  \item Grænserne for viden
\end{enumerate}
\end{samepage}

Lektion \ref{les:5} udforsker, hvordan Bitcoin's oprindelseshistorie ikke kun er 
fascinerende, men absolut essentiel for et lederløst system. De sidste to 
lektioner i dette kapitel udforsker ytringsfrihedens magt og grænserne for vores 
individuelle viden, afspejlet af den overraskende dybde af Bitcoin-kaninhullet.

Jeg håber, at du vil finde verdenen af Bitcoin lige så uddannelsesmæssig, 
fascinerende og underholdende som jeg gjorde og stadig gør. Jeg inviterer dig 
til at følge den hvide kanin og udforske dybderne af dette kaninhul. Hold nu 
fast i dit lommeur, dyk ned og nyd faldet.

\chapter{Uforanderlighed og Forandring}
\label{les:1}

\begin{chapquote}{Alice}
\enquote{Jeg undrer mig over, om jeg er blevet forandret i løbet af natten. Lad 
mig tænke. Var jeg den samme, da jeg stod op i morges? Jeg tror næsten, jeg kan 
huske at have følt mig en smule anderledes. Men hvis jeg ikke er den samme, er 
det næste spørgsmål: 'Hvem i verden er jeg?' Ah, det er den store gåde!}
\end{chapquote}

Bitcoin er i sin natur svær at beskrive. Det er en \textit{ny ting}, og et hver
forsøg på at lave en sammenligning med tidligere begreber - om det så er ved at 
kalde det digitalt guld eller pengenes internet - vil uden tvivl mangle nuancer 
af det fulde billede. Uanset hvad din foretrukne analogi måtte være, er to 
aspekter af Bitcoin helt essentielle: decentralisering og uforanderlighed.

\paragraph{}
En måde at tænke på Bitcoin er som en automatiseret social kontrakt
\footnote{Hasu, Unpacking Bitcoin's Social Contract~\cite{social-contract}}. 
Softwaren er bare ét element i puslespillet, og håbet om at ændre Bitcoin ved 
at ændre softwaren er en øvelse i magtesløshed. Man skal overbevise resten af
netværket om at vedtage ændringerne, hvilket er mere en psykologisk indsats end 
en softwareteknisk.

\paragraph{}
Det følgende kan lyde absurd ved første øjekast, ligesom som så mange andre 
ting i denne verden, men jeg tror alligevel dybt på, at det er sandt: Du
kan ikke ændre Bitcoin, men Bitcoin vil ændre dig.

\begin{quotation}\begin{samepage}
\enquote{Bitcoin vil ændre os mere, end vi vil ændre det.}
\begin{flushright} -- Marty Bent\footnote{Tales From the Crypt~\cite{tftc21}}
\end{flushright}\end{samepage}\end{quotation}

Det tog mig lang tid at indse dybden af dette. Da Bitcoin bare er software, 
og alting er open source (offentligt tilgængeligt), kan du bare ændre ting 
efter behov, ikke sandt? Forkert. \textit{Meget} forkert. Ikke overraskende 
vidste Bitcoin's skaber det allerede.

\begin{quotation}\begin{samepage}
\enquote{Bitcoin's natur er sådan, at når version 0.1 blev frigivet, var kernen
designet fastlagt for resten af dens levetid.}
\begin{flushright} -- Satoshi Nakamoto\footnote{BitcoinTalk forumindlæg: 'Re:
Transactions and Scripts\ldots'~\cite{satoshi-set-in-stone}}
\end{flushright}\end{samepage}\end{quotation}

Mange har forsøgt at ændre Bitcoin's natur. Indtil videre er alle
fejlet. Mens der er et uendeligt hav af forgreninger og alternative mønter,
gør Bitcoin-netværket stadig sit arbejde, ligesom det gjorde, da den første
node gik online. Alternative mønter betyder ikke noget i det lange løb. 
Forgreningerne vil til sidst sulte ihjel. Det er Bitcoin, der betyder noget. 
Så længe vores grundlæggende forståelse af matematik og/eller fysik ikke ændres,
vil Bitcoin fortsætte uden bekymring.

\begin{quotation}\begin{samepage}
\enquote{Bitcoin er det første eksempel på en ny livsform. Den lever og ånder
på internettet. Den lever, fordi den kan betale folk for at holde den i live.
[\ldots] Den kan ikke ændres. Den kan ikke diskuteres. Den kan ikke manipuleres.
Den kan ikke ødelægges. Den kan ikke stoppes. [\ldots] Hvis en atomkrig 
ødelagde halvdelen af vores planet, ville den fortsætte med at leve, og være 
intakt og uændret.}
\begin{flushright} -- Ralph Merkle\footnote{DAOs, Democracy and
Governance,~\cite{merkle-dao}} 
\end{flushright}\end{samepage}\end{quotation}

Bitcoin-netværkets hjerteslag vil overleve alle vores.

~

At indse det ovenstående ændrede mig meget mere end de seneste blokke i 
Bitcoin's blockchain nogensinde vil gøre. Det ændrede min tidspræference, min
forståelse af økonomi, mine politiske synspunkter og meget mere. Faktisk ændrer
det endda menneskers kostvaner\footnote{Inside the World of the Bitcoin
Carnivores,~\cite{carnivores}}. Hvis alt dette lyder vanvittigt for dig, er du
i godt selskab. Alt dette er vanvittigt, og alligevel sker det.

~

\paragraph{Bitcoin lærte mig, at den ikke ændre sig. Jeg vil.}

% ---
%
% #### Through the Looking-Glass
%
% - [Bitcoin's Gravity: How idea-value feedback loops are pulling people in][gravity]
% - [Lesson 18: Move slowly and don't break things][lesson18]
%
% #### Down the Rabbit Hole
%
% - [Unpacking Bitcoin's Social Contract][automated social contract]: A framework for skeptics by Hasu
% - [DAOs, Democracy and Governance][Ralph Merkle] by Ralph C. Merkle
% - [Marty's Bent][bent]: A daily newsletter highlighting signal in Bitcoin by Marty Bent
% - [Technical Discussion on Bitcoin's Transactions and Scripts][Satoshi Nakamoto] by Satoshi Nakamoto, Gavin Andresen, and others
% - [Inside the World of the Bitcoin Carnivores][carnivores]: Why a small community of Bitcoin users is eating meat exclusively by Jordan Pearson
% - [Tales From the Crypt][tftc] hosted by Marty Bent
%
% <!-- Internal -->
% [gravity]: 
% [lesson18]: {{ 'bitcoin/lessons/ch3-18-move-slowly-and-dont-break-things' | absolute_url }}
%
% <!-- Further Reading -->
% [automated social contract]: https://medium.com/@hasufly/bitcoins-social-contract-1f8b05ee24a9
% [carnivores]: https://motherboard.vice.com/en_us/article/ne74nw/inside-the-world-of-the-bitcoin-carnivores
% [tftc]: https://tftc.io/tales-from-the-crypt/
% [bent]: https://tftc.io/martys-bent/
%
% <!-- Quotes -->
% [Ralph Merkle]: http://merkle.com/papers/DAOdemocracyDraft.pdf
% [Satoshi Nakamoto]: https://bitcointalk.org/index.php?topic=195.msg1611#msg1611
%
% <!-- Twitter People -->
% [Marty Bent]: https://twitter.com/martybent
%
% <!-- Wikipedia -->
% [alice]: https://en.wikipedia.org/wiki/Alice%27s_Adventures_in_Wonderland
% [carroll]: https://en.wikipedia.org/wiki/Lewis_Carroll


\chapter{Knapheden på Knaphed}
\label{les:2}

\begin{chapquote}{Alice}
\enquote{Det er helt nok - jeg håber, jeg ikke vokser mere\ldots}
\end{chapquote}

Generelt set ser det ud til, at teknologiens fremskridt gør ting mere 
tilgængelige. Flere og flere mennesker har mulighed for at nyde det, der 
tidligere har været luksusvarer. Snart vil vi alle leve som konger. De fleste 
af os gør det allerede. Som Peter Diamandis skrev i Abundance~\cite{abundance}: 
\enquote{Teknologi er en ressourcebefriende mekanisme. Den kan gøre det, der 
engang var knapt, nu overflod.}

Bitcoin, i sig selv en avanceret teknologi, bryder denne tendens og skaber en 
ny vare, der virkelig er knap. Nogle argumenterer endda for, at det er en af 
de sjældneste ting i universet. Udbuddet kan ikke oppustes, uanset hvor meget 
indsats man vælger at investere i at skabe mere.

\begin{quotation}\begin{samepage}
\enquote{Kun to ting er ægte knappe: tid og bitcoin.}
\begin{flushright} -- Saifedean Ammous\footnote{Præsentation om The Bitcoin 
    Standard~\cite{bitcoinstandard-pres}}
\end{flushright}\end{samepage}\end{quotation}

Paradoksalt nok gør den dette ved hjælp af en mekanisme af kopiering. 
Transaktioner bliver sendt ud, blokke bliver spredt, den distribuerede hovedbog 
er --- tja, du gættede det --- distribueret. Alle disse er bare fancy ord for 
kopiering. Faktisk kopierer Bitcoin endda sig selv til så mange computere som 
muligt ved at tilskynde enkeltpersoner til at køre fulde knudepunkter og mine 
nye blokke.

Alt dette duplikation arbejder vidunderligt sammen i en samordnet indsats for 
at producere knaphed.

\paragraph{I en tid med overflod lærte Bitcoin mig, hvad ægte knaphed er.}

% ---
%
% #### Through the Looking-Glass
%
% - [Lesson 14: Sound money][lesson14]
%
% #### Down the Rabbit Hole
%
% - [The Bitcoin Standard: The Decentralized Alternative to Central Banking][bitcoin-standard]
% - [Abundance: The Future Is Better Than You Think][Abundance] by Peter Diamandis
% - [Presentation on The Bitcoin Standard][bitcoin-standard-presentation] by Saifedean Ammous
% - [Modeling Bitcoin's Value with Scarcity][planb-scarcity] by PlanB
% - 🎧 [Misir Mahmudov on the Scarcity of Time & Bitcoin][tftc60] TFTC #60 hosted by Marty Bent
% - 🎧 [PlanB – Modelling Bitcoin's digital scarcity through stock-to-flow techniques][slp67] SLP #67 hosted by Stephan Livera
%
% <!-- Through the Looking-Glass -->
% [lesson14]: {{ 'bitcoin/lessons/ch2-14-sound-money' | absolute_url }}
%
% <!-- Down the Rabbit Hole -->
% [Abundance]: https://www.diamandis.com/abundance
% [bitcoin-standard]: http://amzn.to/2L95bJW
% [bitcoin-standard-presentation]: https://www.bayernlb.de/internet/media/de/ir/downloads_1/bayernlb_research/sonderpublikationen_1/bitcoin_munich_may_28.pdf
% [planb-scarcity]: https://medium.com/@100trillionUSD/modeling-bitcoins-value-with-scarcity-91fa0fc03e25
% [tftc60]: https://anchor.fm/tales-from-the-crypt/episodes/Tales-from-the-Crypt-60-Misir-Mahmudov-e3aibh
% [slp67]: https://stephanlivera.com/episode/67
%
% <!-- Wikipedia -->
% [alice]: https://en.wikipedia.org/wiki/Alice%27s_Adventures_in_Wonderland
% [carroll]: https://en.wikipedia.org/wiki/Lewis_Carroll

\chapter{Replikation og Lokalitet}
\label{les:3}

\begin{chapquote}{Lewis Carroll, \textit{Alice i Eventyrland}}
Derefter kom en vred stemme -- kaninen's -- \enquote{Pat, Pat! hvor er du?}
\end{chapquote}

Ignorer kvantemekanik. Lokalitet er ikke et problem i den fysiske verden. 
Spørgsmålet \textit{\enquote{Hvor er X?}} kan let besvares, uanset om 
X er en person eller en genstand. I den digitale verden er spørgsmålet om 
\textit{hvor} allerede et svært spørgsmål, men ikke umuligt at besvare. Hvor er 
dine e-mails, egentlig? Et dårligt svar ville være \enquote{skyen}, hvilket 
bare er en anden persons computer. Alligevel, hvis du ønskede at spore hver 
lagringsenhed, der har dine e-mails, kunne du teoretisk set finde dem.

Med bitcoin er spørgsmålet om \enquote{hvor} \textit{rigtig} svært. Hvor er 
dine bitcoins, præcist?

\begin{quotation}\begin{samepage}
\enquote{Jeg åbnede mine øjne, kiggede rundt og stillede det uundgåelige, det 
traditionelle, det beklageligt kliché postoperative spørgsmål: `Hvor er jeg?'}
\begin{flushright} -- Daniel Dennett\footnote{Daniel Dennett, 
    \textit{Hvor Er Jeg?}~\cite{where-am-i}}
\end{flushright}\end{samepage}\end{quotation}

Problemet er tofoldigt: For det første distribueres den distribuerede logbog 
ved fuld replikation, hvilket betyder, at logbogen er overalt. For det andet 
er der ingen bitcoins. Ikke kun fysisk, men også \textit{teknisk}.

Bitcoin holder styr på en række ubrugte transaktioner uden nogensinde at 
skulle henvise til en enhed, der repræsenterer en bitcoin. Existensen af en 
bitcoin bestemmes ved at se på antallet af ubrugte transaktioner og ved at
gennemgå alle tidligere overførseler.

\begin{quotation}\begin{samepage}
\enquote{Hvor er den på dette tidspunkt, er den under vejs? [...] For det første 
er der ingen bitcoins. De er der bare ikke. De eksisterer ikke. Der er kun 
bogføringsposter i en fælles logbog [...] De eksisterer ikke fysisk nogen steder. 
Logbogen eksisterer over alt i mange fysiske lokationer. Geografi giver 
ikke mening her --- det vil ikke hjælpe dig med at forstå dette.}
\begin{flushright} -- Peter Van Valkenburgh\footnote{Peter Van Valkenburgh på 
    \textit{What Bitcoin Did}-podcast, episode 49 \cite{wbd049}}
\end{flushright}\end{samepage}\end{quotation}

Så hvad ejer du egentlig, når du siger \textit{\enquote{Jeg har en bitcoin}}, 
hvis der ikke er nogen bitcoins? Kan du husje alle disse mærkelige ord, som du
blev tvunget til at skrive ned? Det viser sig, at disse 
magiske ord er det, du ejer: en magisk formular\footnote{Cryptography's Magic 
Dust: How digital information is changing our society \cite{gigi:magic-spell}}, 
som kan bruges til at tilføje nogle poster til den offentlige logbog --- 
nøglerne til at \enquote{flytte} nogle bitcoins. Derfor er dine private nøgler 
\textit{for alle formål} dine bitcoins. Hvis du tror, det er noget jeg har 
fundet på for sjov, så er du velkommen til at sende mig dine private nøgler.

\paragraph{Bitcoin lærte mig, at lokalitet er en vanskelig sag.}

% ---
%
% #### Through the Looking-Glass
%
% - [The Magic Dust of Cryptography: How digital information is changing our society][a magic spell]
%
% #### Down the Rabbit Hole
%
% - [Where Am I?][Daniel Dennett] by Daniel Dennett
% - 🎧 [Peter Van Valkenburg on Preserving the Freedom to Innovate with Public Blockchains][wbd049] WBD #49 hosted by Peter McCormack
%
% <!-- Through the Looking-Glass -->
% [a magic spell]: 
%
% <!-- Down the Rabbit Hole -->
% [Daniel Dennett]: https://www.lehigh.edu/~mhb0/Dennett-WhereAmI.pdf
% [1st Amendment]: https://en.wikipedia.org/wiki/First_Amendment_to_the_United_States_Constitution
% [wbd049]: https://www.whatbitcoindid.com/podcast/coin-centers-peter-van-valkenburg-on-preserving-the-freedom-to-innovate-with-public-blockchains
%
% <!-- Wikipedia -->
% [alice]: https://en.wikipedia.org/wiki/Alice%27s_Adventures_in_Wonderland
% [carroll]: https://en.wikipedia.org/wiki/Lewis_Carroll

\chapter{Identitetsproblemet}
\label{les:4}

\begin{chapquote}{Lewis Carroll, \textit{Alice i Eventyrland}}
\enquote{Hvem er du?} sagde larven.
\end{chapquote}

Nic Carter, som en hyldest til Thomas Nagels behandling af det samme spørgsmål 
vedrørende en flagermus, skrev et glimrende stykke, der drøfter følgende 
spørgsmål: Hvad er det at være en bitcoin? Han viser på en glimrende måde, at 
åbne, offentlige blockchains generelt, og Bitcoin i særdeleshed, lider af det 
samme dilemma som Theseus' skib\footnote{I identitetens metafysik er Theseus' 
skib et tankeeksperiment, der rejser spørgsmålet om, hvorvidt et objekt, der 
har haft alle sine komponenter udskiftet, stadig er fundamentalt det samme 
objekt.~\cite{wiki:theseus}}: hvilken Bitcoin er den virkelige Bitcoin?

\begin{quotation}\begin{samepage}
\enquote{Overvej, hvor lidt vedvarende Bitcoin's komponenter har. Hele 
kodebasen er blevet omarbejdet, ændret og udvidet, så den næsten ikke ligner 
sin oprindelige version. [...] Registeret over hvem der ejer hvad, selve 
logbogen, er stort set den eneste vedvarende egenskab ved netværket [...]
For at blive betragtet som virkelig lederløs må du opgive den nemme løsning med 
at have en enhed, der kan udpege én kæde som den legitime.}
\begin{flushright} -- Nic Carter\footnote{Nic Carter, \textit{Hvad er det at 
    være en bitcoin?} \cite{bitcoin-identity}}
\end{flushright}\end{samepage}\end{quotation}

Det ser ud til, at teknologiens fremskridt bliver ved med at tvinge os til at 
tage disse filosofiske spørgsmål alvorligt. Før eller senere vil selvkørende 
biler blive konfronteret med virkelige versioner af sporvognsproblemet, der 
tvinger dem til at træffe etiske beslutninger om, hvilke liv der betyder noget, 
og hvilke der ikke gør.

Kryptovalutaer, især siden den første kontroversielle hard fork, tvinger os til 
at tænke over og blive enige om identitetens metafysik. Interessant nok har de 
to største eksempler, vi har hidtil, ført til to forskellige svar. Den 1. 
august 2017 splittes Bitcoin i to lejre. Markedet besluttede, at den uændrede 
kæde er den originale Bitcoin. Et år tidligere, den 25. oktober 2016, splittes 
Ethereum i to lejre. Markedet besluttede, at den \textit{ændrede} kæde er den 
originale Ethereum.

Hvis det er ordentligt decentraliseret, vil spørgsmålene rejst af 
\textit{Theseus' skib} nødt til at blive besvaret i al evighed, så længe disse 
værdioverførselsnetværk eksisterer.

\paragraph{Bitcoin lærte mig, at decentralisering modsiger identitet.}

% ---
%
% #### Down the Rabbit Hole
%
% - [What Is It Like to be a Bat?][in regards to a bat] by Thomas Nagel
% - [What is it like to be a bitcoin?] by Nic Carter
% - [Ship of Theseus], [trolley problem] on Wikipedia
%
% [in regards to a bat]: https://en.wikipedia.org/wiki/What_Is_it_Like_to_Be_a_Bat%3F
% [What is it like to be a bitcoin?]: https://medium.com/s/story/what-is-it-like-to-be-a-bitcoin-56109f3e6753
% [Ship of Theseus]: https://en.wikipedia.org/wiki/Ship_of_Theseus
% [trolley problem]: https://en.wikipedia.org/wiki/Trolley_problem
%
% <!-- Wikipedia -->
% [alice]: https://en.wikipedia.org/wiki/Alice%27s_Adventures_in_Wonderland
% [carroll]: https://en.wikipedia.org/wiki/Lewis_Carroll

\chapter{En Pletfri Undfangelse}
\label{les:5}

\begin{chapquote}{Lewis Carroll, \textit{Alice i Eventyrland}}
\enquote{Der eres hoveder,} råbte soldaterne som svar\ldots
\end{chapquote}

Alle elsker en god oprindelseshistorie. Oprindelseshistorien for Bitcoin er fascinerende, og detaljerne er vigtigere, end man måske tror ved første øjekast. Hvem er Satoshi Nakamoto? Var det én person eller en gruppe mennesker? Var det en kvinde? En tidsrejsende alien eller avanceret AI? Ud over spekulative teorier vil vi sandsynligvis aldrig få det at vide. Og det er vigtigt.

Satoshi valgte at være anonym. Han såede frøet til Bitcoin. Han blev længe nok til at sikre, at netværket ikke døde i sin spæde barndom. Og så forsvandt han.

Hvad der måske ligner et mærkeligt anonymitetsstunt, er faktisk afgørende for et sandt decentraliseret system. Ingen centraliseret kontrol. Ingen centraliseret myndighed. Ingen opfinder. Ingen at retsforfølge, torturere, afpresse eller true. En pletfri undfangelse af teknologi.

\begin{quotation}\begin{samepage}
\enquote{En af de største ting, Satoshi gjorde, var at forsvinde.}
\begin{flushright} -- Jimmy Song\footnote{Jimmy Song, \textit{Hvorfor Bitcoin er anderledes} \cite{bitcoin-different}}
\end{flushright}\end{samepage}\end{quotation}

\newpage

Siden fødslen af Bitcoin er der blevet skabt tusindvis af andre kryptovalutaer. Ingen af disse kloner deler dens oprindelseshistorie. Hvis du vil overgå Bitcoin, bliver du nødt til at transcendere dens oprindelseshistorie. I en krig af ideer dikterer fortællinger overlevelsen.

\begin{quotation}\begin{samepage}
\enquote{Guld blev først lavet til smykker og brugt til byttehandel for over 7.000 år siden. Guldets fængslende glans førte til, at det blev betragtet som en gave fra guderne.}
\begin{flushright} Østrigske Mønt\footnote{Den østrigske Mønt, \textit{Guld: Det Ekstraordinære Metal} \cite{gold-gift-gods}}
\end{flushright}\end{samepage}\end{quotation}

Som guld i oldtiden kan Bitcoin betragtes som en gave fra guderne. I modsætning til guld er Bitcoins oprindelse alt for menneskelig. Og denne gang ved vi, hvem udviklings- og vedligeholdelsesguderne er: mennesker over hele verden, anonyme eller ej.

\paragraph{Bitcoin lærte mig, at fortællinger er vigtige.}

% ---
%
% #### Down the Rabbit Hole
%
% - [Why Bitcoin is different][Jimmy Song] by Jimmy Song
% - [Gold: The Extraordinary Metal] by the Austrian Mint
%
% <!-- Down the Rabbit Hole -->
% [Jimmy Song]: https://medium.com/@jimmysong/why-bitcoin-is-different-e17b813fd947
% [Gold: The Extraordinary Metal]: https://www.muenzeoesterreich.at/eng/discover/for-investors/gold-the-extraordinary-metal
%
% <!-- Wikipedia -->
% [alice]: https://en.wikipedia.org/wiki/Alice%27s_Adventures_in_Wonderland
% [carroll]: https://en.wikipedia.org/wiki/Lewis_Carroll


\chapter{Ytringsfrihedens Magt}
\label{les:6}

\begin{chapquote}{Lewis Carroll, \textit{Alice i Eventyrland}}
\enquote{Undskyld mig?} sagde musen, rynkende på panden, men meget høfligt, 
\enquote{talte du?}
\end{chapquote}

Bitcoin er en idé. En idé, som i sin nuværende form er manifestationen af en 
mekanisme, der udelukkende er drevet af tekst. Hver eneste aspekt af Bitcoin 
er tekst: Hvidbogen er tekst. Softwaren, der køres af dens knudepunkter, er 
tekst. logbogen er tekst. Transaktioner er tekst. Offentlige og private 
nøgler er tekst. Hver eneste aspekt af Bitcoin er tekst og dermed ækvivalent 
med ytringsfrihed.

\begin{quotation}\begin{samepage}
\enquote{Kongressen må ikke vedtage love, der vedrører en etablering af religion
eller forhindrer den frie udøvelse heraf; eller forkorter ytringsfriheden eller
trykkefriheden; eller borgerrettigheden for folk at samles fredeligt og
fremsætte anmodninger til regeringen om at rette op på uretfærdigheder.}
\begin{flushright} -- Første tillæg til den amerikanske forfatning
\end{flushright}\end{samepage}\end{quotation}

Selvom den endelige kamp i Crypto Wars\footnote{\textit{Crypto Wars} er en 
uformel betegnelse for de amerikanske og allierede regeringers forsøg på at 
underminere kryptering.\cite{eff-cryptowars}\cite{wiki:cryptowars}} ikke er 
blevet udkæmpet endnu, vil det være meget svært at kriminalisere en idé, ikke 
mindst en idé, der er baseret på udvekslingen af tekstbeskeder. Hver gang en 
regering forsøger at forbyde tekst eller tale, glider vi ned ad en sti af 
absurditet, der uundgåeligt fører til grusomheder som ulovlige tal\footnote{Et 
ulovligt tal er et tal, der repræsenterer information, som det er ulovligt at 
besidde, udtale, sprede eller på anden måde transmittere i nogle retslige 
jurisdiktioner.\cite{wiki:illegal-number}} og ulovlige primtal\footnote{Et 
ulovligt primtal er et primtal, der repræsenterer information, hvis besiddelse 
eller distribution er forbudt i nogle retslige jurisdiktioner. Et af de første 
ulovlige primtal blev fundet i 2001. Når det fortolkes på en bestemt måde, 
beskriver det et computerprogram, der omgår den digitale retsstyring, der 
anvendes på DVD'er. Distribution af et sådant program i USA er ulovligt under 
Digital Millennium Copyright Act. Et ulovligt primtal er en form for ulovligt 
tal.\cite{wiki:illegal-prime}}.

Så længe der er en del af verden, hvor ytringsfrihed er fri som i 
\textit{frihed}, er Bitcoin ustoppelig.

\begin{quotation}\begin{samepage}
\enquote{Der er ingen tidspunkt i nogen Bitcoin-transaktion, hvor Bitcoin
ophører med at være \textit{tekst}. Det er \textit{alt tekst}, hele tiden.
[...] Bitcoin er \textit{tekst}. Bitcoin er \textit{ytring}. Det kan ikke 
reguleres i et frit land som USA med garanterede ukrænkelige rettigheder og en 
første tillæg, der eksplicit udelukker offentliggørelseshandlingen fra 
regeringsindsyn.}
\begin{flushright} -- Beautyon\footnote{Beautyon, \textit{Hvorfor Amerika ikke 
    kan regulere
Bitcoin} \cite{america-regulate-bitcoin}}
\end{flushright}\end{samepage}\end{quotation}

\paragraph{Bitcoin lærte mig, at i et frit samfund er fri ytring og fri 
software ustoppelige.}

% ---
%
% #### Through the Looking-Glass
%
% - [The Magic Dust of Cryptography: How digital information is changing our society][a magic spell]
%
% #### Down the Rabbit Hole
%
% - [Why America can't regulate Bitcoin][Beautyon] by Beautyon
% - [First Amendment to the United States Constitution][1st Amendment], [Crypto Wars], [illegal numbers], [illegal primes] on Wikipedia
%
% <!-- Through the Looking-Glass -->
% [a magic spell]: 
%
% <!-- Down the Rabbit Hole -->
% [1st Amendment]: https://en.wikipedia.org/wiki/First_Amendment_to_the_United_States_Constitution
% [Crypto Wars]: https://en.wikipedia.org/wiki/Crypto_Wars
% [illegal numbers]: https://en.wikipedia.org/wiki/Illegal_number
% [illegal primes]: https://en.wikipedia.org/wiki/Illegal_prime
% [Beautyon]: https://hackernoon.com/why-america-cant-regulate-bitcoin-8c77cee8d794
%
% <!-- Wikipedia -->
% [alice]: https://en.wikipedia.org/wiki/Alice%27s_Adventures_in_Wonderland
% [carroll]: https://en.wikipedia.org/wiki/Lewis_Carroll

\chapter{Videns Grænser}
\label{les:7}

\begin{chapquote}{Lewis Carroll, \textit{Alice i Eventyrland}}
\enquote{Ned, ned, ned. Ville faldet aldrig få en ende?}
\end{chapquote}

At dykke ned i Bitcoin er en ydmygende oplevelse. Jeg troede, at jeg vidste 
ting. Jeg troede, at jeg var veluddannet. Jeg troede, at jeg kendte min 
datalogi, i det mindste. Jeg studerede det i årevis, så jeg må da vide alt om 
digitale signature, hashfunktioner, kryptering, operationel sikkerhed og 
netværk, ikke sandt?

\paragraph{}
Forkert.

\paragraph{}
At lære alle grundlæggende principper, der får Bitcoin til at fungere, er 
svært. At forstå dem alle dybt er grænsende til umuligt.

\begin{quotation}\begin{samepage}
\enquote{Ingen har fundet bunden af Bitcoin-kaninhullet.}
\begin{flushright} -- Jameson Lopp\footnote{Jameson Lopp, tweet fra 11.
    november 2018 \cite{lopp-tweet}}
\end{flushright}\end{samepage}\end{quotation}

\begin{center}
    \centering
    \includegraphics[width=7cm]{assets/images/rabbit-hole-bottomless.png}
    \captionof{figure}{Bitcoin-kaninhullet er bundløst.}
    \label{fig:rabbit-hole-bottomless}
\end{center}

Min liste over bøger, jeg skal læse, udvider sig langt hurtigere, end jeg 
nogensinde kan nå at læse dem. Listen over artikler og papirer, jeg skal 
læse, er stort set endeløs. Der er flere podcasts om alle disse emner, end 
jeg nogensinde kunne nå at lytte til. Det er virkelig ydmygende. Desuden 
udvikler Bitcoin sig, og det er næsten umuligt at holde sig opdateret med den 
accelererende innovationshastighed. Støvet fra den første lag er knap nok lagt 
sig, og folk har allerede bygget det andet lag og arbejder på det tredje.

\paragraph{Bitcoin lærte mig, at jeg ved meget lidt om næsten alt. Den lærte 
mig, at dette kaninhul er bundløst.}

% ---
%
% #### Down the Rabbit Hole
%
% - [Bitcoin Literature] by the Satoshi Nakamoto Institute
% - [Bitcoin Information & Resources][lopp-resources] by Jameson Lopp
% - [Educational Resources][bitcoin-only] by Bitcoin Only
%
% <!-- Twitter -->
% [Jameson Lopp]: https://twitter.com/lopp/status/1061415918616698881
%
% <!-- Down the Rabbit Hole -->
% [lopp-resources]: https://www.lopp.net/bitcoin-information.html
% [bitcoin-only]: https://bitcoin-only.com/#learning
% [Bitcoin Literature]: https://nakamotoinstitute.org/literature/
%
% <!-- Wikipedia -->
% [alice]: https://en.wikipedia.org/wiki/Alice%27s_Adventures_in_Wonderland
% [carroll]: https://en.wikipedia.org/wiki/Lewis_Carroll

\part{Økonomi}
\label{ch:økonomi}
\chapter*{Økonomi}

\begin{chapquote}{Lewis Carroll, \textit{Alice i Eventyrland}}
\enquote{Et stort rosetræ stod nær indgangen til haven: roserne på det var 
hvide, men der var tre gartnere ved det, travlt beskæftiget med at male dem 
røde. Dette tænkte Alice var en meget mærkelig ting...}
\end{chapquote}

Penge vokser ikke på træer. At tro det, er dumt, og vores forældre sørger for, 
at vi ved det ved at gentage denne sætning som en mantra. Vi opfordres til at 
bruge penge klogt, at ikke bruge dem tankeløst og spare dem i gode tider for at 
hjælpe os gennem de dårlige. Penge vokser trods alt ikke på træer.

Bitcoin lærte mig mere om penge, end jeg nogensinde troede, jeg ville have brug 
for at vide. Gennem det blev jeg tvunget til at udforske historien om penge, 
bankvæsent, forskellige skoler inden for økonomisk tænkning og mange andre ting. 
Jagten på at forstå Bitcoin førte mig ned ad en overflod af stier, nogle af dem 
forsøger jeg at udforske i dette kapitel.

I de første syv lektioner blev nogle af de filosofiske spørgsmål, Bitcoin 
berører, diskuteret. De næste syv lektioner vil se nærmere på penge og økonomi.

~

\begin{samepage}
Del~\ref{ch:økonomi} -- Økonomi:

\begin{enumerate}
  \setcounter{enumi}{7}
  \item Finansiel uvidenhed
  \item Inflation
  \item Værdi
  \item Penge
  \item Historien og faldet af penge
  \item Fractional reserve insanity
  \item Lyd penge
\end{enumerate}
\end{samepage}

Igen vil jeg kun kunne ridse overfladen. Bitcoin er ikke kun ambitiøs, men også 
bred og dyb i omfang, hvilket gør det umuligt at dække alle relevante emner i en 
enkelt lektion, artikel eller bog. Jeg tvivler endda på, om det er muligt 
overhovedet.

Bitcoin er en ny form for penge, hvilket gør læring om økonomi afgørende for at 
forstå det. At beskæftige sig med menneskelig handlinger og interaktionerne 
mellem økonomiske mæglere er sandsynligvis et af de største og mest uklare 
elementer i Bitcoin-puslespillet.

Igen, disse lektioner en udforskning af de forskellige ting, jeg har lært af 
Bitcoin. De er en personlig refleksion over min rejse ned i kaninhullet. Uden 
baggrund inden for økonomi er jeg helt sikkert uden for min komfortzone og 
særlig opmærksom på, at enhver forståelse, jeg måtte have, er ufuldstændig. Jeg 
vil gøre mit bedste for at skitsere, hvad jeg har lært, selvom det indebærer 
risikoen for at gøre mig selv til nar. Trods alt forsøger jeg stadig at besvare
spørgsmålet: \textit{\enquote{Hvad har du lært af Bitcoin?}}

Efter syv lektioner undersøgt gennem filosofiens linse, lad os bruge økonomiens 
linse til at se på syv mere. Undervisning i økonomi er alt, hvad jeg kan 
tilbyde denne gang. Endelig destination: \textit{rigtige penge}.

% [the question]: https://twitter.com/arjunblj/status/1050073234719293440

\chapter{Økonomisk Uvidenhed}
\label{les:8}

\begin{chapquote}{Lewis Carroll, \textit{Alice i Eventyrland}}
\enquote{Og hvilken uvidende lille pige, hun vil tro, jeg er, for at spørge! 
Nej, det holder aldrig: måske ser jeg det skrevet et eller andet sted.}
\end{chapquote}

En af de mest overraskende ting for mig var mængden af finans, økonomi og
psykologi, der kræves for at få greb om det, der ved første øjekast synes at 
være et rent \textit{teknisk} system --- et computer netværk. For at 
parafrasere en lille fyr med behårede fødder: \enquote{Det er en farlig 
affære, Frodo, at træde ind i Bitcoin. Du læser hvidbogen, og hvis du ikke 
passer på dine skridt, er der ingen måde at vide, hvor du kan blive skyllet 
hen.}

For at forstå et nyt monetært system skal du stifte bekendtskab med det gamle. 
Jeg begyndte meget hurtigt at indse, at mængden af økonomisk uddannelse, jeg 
nød i uddannelsessystemet, var essentielt \textit{nul}.

\paragraph{}
Som en femårig begyndte jeg at stille mig selv en masse spørgsmål: Hvordan 
fungerer bankvæsenet? Hvordan fungerer aktiemarkedet? Hvad er fiatpenge? Hvad 
er \textit{almindelige} penge? Hvorfor er der så meget gæld?
\footnote{\url{https://www.usdebtclock.org/}} Hvor mange penge bliver faktisk
printet, og hvem beslutter det?

\newpage

Efter en mild panik over omfanget af min uvidenhed fandt jeg trøst i 
erkendelsen af, at jeg var i godt selskab.

\begin{quotation}\begin{samepage}
\enquote{Er det ikke ironisk, at Bitcoin har lært mig mere om penge end alle 
de år, jeg har brugt på at arbejde for finansielle institutioner? \ldots 
inklusive starten af min karriere ved en centralbank.}
\begin{flushright} -- Aaron\footnote{Aaron (\texttt{@aarontaycc}, 
    \texttt{@fiatminimalist}), tweet fra 12. dec. 2018~\cite{aarontaycc-tweet}}
\end{flushright}\end{samepage}\end{quotation}

\begin{quotation}\begin{samepage}
\enquote{Jeg har lært mere om økonomi, finans, teknologi, kryptografi, 
menneskelig psykologi, politik, spilteori, lovgivning og mig selv de sidste 
tre måneder af krypto end de sidste tre og et halvt år på college.}
\begin{flushright} -- Dunny\footnote{Dunny (\texttt{@BitcoinDunny}), tweet fra 
    28. nov. 2017~\cite{bitcoindunny-tweet}}
\end{flushright}\end{samepage}\end{quotation}

Dette er blot to af mange bekendelser over hele Twitter.\footnote{Se 
\url{http://bit.ly/btc-learned} for flere bekendelser på Twitter.} Bitcoin, 
som blev udforsket i Lektion \ref{les:1}, er en levende ting. Mises hævdede, 
at økonomi også er en levende ting. Og som vi alle ved fra personlig erfaring, 
er levende ting intrinsisk svære at forstå.

\begin{quotation}\begin{samepage}
\enquote{Et videnskabeligt system er kun en station i en uendelig progressiv
søgen efter viden. Det er nødvendigvis påvirket af utilstrækkeligheden, der er 
iboende i enhver menneskelig bestræbelse. Men at erkende disse kendsgerninger
betyder ikke, at nutidens økonomi er tilbagestående. Det betyder blot, at
økonomi er en levende ting --- og at leve indebærer både ufuldkommenhed
og forandring.}
\begin{flushright} -- Ludwig von Mises\footnote{Ludwig von Mises, 
    \textit{Human Action}
\cite{human-action}}
\end{flushright}\end{samepage}\end{quotation}

\newpage

Vi læser alle om forskellige finanskriser i nyhederne, undrer os over, hvordan 
disse store redningsaktioner fungerer, og undrer os over, at ingen nogensinde 
synes at blive holdt ansvarlig for skader, der er i billionklassen. Jeg er 
stadig forundret, men i det mindste begynder jeg at få et glimt af, hvad der 
foregår i finansverdenen.

Nogle mennesker går endda så langt som at tilskrive den generelle uvidenhed om 
disse emner til systematisk, vilfuld uvidenhed. Mens historie, fysik, biologi, 
matematik og sprog alle er en del af vores uddannelse, udforskes verdenen af 
penge og finans overraskende kun overfladisk, hvis overhovedet. Jeg spekulerer 
på, om folk stadig ville være villige til at pådrage sig så meget gæld, som de 
gør i øjeblikket, hvis alle blev uddannet i personlig økonomi og penge- og 
gældsforhold. Så spekulerer jeg på, hvor mange lag aluminium der skal til for 
at lave en effektiv tinfoil-hat. Sandsynligvis tre.

\begin{quotation}\begin{samepage}
\enquote{Disse kriser, disse redningsaktioner, er ikke tilfældige. Og det er 
heller ikke en tilfældighed, at der ikke er nogen økonomisk uddannelse i skolen.
 [...] Det er forudoverlagt. Ligesom det var ulovligt at uddanne en slave før 
 borgerkrigen, må vi ikke lære om penge i skolen.}
\begin{flushright} -- Robert Kiyosaki\footnote{Robert Kiyosaki, 
    \textit{Why the Rich are Getting Richer}\cite{robert-kiyosaki}}
\end{flushright}\end{samepage}\end{quotation}

Som i Troldmanden fra Oz bliver vi bedt om ikke at lægge mærke til manden 
bag gardinet. I modsætning til i Troldmanden fra Oz har vi nu reel
trolddom\footnote{\url{http://bit.ly/btc-wizardry}}: et censurresistent, 
åbent, grænseløst netværk for værdioverførsel. Der er ingen forhæng, og 
magien er synlig for alle.\footnote{\url{https://github.com/bitcoin/bitcoin}}

\paragraph{Bitcoin lærte mig at se bag gardinet og konfrontere min økonomiske 
uvidenhed.}

% ---
%
% #### Down the Rabbit Hole
%
% - [Human Action][Ludwig von Mises] by Ludwig von Mises
% - [Why the Rich are Getting Richer][Robert Kiyosaki] by Robert Kiyosaki
%
% [real wizardry]: https://external-preview.redd.it/8d03MWWOf2HIyKrT8ThBGO4WFv-u25JaYqhbEO9b1Sk.jpg?width=683&auto=webp&s=dc5922d84717c6a94527bafc0189fd4ca02a24bb
% [visible to anyone]: https://github.com/bitcoin/bitcoin
%
% <!-- Wikipedia -->
% [alice]: https://en.wikipedia.org/wiki/Alice%27s_Adventures_in_Wonderland
% [carroll]: https://en.wikipedia.org/wiki/Lewis_Carroll

\chapter{Inflation}
\label{les:9}

\begin{chapquote}{Hjerterdronningen}
\enquote{Kære, her skal vi løbe så hurtigt, som vi kan, bare for at blive på 
stedet. Og hvis du ønsker at komme et sted hen, skal du løbe dobbelt så 
hurtigt som det.}
\end{chapquote}

At forsøge at forstå pengeinflation og hvordan et ikke-inflationært system 
som Bitcoin kunne ændre vores tilgang til tingene, var startpunktet for mit 
dyk ned i økonomien. Jeg vidste, at inflation var satsen, hvormed der blev 
skabt nye penge, men jeg vidste ikke meget ud over det.

Mens nogle økonomer argumenterer for, at inflation er en god ting, hævder 
andre, at \enquote{hårde} penge, der ikke let kan infleres - som vi havde 
det under guldstandarden - er essentielle for en sund økonomi. Bitcoin, der 
har et fast udbud på 21 millioner, er enig med sidstnævnte lejr.

Normalt er virkningerne af inflation ikke umiddelbart åbenlyse. Afhængigt af 
inflationsraten (samt andre faktorer) kan tiden mellem årsag og virkning være 
adskillige år. Ikke kun det, men inflation påvirker forskellige grupper af
mennesker mere end andre. Som Henry Hazlitt påpeger i \textit{Økonomi på ét 
minut}: \enquote{Økonomiens kunst består i ikke kun at se på det øjeblikkelige, 
men på de længerevarende virkninger af enhver handling eller politik; det 
består i at spore konsekvenserne af den politik ikke kun for én gruppe, men 
for alle grupper.}

Et af mine personlige øjeblikke af forståelse var erkendelsen af, at udstedelse 
af ny valuta - at trykke flere penge - er en \textit{fuldstændig} anderledes
økonomisk aktivitet end alle andre økonomiske aktiviteter. Mens reelle varer 
og reelle tjenester producerer reel værdi for virkelige mennesker, gør trykning 
af penge effektivt det modsatte: det tager værdi væk fra alle, der holder den 
inflerede valuta.

\begin{quotation}\begin{samepage}
\enquote{Bare inflation - det vil sige, bare udstedelse af flere penge, med 
konsekvensen af højere lønninger og priser - kan ligne skabelsen af mere 
efterspørgsel. Men i forhold til den faktiske produktion og udveksling af 
virkelige ting er det det ikke.} \begin{flushright} -- Henry Hazlitt
    \footnote{Henry Hazlitt, \textit{Økonomi på ét minut} \cite{hazlitt}}
\end{flushright}\end{samepage}\end{quotation}

Inflationens ødelæggende kraft bliver tydelig, så snart lidt inflation bliver 
til \textit{meget}. Hvis penge hyperinflates, bliver tingene grimme meget 
hurtigt.\footnote{\url{https://en.wikipedia.org/wiki/Hyperinflation}
\cite{wiki:hyperinflation}} Når den inflerende valuta falder fra hinanden, 
vil den ikke være i stand til at bevare værdi over tid, og folk vil skynde 
sig at få fat i varer, der måske kan det.

\paragraph{}
En anden konsekvens af hyperinflation er, at al den opsparing, som folk har 
samlet gennem deres liv, effektivt vil forsvinde. Pengesedlerne i din tegnebog 
vil stadig være der, selvfølgelig. Men det vil være præcis det: værdiløst papir.

\begin{figure}[h!]
    \centering
    \includegraphics[width=\textwidth]{assets/images/children-playing-with-money.png}
    \caption{Hyperinflation i Weimar-republikken (1921-1923)}
    \label{fig:children-playing-with-money}
\end{figure}

\paragraph{}
Penge mister værdi med såkaldt \enquote{mild} inflation også. Det sker bare 
langsomt nok til, at de fleste ikke bemærker den svækkelse af deres købekraft.
Og når trykpresserne kører, kan valuta let infleres, og hvad der plejede at 
være mild inflation, kan blive til en kraftig portion inflation med et tryk på 
en knap. Som Friedrich Hayek påpegede i et af sine essays, fører mild inflation 
normalt til åbenlys inflation.

\begin{quotation}\begin{samepage}
\enquote{'Mild' stabil inflation kan ikke hjælpe - den kan kun føre til åbenlys
inflation.} \begin{flushright} -- Friedrich Hayek\footnote{Friedrich Hayek, 
    \textit{1980'erne Arbejdsløshed og fagforeningerne} \cite{hayek-inflation}}
\end{flushright}\end{samepage}\end{quotation}

Inflation er særlig snedig, fordi den favoriserer dem, der er tættere på 
trykpresserne. Det tager tid, før de nyoprettede penge cirkulerer, og priserne 
justeres, så hvis du kan få fat i flere penge, før alle andres mister værdi, 
er du foran inflationskurven. Det er også derfor, inflation kan ses som en 
skjult skat, fordi regeringerne til sidst tjener på det, mens alle andre ender 
med at betale prisen.

\begin{quotation}\begin{samepage}
\enquote{Jeg mener ikke, det er en overdrivelse at sige, at historien i vid 
udstrækning er en historie om inflation, og normalt om inflationer, der er 
orkestreret af regeringer til fordel for regeringer.}
\begin{flushright} -- Friedrich Hayek\footnote{Friedrich Hayek, \textit{Godt Penge} \cite{hayek-good-money}}
\end{flushright}\end{samepage}\end{quotation}

\newpage

So far, all government-controlled currencies have eventually been
replaced or have collapsed completely. No matter how small the rate of
inflation, \enquote{steady} growth is just another way of saying exponential
growth. In nature as in economics, all systems which grow exponentially
will eventually have to level off or suffer from catastrophic collapse.

\paragraph{}
\enquote{It can't happen in my country,} is what you're probably thinking. 
You don't think that if you are from Venezuela, which is currently suffering 
from hyperinflation. With an inflation rate of over 1 million percent, money is
basically worthless. \cite{wiki:venezuela}

\paragraph{}
It might not happen in the next couple of years, or to the particular currency
used in your country. But a glance at the list of historical
currencies\footnote{See \textit{List of historical currencies} on Wikipedia.
\cite{wiki:historical-currencies}} shows that it will inevitably happen over a
long enough period of time. I remember and used plenty of those listed: the
Austrian schilling, the German mark, the Italian lira, the French franc, the
Irish pound, the Croatian dinar, etc. My grandma even used the Austro-Hungarian
Krone. As time moves on, the currencies currently in use\footnote{See
\textit{List of currencies} on Wikipedia \cite{wiki:list-of-currencies}} will
slowly but surely move to their respective graveyards. They will hyperinflate or
be replaced. They will soon be historical currencies. We will make them
obsolete.

\begin{quotation}\begin{samepage}
\enquote{History has shown that governments will inevitably succumb to the
temptation of inflating the money supply.}
\begin{flushright} -- Saifedean Ammous\footnote{Saifedean Ammous, 
    \textit{The Bitcoin
Standard} \cite{bitcoin-standard}}
\end{flushright}\end{samepage}\end{quotation}

\newpage

Why is Bitcoin different? In contrast to currencies mandated by the government,
monetary goods which are not regulated by governments, but by the laws of
physics\footnote{Gigi, \textit{Bitcoin's Energy Consumption - A shift in
perspective} \cite{gigi:energy}}, tend to survive and even hold their respective
value over time. The best example of this so far is gold, which, as the
aptly-named \textit{Gold-to-Decent-Suit Ratio}\footnote{History shows that the
price of an ounce of gold equals the price of a decent men's suit, according to 
Sionna investment managers \cite{web:gold-to-decent-suite-ratio}} shows, is 
holding its value over hundreds and even thousands of years. It might not be 
perfectly \enquote{stable} --- a questionable concept in the first place --- 
but the value it holds will at least be in the same order of magnitude.

If a monetary good or currency holds its value well over time and space,
it is considered to be \textit{hard}. If it can't hold its value, because it
easily deteriorates or inflates, it is considered a \textit{soft} currency. The
concept of hardness is essential to understand Bitcoin and is worthy of
a more thorough examination. We will return to it in the last economic
lesson: sound money.

\paragraph{}
As more and more countries suffer from
hyperinflation more and more people will have to face the reality
of hard and soft money. If we are lucky, maybe even some central bankers will be
forced to re-evaluate their monetary policies. Whatever might happen, the
insights I have gained thanks to Bitcoin will probably be invaluable, no matter
the outcome.

\paragraph{Bitcoin taught me about the hidden tax of inflation and the 
catastrophe of hyperinflation.}

% ---
%
% #### Down the Rabbit Hole
%
% - [Economics in One Lesson][Henry Hazlitt] by Henry Hazlitt
% - [1980's Unemployment and the Unions][unions] by Friedrich Hayek
% - [Good Money, Part II][good-money]: Volume Six of the Collected Works of F.A. Hayek
% - [The Bitcoin Standard] by Saifedean Ammous
% - [Hyperinflation][hyperinflates], [economic crisis in Venezuela][wiki-venezuela], [list of historical currencies], [list of currencies][currently in use] on Wikipedia
%
% [unions]: https://books.google.com/books/about/1980s_unemployment_and_the_unions.html?id=xM9CAQAAIAAJ
% [good-money]: https://books.google.com/books?id=l_A1vVIaYBYC
%
% [Henry Hazlitt]: https://mises.org/library/economics-one-lesson
% [hyperinflates]: https://en.wikipedia.org/wiki/Hyperinflation
% [inflation cannot help]: https://books.google.com/books?id=zZu3AAAAIAAJ&dq=%22only+while+it+accelerates%22&focus=searchwithinvolume&q=%22steady+inflation+cannot+help%22
% [history of inflation]: https://books.google.com/books?id=l_A1vVIaYBYC&pg=PA142&dq=%22history+is+largely+a+history+of+inflation%22&hl=en&sa=X&ved=0ahUKEwi90NDLrdnfAhUprVkKHUx1CmIQ6AEIKjAA#v=onepage&q=%22history%20is%20largely%20a%20history%20of%20inflation%22&f=false
% [wiki-venezuela]: https://en.wikipedia.org/wiki/Crisis_in_Venezuela#Economic_crisis
% [by the laws of physics]: https://link.medium.com/9fzq2L0J3S
% [\textit{Gold-to-Decent-Suit Ratio}]: https://www.businesswire.com/news/home/20110819005774/en/History-Shows-Price-Ounce-Gold-Equals-Price
% [The Bitcoin Standard]: https://thesaifhouse.wordpress.com/book/
%
% <!-- Wikipedia -->
% [alice]: https://en.wikipedia.org/wiki/Alice%27s_Adventures_in_Wonderland
% [carroll]: https://en.wikipedia.org/wiki/Lewis_Carroll

\chapter{Værdi}
\label{les:10}

\begin{chapquote}{Lewis Carroll, \textit{Alice i Eventyrland}}
\enquote{Det var den hvide kanin, der langsomt travede tilbage og så sig 
nervøst omkring, som om den havde mistet noget\ldots}
\end{chapquote}

Værdi er på en måde paradoksal, og der er flere teorier\footnote{Se 
\textit{Teori om værdi (økonomi)} på Wikipedia \cite{wiki:theory-of-value}}, 
der forsøger at forklare, hvorfor vi værdsætter visse ting mere end andre.
Mennesker har været opmærksomme på dette paradoks i tusinder af år. Som Plato 
skrev i sin dialog med Euthydemus, værdsætter vi nogle ting, fordi de er 
sjældne, og ikke kun baseret på deres nødvendighed for vores overlevelse.

\begin{quotation}\begin{samepage}
\enquote{Og hvis du er klog, vil du give denne samme rådgivning til dine 
elever også --- at de aldrig skal tale med nogen undtagen dig og hinanden. 
For det er det sjældne, Euthydemus, der er dyrebart, mens vand er billigst, 
selvom det er det bedste, som Pindar sagde.}
\begin{flushright} -- Plato\footnote{Plato, 
    \textit{Euthydemus} \cite{euthydemus}}
\end{flushright}\end{samepage}\end{quotation}

Denne værdiparadoks\footnote{Se \textit{Paradoks om værdi} på Wikipedia
\cite{wiki:paradox-of-value}} viser noget interessant om os mennesker: vi synes 
at værdsætte ting på en subjektiv\footnote{Se \textit{Subjektiv værditeori} på 
Wikipedia \cite{wiki:subjective-theory-of-value}} basis, men gør det med visse 
ikke-arbitrære kriterier. Noget kan være \textit{dyrebart} for os af 
forskellige årsager, men ting, vi værdsætter, deler visse karakteristika. Hvis 
vi kan kopiere noget meget let, eller hvis det er naturligt rigeligt, 
værdsætter vi det ikke.

Det ser ud til, at vi værdsætter noget, fordi det er sjældent (guld, 
diamanter, tid), svært eller arbejdskrævende at producere, ikke kan 
erstattes (et gammelt billede af en elsket person), er nyttigt på en måde, der 
gør det muligt for os at gøre ting, som vi ellers ikke kunne, eller en 
kombination af disse, såsom store kunstværker.

Bitcoin er alt dette: den er ekstremt sjælden (21 millioner), bliver stadig 
sværere at producere (halvering af belønningen), kan ikke erstattes (en mistet 
privat nøgle er tabt for evigt) og gør det muligt for os at gøre nogle meget 
nyttige ting. Det er sandsynligvis det bedste redskab til værdioverførsel over 
grænser, næsten immun over for censur og beslaglæggelse i processen, og det 
er desuden en selvstændig værdilager, der giver enkeltpersoner mulighed for at 
opbevare deres formue uafhængigt af banker og regeringer, bare for at nævne to.

\paragraph{Bitcoin lærte mig, at værdi er subjektiv, men ikke arbitrær.}

% ---
%
% #### Down the Rabbit Hole
%
% - [Euthydemus] by Plato
% - [Theory of Value][multiple theories], [Paradox of Value][paradox of value], [Subjective Theory of Value][subjective] on Wikipedia
%
% [Euthydemus]: http://www.perseus.tufts.edu/hopper/text?doc=Perseus:text:1999.01.0178:text=Euthyd.
% [Plato]: http://www.perseus.tufts.edu/hopper/text?doc=plat.+euthyd.+304b
%
% <!-- Wikipedia -->
% [multiple theories]: https://en.wikipedia.org/wiki/Theory_of_value_%28economics%29
% [paradox of value]: https://en.wikipedia.org/wiki/Paradox_of_value
% [subjective]: https://en.wikipedia.org/wiki/Subjective_theory_of_value
% [alice]: https://en.wikipedia.org/wiki/Alice%27s_Adventures_in_Wonderland
% [carroll]: https://en.wikipedia.org/wiki/Lewis_Carroll

\chapter{Penge}
\label{les:11}

\begin{chapquote}{Vismanden}
\enquote{I min ungdom, \ldots \
Holdt jeg alle mine lemmer meget smidige, \
Ved brugen af denne salve, \
fem shilling pr. boks -- \
Lad mig sælge dig et par.}
\end{chapquote}

Hvad er penge? Vi bruger det hver dag, og alligevel er dette spørgsmål 
overraskende svært at besvare. Vi er afhængige af det på store og små måder, 
og hvis vi har for lidt af det, bliver vores liv meget svært. Alligevel tænker 
vi sjældent på det, der angiveligt får verden til at dreje rundt. Bitcoin tvang
mig til at besvare dette spørgsmål igen og igen: Hvad er pokker penge?

I vores \enquote{moderne} verden tænker de fleste sandsynligvis på stykker 
papir, når de taler om penge, selvom det meste af vores penge blot er et tal 
på en bankkonto. Vi bruger allerede nuller og ettaller som vores penge, så 
hvordan adskiller Bitcoin sig? Bitcoin er anderledes, fordi det grundlæggende 
set er en meget anderledes \textit{type} penge end de penge, vi bruger i
øjeblikket. For at forstå dette, bliver vi nødt til at se nærmere på, hvad 
penge er, hvordan det opstod, og hvorfor guld og sølv blev brugt i det meste 
af handelshistorien.

\paragraph{}
Skaller, guld, sølv, papir, bitcoin. På et tidspunkt er \textbf{penge det, 
som folk bruger som penge}, uanset dets form og formål, eller mangel på samme.

Penge, som en opfindelse, er genial. En verden uden penge er vanvittigt 
kompliceret: Hvor mange fisk vil købe mig nye sko? Hvor mange køer vil købe 
mig et hus? Hvad nu hvis jeg ikke har brug for noget lige nu, men jeg skal af
med mine snart rådne æbler? Du behøver ikke meget fantasi for at indse, at en 
bytteøkonomi er vanvittigt ineffektiv.

Det fantastiske ved penge er, at det kan byttes til \textit{hvad som helst 
andet} --- det er en temmelig fantastisk opfindelse! Som Nick Szabo
\footnote{\url{http://unenumerated.blogspot.com/}} brilliant opsummerer i 
\textit{Shelling Out: The Origins of Money} \cite{shelling-out}, har mennesker 
brugt alle mulige ting som penge: perler lavet af sjældne materialer som 
elfenben, skaller eller specielle knogler, forskellige former for smykker og 
senere sjældne metaller som sølv og guld.

\begin{quotation}\begin{samepage}
\enquote{Set på denne måde er det mere typisk for et ædelmetal. I stedet for 
at ændre udbuddet for at holde værdien den samme, er udbuddet forudbestemt, 
og værdien ændres.}
\begin{flushright} -- Satoshi Nakamoto\footnote{Satoshi Nakamoto, i et svar 
    til Sepp
Hasslberger \cite{satoshi-precious-metal}}
\end{flushright}\end{samepage}\end{quotation}

At være de dovne skabninger, vi er, tænker vi ikke for meget over ting, der 
bare fungerer. Penge fungerer for det meste fint for de fleste af os. Ligesom 
med vores biler eller vores computere bliver de fleste af os kun nødt til at 
tænke på tingenes indre virkemåde, hvis de går i stykker. Mennesker, der så 
deres livsbesparelser forsvinde på grund af hyperinflation, kender værdien 
af hård valuta, ligesom mennesker, der så deres venner og familie forsvinde 
på grund af grusomhederne begået af Nazi-Tyskland eller Sovjetunionen, kender 
værdien af privatliv.

Det interessante ved penge er, at det er altomfattende. Penge er halvdelen af 
enhver transaktion, hvilket giver dem, der er ansvarlige for at skabe penge, 
enorm magt.

\begin{quotation}\begin{samepage}
\enquote{Da penge udgør den ene halvdel af enhver kommerciel transaktion, og 
hele civilisationer bogstaveligt talt stiger og falder baseret på kvaliteten 
af deres penge, taler vi om en imponerende magt, en der flyver under dække af 
natten. Det er magten til at væve illusioner, der ser ægte ud, så længe de 
varer. Det er selve kernen af FED's magt.}
\begin{flushright} -- Ron Paul\footnote{Ron Paul, 
    \textit{End the Fed} \cite{end-the-fed}}
\end{flushright}\end{samepage}\end{quotation}

Bitcoin fjerner fredeligt denne magt, da det gør op med pengeoprettelse, og 
det gør det uden brug af magt.

Penge gennemgik adskillige iterationer. De fleste iterationer var gode. De 
forbedrede vores penge på den ene eller anden måde. For nylig blev indre 
funktioner af vores penge imidlertid korrumperet. I dag skabes næsten alle 
vores penge simpelthen \textit{ud af ingenting} af magthaverne. For at forstå, 
hvordan dette kom til at være, måtte jeg lære om historien og efterfølgende 
nedgangen af penge.

Om det vil kræve en række katastrofer eller blot en monumentalt 
uddannelsesmæssig indsats at rette denne korruption, forbliver at se. 
Jeg beder til lydighedens guder, at det vil være det sidste.

\paragraph{Bitcoin lærte mig, hvad penge er.}

% ---
%
% #### Down the Rabbit Hole
%
% - [End the Fed][Ron Paul] by Ron Paul
% - [Money, blockchains, and social scalability][social-scalability] by Nick Szabo
%
% [social-scalability]: https://unenumerated.blogspot.co.at/2017/02/money-blockchains-and-social-scalability.html
%

\chapter{Historien og Nedgangen af Penge}
\label{les:12}

\begin{chapquote}{Lewis Carroll, \textit{Alice i Eventyrland}}
\enquote{De ville ikke huske de simple regler, deres venner havde givet dem, 
såsom, at hvis du kommer ind i ilden, vil den brænde dig, og at hvis du skærer 
din finger dybt med en kniv, bløder det generelt, og hun havde aldrig glemt, 
at hvis du drikker af en flaske mærket 'gift,' er det næsten sikkert, at du 
er uenig med det, før eller senere.}
\end{chapquote}

Mange mennesker tror, at penge er bakket op af guld, som er låst inde i store 
kasser, beskyttet af tykke mure. Dette holdt op med at være sandt for mange 
årtier siden. Jeg er ikke sikker på, hvad jeg tænkte, da jeg var i meget 
dybere problemer og stort set ingen forståelse af guld, papirpenge eller 
hvorfor det overhovedet skulle bakkes op af noget.

En del af at lære om Bitcoin er at lære om fiat-penge: hvad det betyder, 
hvordan det kom til at være, og hvorfor det måske ikke er den bedste idé, 
vi nogensinde har haft. Så hvad er præcis fiat-penge? Og hvordan endte vi 
med at bruge det?

Hvis noget pålægges ved \textit{fiat}, betyder det simpelthen, at det pålægges 
ved formel tilladelse eller forslag. Derfor er fiat-penge penge simpelthen 
fordi \textit{nogen} siger, at det er penge. Da alle regeringer i dag bruger 
fiat-valuta, er dette nogen \textit{din} regering. Desværre er du ikke 
\textit{fri} til at være uenig i denne værdi. Du vil hurtigt føle, at dette 
forslag er alt andet end ikke-voldeligt. Hvis du nægter at bruge denne 
papirvaluta til at drive forretning og betale skatter, vil de eneste mennesker, 
du vil kunne diskutere økonomi med, være dine cellekammerater.

Værdien af fiat-penge stammer ikke fra dens iboende egenskaber. Hvor god en 
bestemt type fiat-penge er, korrelerer kun med politisk og økonomisk 
(u)stabilitet hos dem, der drømmer det til eksistens. Dens værdi pålægges ved 
dekret, vilkårligt.

\begin{figure}[htbp]
  \centering
  \includegraphics[width=8cm]{assets/images/fiat-definition.png}
  \caption{fiat --- 'Lad det ske'}
  \label{fig:fiat-definition}
\end{figure}
  
\paragraph{}
Indtil for nylig blev to typer penge brugt: \textbf{råvaremønt}, lavet
af kostbare \textit{sager}, og \textbf{repræsentativ mønt}, som simpelthen
\textit{repræsenterer} den kostbare ting, mest i skrift.

\paragraph{}
Vi rørte allerede ved råvaremønt ovenfor. Folk brugte specielle knogler,
muslingeskaller og kostbare metaller som penge. Senere blev især mønter lavet af
kostbare metaller som guld og sølv brugt som penge. Den ældste mønt, der hidtil 
er fundet, er lavet af en naturlig blanding af guld og sølv og blev lavet for 
mere end 2700 år siden.\footnote{Ifølge den græske historiker Herodot, der 
skrev i det femte århundrede f.Kr., var lyderne de første, der brugte mønter
af guld og sølv.\cite{coinage-origins}} Hvis der er noget nyt i Bitcoin, er
 møntens koncept det ikke.

\newpage

\begin{figure}[htbp]
  \centering
  \includegraphics[width=5cm]{assets/images/lydian-coin-stater.png}
  \caption{Lydian elektromønt. Billede cc-by-sa Classical Numismatic
   Group, Inc.}
  \label{fig:lydian-coin-stater}
\end{figure}

Viser sig, at at gemme mønter, eller hodle, for at bruge dagens sprogbrug, er 
næsten lige så gammelt som mønter selv. Den tidligste mønt-hodler var en 
person, der lagde næsten hundrede af disse mønter i en gryde og begravde den 
i fundamentet af et tempel, kun for at blive fundet 2500 år senere. Ret god 
kold opbevaring, hvis du spørger mig.

En af ulemperne ved at bruge mønter af kostbare metaller er, at de kan 
klippes, hvilket effektivt devaluerer møntens værdi. Nye mønter kan præges 
af klipningerne, hvilket inflerer pengeudbuddet over tid og devaluerer hver 
enkelt mønt i processen. Folk skar bogstaveligt talt så meget af som de 
kunne slippe afsted med af deres sølvdollars. Jeg undrer mig over, hvilken 
slags \textit{Dollar Shave Club}-annoncer de havde dengang.

Da regeringer kun er cool med inflation, hvis de er dem, der gør det, blev 
der gjort forsøg på at stoppe denne gerilladebasering. I klassisk 
politi-og-røvere-stil blev møntklippere stadig mere kreative med deres 
teknikker, hvilket tvang \enquote{møntmestrene} til at blive endnu mere 
kreative med deres modforanstaltninger. Isaac Newton, den verdensberømte 
fysiker kendt for \textit{Principia Mathematica}, plejede at være en af 
disse mestre. Han tilskrives tilføjelsen af de små striber på mønternes 
side, som stadig er til stede i dag. Slut var dagene med let møntskæring.

\begin{figure}[htbp]
  \centering
  \includegraphics[width=\textwidth]{assets/images/clipped-coins.png}
  \caption{Klippede sølvmønter af varierende grad.}
  \label{fig:clipped-coins}
\end{figure}

Selv med disse metoder til møntdebasering\footnote{Udover klipning var svedning
(ryste mønterne i en pose og indsamle det slidte støv) og propning
(udhule en mønt i midten og banke mønten flad for at lukke hullet)
de mest fremtrædende metoder til møntdebasering. \cite{wiki:coin-debasement}}
holdt i skak lider mønter stadig af andre problemer. De er klodsede og ikke 
særlig praktiske at transportere, især når der skal foretages store
værdioverførsler. Det er ikke særlig praktisk at dukke op med en kæmpe pose 
sølvdollars hver gang du vil købe en Mercedes.

Når vi taler om tyske ting: Hvordan den amerikanske \textit{dollar} fik sit 
navn, er en anden interessant historie. Ordet \enquote{dollar} stammer fra det 
tyske ord \textit{Thaler}, forkortelse for 
\textit{Joachimsthaler}~\cite{wiki:thaler}. En Joachimsthaler var en mønt
præget i byen \textit{Sankt Joachimsthal}. Thaler er simpelthen en forkortelse 
for nogen (eller noget), der kommer fra dalen, og fordi Joachimsthal var 
\textit{dalen} for produktion af sølvmønter, kaldte folk simpelthen disse 
sølvpenge for \textit{Thaler.} Thaler (tysk) udviklede sig til daalders 
(hollandsk) og endelig dollars (engelsk).


\begin{figure}[htbp]
  \centering
  \includegraphics[width=5cm]{assets/images/joachimsthaler.png}
  \caption{Den originale 'dollar'. Saint Joachim er afbildet med 
  sin robe og troldmandshat. Billede cc-by-sa Wikipedia-bruger Berlin-George}
  \label{fig:joachimsthaler}
\end{figure}

Indførelsen af repræsentativ mønt markerede nedgangen for hårde
penge. Guldcertifikater blev indført i 1863, og cirka femten
år senere begyndte sølvdollaren også langsomt, men sikkert at blive
erstattet af en papirproxy: sølvcertifikatet. \cite{wiki:silver-certificate}

Det tog cirka 50 år fra indførelsen af de første sølvcertifikater, indtil
disse stykker papir udviklede sig til noget, vi i dag ville genkende som én
amerikansk dollar.

\begin{figure}[htbp]
  \centering
  \includegraphics[width=\textwidth]{assets/images/us-silver-dollar-note-smaller.png}
  \caption{En amerikansk sølvdollarseddel fra 1928. 'Betalelig til 
  bæreren på forlangende.' Billede cc-by-sa National Numismatic Collection 
  ved Smithsonian Institution}
  \label{fig:us-silver-dollar-note-smaller}
\end{figure}

Bemærk, at den amerikanske sølvdollar fra 1928 i
Figur~\ref{fig:us-silver-dollar-note-smaller} stadig går under navnet
\textit{sølvcertifikat}, hvilket indikerer, at dette faktisk bare er et 
dokument, der angiver, at bæreren af dette stykke papir skylder en sølvmønt. 
Det er interessant at se, at teksten, der angiver dette, blev mindre over tid. 
Sporet af \enquote{certifikat} forsvandt fuldstændigt efter et stykke tid og 
blev erstattet af forsikringen om, at disse er føderale reserve sedler.

Som nævnt ovenfor skete det samme med guld. Størstedelen af verden anvendte en
bimetallisk standard~\cite{wiki:bimetallism}, hvilket betyder, at mønter primært
var lavet af guld og sølv. At have certifikater for guld, indløselige i
guld mønter, var nok en teknologisk forbedring. Papir er mere praktisk,
lettere, og da det kan opdeles arbitrært ved blot at trykke et mindre
nummer på det, er det nemmere at opdele i mindre enheder.

For at minde bærerne (brugerne) om, at disse certifikater var
repræsentative for faktisk guld og sølv, blev de farvet derefter
og angav tydeligt dette på selve certifikatet. Du kan glidende læse
teksten fra top til bund:

\begin{quotation}\begin{samepage}
  \enquote{Dette bekræfter, at der er deponeret i skatkammeret for
  Amerikas Forenede Stater hundrede dollars i guld mønt, betalbar til
  bæreren på forlangende.}
\end{samepage}\end{quotation}
  
\begin{figure}[htbp]
  \centering
  \includegraphics[width=\textwidth]{assets/images/us-gold-cert-100-smaller.png}
  \caption{En amerikansk 100 dollars guldcertifikat fra 1928. 
  Billede cc-by-sa National Numismatic Collection, National Museum of American
  History.}
  \label{fig:us-gold-cert-100-smaller}
\end{figure}

I 1963 blev ordene \enquote{BETALBAR TIL BÆREREN PÅ FORLANGENDE} fjernet fra
alle nyudstedte sedler. Fem år senere blev indløsningen af papirsedler
til guld og sølv afsluttet.

Ordene, der antydede oprindelsen og ideen bag papirpenge, blev
fjernet. Den gyldne farve forsvandt. Alt, hvad der var tilbage, var papiret
og med det evnen for regeringen til at trykke så meget af det, den ønsker.

Med afskaffelsen af guldfoden i 1971 var denne århundredlange
illusion fuldendt. Penge blev den illusion, vi alle deler den dag i dag: 
fiat-penge. Det er værdifuldt, fordi nogen, der kommanderer en hær og driver 
fængsler, siger, at det er værdifuldt. Som det tydeligt kan læses på hver 
dollar seddel i omløb i dag, \enquote{DENNE SEDDEL ER LEGALT BETALINGSMIDDEL}. 
Med andre ord: Den har værdi, fordi sedlen siger det.

\begin{figure}[htbp]
  \centering
  \includegraphics[width=\textwidth]{assets/images/us-dollar-2004.jpg}
  \caption{En 2004-serie af amerikanske tyve dollarsedler, der 
  bruges i dag. 'DENNE SEDDEL ER LEGALT BETALINGSMIDDEL'}
  \label{fig:us-dollar-2004}
\end{figure}
  
I øvrigt er der en anden interessant lektion på dagens sedler,
skjult lige for øjnene af os. Den anden linje siger, at dette er legalt 
betalingsmiddel \enquote{TIL ALLE GÆLD, OFFENTLIG OG PRIVAT}. Hvad der måske 
er åbenlyst for økonomer, var overraskende for mig: Alle penge er gæld. Mit 
hoved gør stadig ondt på grund af det, og jeg vil lade udforskningen af 
forholdet mellem penge og gæld være en øvelse for læseren.

\paragraph{}
Som vi har set, blev guld og sølv brugt som penge i årtusinder. Over tid
blev mønter lavet af guld og sølv erstattet af papir. Papir
blev langsomt accepteret som betaling. Denne accept skabte en
illusion --- illusionen om, at papiret selv har værdi. Det endelige
skridt var at fuldstændigt bryde forbindelsen mellem repræsentationen og
det faktiske: at afskaffe guldfoden og overbevise alle om, at
papiret i sig selv er kostbart.

\paragraph{Bitcoin lærte mig om historien om penge og det største tryllenummer
i økonomisk historie: fiat-valuta.}

% ---
%
% #### Down the Rabbit Hole
%
% - [Shelling Out: The Origins of Money] by Nick Szabo
% - [Methods of Coin Debasement][coin debasement], [Thaler], [U.S. Silver Certificate][silver certificates], [Bimetallism][bimetallic standard] on Wikipedia
%
% [oldest coin]: https://www.britishmuseum.org/explore/themes/money/the_origins_of_coinage.aspx
% [coin debasement]: https://en.wikipedia.org/wiki/Methods_of_coin_debasement
% [Thaler]: https://en.wikipedia.org/wiki/Thaler
% [Berlin-George]: https://en.wikipedia.org/wiki/File:Bohemia,_Joachimsthaler_1525_Electrotype_Copy._VF._Obverse..jpg
% [silver certificates]: https://en.wikipedia.org/wiki/Silver_certificate_%28United_States%29
% [bimetallic standard]: https://en.wikipedia.org/wiki/Bimetallism
% [Shelling Out: The Origins of Money]: https://nakamotoinstitute.org/shelling-out/
%
% <!-- Wikipedia -->
% [alice]: https://en.wikipedia.org/wiki/Alice%27s_Adventures_in_Wonderland
% [carroll]: https://en.wikipedia.org/wiki/Lewis_Carroll

\chapter{Fractional Reserve Insanity}
\label{les:13}

\begin{chapquote}{Lewis Carroll, \textit{Alice in Wonderland}}
Ak, det var for sent: hun fortsatte med at vokse og vokse, og meget snart var 
der ikke plads til engang dette, og hun prøvede virkningen af at lægge sig 
ned, med den ene albue mod døren og den anden arm krøllet omkring hendes hoved. 
Alligevel fortsatte hun med at vokse, og som en sidste udvej stak hun den ene 
arm ud af vinduet, og det ene ben op i skorstenen, og sagde til sig selv
\enquote{nu kan jeg ikke gøre mere — hvad vil der ske med mig?}
\end{chapquote}

Værdi og penge er ikke trivielle emner, især ikke i dagens tid. Processen med 
pengeoprettelse i vores banksystem er lige så ikke-triviel, og jeg kan ikke 
slippe følelsen af, at det er med vilje. Det, jeg tidligere kun har stødt på 
i akademiske og juridiske tekster, synes også at være almindelig praksis i 
finansverdenen: intet forklares i enkle vilkår, ikke fordi det er virkelig 
komplekst, men fordi sandheden er skjult bag lag og lag af jargon og 
\textit{tilladt} kompleksitet. \enquote{Ekspansiv pengepolitik, kvantitativ 
lempelse, finansiel stimulans til økonomien.} Publikum nikker enig,
hipnotiseret af de prangende ord.

Fractional reserve banking og kvantitativ lempelse er to af disse prangende ord,
der slører, hvad der virkelig sker, ved at maskere det som komplekst og 
svært at forstå. Hvis du skulle forklare dem til en femårig, vil vanviddet af 
begge blive tydeligt hurtigt.

Godfrey Bloom sagde det meget bedre, da han talte til Europa-Parlamentet 
under en fælles debat:

\begin{quotation}\begin{samepage}
\enquote{[...] I forstår ikke rigtig bankvirksomhedens koncept. Alle
bankerne er fallit. Bank Santander, Deutsche Bank, Royal Bank of
Scotland --- de er alle fallit! Og hvorfor er de fallit? Det er ikke en
Guds handling. Det er ikke en slags tsunami. De er fallit, fordi vi
har et system kaldet 'fractional reserve banking', hvilket betyder, at
bankerne kan låne penge ud, som de faktisk ikke har! Det er en kriminel
skandale, og det har stået på for længe. [...]
Vi har forfalskning --- nogle gange kaldet kvantitativ lempelse ---
men forfalskning ved ethvert andet navn. Den kunstige
pengesedelproduktion, som hvis enhver almindelig person gjorde det, ville de 
ende i fængsel i meget lang tid [...] og indtil vi begynder at sende
bankfolk --- og jeg inkluderer centralbankfolk og politikere --- i
fængsel for denne foragtelige handling, vil det fortsætte.}
\begin{flushright} -- Godfrey Bloom\footnote{Fælles debat om
bankunionen~\cite{godfrey-bloom}}
\end{flushright}\end{samepage}\end{quotation}

Lad mig gentage den mest vigtige del: Banker kan låne penge ud, som de faktisk 
ikke har.

Takket være fractional reserve banking skal en bank kun beholde en lille
\textit{fraktion} af hver dollar, den får. Det er et sted mellem $0$ og $10\%$,
normalt i den lavere ende, hvilket gør tingene endnu værre.

Lad os bruge et konkret eksempel for bedre at forstå denne skøre idé: En
fraktion på $10\%$ vil gøre tricket, og vi burde være i stand til at lave alle
beregningerne i vores hoved. Win-win. Så hvis du tager \$100 til en
bank --- fordi du ikke ønsker at opbevare det under din madras --- skal de kun
beholde den aftalte \textit{fraktion} af det. I vores eksempel ville det
være \$10, fordi 10\% af \$100 er \$10. Nemt, ikke sandt?

Så hvad gør banker med resten af pengene? Hvad sker der med dine \$90? De
gør, hvad banker gør, de låner dem ud til andre mennesker. Resultatet er en 
pengemultiplikatoreffekt, der øger pengeforsyningen i økonomien enormt
(Figur~\ref{fig:money-multiplier}). Dit indledende indskud på \$100 vil snart
blive til \$190. Ved at låne en fraktion på 90\% af de nyoprettede \$90 vil der
snart være \$271 i økonomien. Og \$343.90 derefter. Pengeforsyningen er
recursivt stigende, da banker bogstaveligt talt låner penge, de ikke
har~\cite{wiki:money-multiplier}. Uden et eneste Abrakadabra forvandler
banker magisk \$100 til tusind dollars eller mere. Det viser sig, at 10x er 
nemt. Det kræver kun et par udlånsskridt.

\begin{center}
  \centering
  \includegraphics[width=\textwidth]{assets/images/money-multiplier.png}
  \captionof{figure}{Pengemultiplikatoreffekten}
  \label{fig:money-multiplier}
\end{center}
  
\paragraph{}
Misforstå mig ikke: Der er ikke noget galt med at låne penge ud. Der er
ikke noget galt med renter. Der er endda ikke noget galt med gode,
gamle almindelige banker for at opbevare din formue et mere sikkert sted end i
din strømpeskuffe.

Centralbanker derimod er en anden størrelse. Grusomheder inden for finansiel
regulering, halvt offentlige halvt private, der leger gud med noget, der
påvirker alle, der er en del af vores globale civilisation, uden en
samvittighed, kun interesseret i den umiddelbare fremtid og tilsyneladende
uden nogen form for ansvarlighed eller auditabilitet (se Figur~\ref{fig:bsg}).

\begin{center}
  \centering
  \includegraphics[width=\textwidth]{assets/images/bsg.jpg}
  \captionof{figure}{Yellen er stærkt imod revision af Fed, mens Bitcoin Sign 
  Guy stærkt er tilhænger af at købe bitcoin.}
  \label{fig:bsg}
\end{center}

Mens Bitcoin stadig er inflationsmæssig, vil det ophøre med at være det ret 
snart. Den strengt begrænsede forsyning af 21 millioner bitcoins vil med 
tiden fjerne inflationen fuldstændigt. Vi har nu to monetære verdener: en
inflationær, hvor penge trykkes vilkårligt, og verden af
Bitcoin, hvor den endelige forsyning er fast og let auditabel for alle.
Den ene påtvinges os ved vold, den anden kan tilsluttes af enhver, der ønsker
det. Ingen indgangsbarrierer, ingen at spørge om tilladelse.
Frivillig deltagelse. Det er skønheden ved Bitcoin.

Jeg ville argumentere for, at argumentet mellem keynesianske\footnote{Teorier i
henhold til John Maynard Keynes og hans disciple~\cite{wiki:keynesian}} og
østrigske\footnote{Skole inden for økonomisk tænkning baseret på metodologisk
individualisme~\cite{wiki:austrian}} økonomer ikke længere er rent akademisk.
Satoshi formåede at opbygge et system til værdioverførsel på steroider, hvilket 
skabte den mest lydige valuta, der nogensinde har eksisteret i processen. På 
den ene eller anden måde vil flere og flere mennesker lære om fidusen, som er 
fractional reserve banking. Hvis de når til lignende konklusioner som de fleste 
østrigere og bitcoinere, kunne de tilslutte sig det stadig voksende internet af 
penge. Ingen kan stoppe dem, hvis de vælger at gøre det.

\paragraph{Bitcoin lærte mig, at fractional reserve banking er ren vanvid.}

% ---
%
% #### Down the Rabbit Hole
%
% - [The Creature From Jekyll Island] by G. Edward Griffin
% - [Money Multiplier][money multiplier], [Keynesian Economics][Keynesian], [Austrian School][Austrian] on Wikipedia
%
% [The Creature From Jekyll Island]: https://archive.org/details/pdfy--Pori1NL6fKm2SnY
%
% [joint debate]: https://www.youtube.com/watch?v=hYzX3YZoMrs
% [money multiplier]: https://en.wikipedia.org/wiki/Money_multiplier
% [auditability]: https://i.ytimg.com/vi/ThFGs347MW8/maxresdefault.jpg
% [Keynesian]: https://en.wikipedia.org/wiki/Keynesian_economics
% [Austrian]: https://en.wikipedia.org/wiki/Austrian_School
%
% <!-- Wikipedia -->
% [alice]: https://en.wikipedia.org/wiki/Alice%27s_Adventures_in_Wonderland
% [carroll]: https://en.wikipedia.org/wiki/Lewis_Carroll

Den vigtigste lektion, jeg har lært af Bitcoin, er, at på lang sigt er hård valuta overlegen blød valuta. Hård valuta, også kaldet \textit{sound money} (lydige penge), er enhver globalt handlet valuta, der fungerer som en pålidelig værdibevarer.

Selvfølgelig er Bitcoin stadig ung og volatil. Kritikere vil sige, at den ikke pålideligt bevarer værdi. Argumentet om volatilitet overser dog pointen. Volatilitet skal forventes. Markedet vil tage tid om at finde den retfærdige pris på denne nye valuta. Derudover påpeges det ofte spøgende, at det er baseret på en målefejl. Hvis du tænker i dollars, vil du overse, at en bitcoin altid vil være en bitcoin.

\begin{quotation}\begin{samepage}
\enquote{En fast pengemængde eller en mængde, der kun ændres i overensstemmelse med
objektive og beregnelige kriterier, er en nødvendig betingelse for en
meningsfuld retfærdig pris på penge.}
\begin{flushright} -- Fr. Bernard W. Dempsey, S.J.\footnote{Perry J. Roets, S.J., \textit{Review of Social Economy} \cite{review-social-economy}}
\end{flushright}\end{samepage}\end{quotation}

\newpage

Som en hurtig tur gennem gravpladsen forglemte valutaer har vist,
vil penge, der kan trykkes, blive trykt. Indtil videre har ingen mennesker i
historien været i stand til at modstå denne fristelse.

Bitcoin gør op med fristelsen til at trykke penge på en genial
måde. Satoshi var opmærksom på vores grådighed og fejlbarlighed --- derfor valgte han
noget mere pålideligt end menneskelig tilbageholdenhed: matematik.

\begin{center}
  \centering
  \begin{equation}
  \sum\limits_{i=0}^{32} \frac{21000 \lfloor \frac{50*10^8}{2^i} \rfloor}{10^8}
  \end{equation}
  \captionof{figure}{Bitcoin's supply formula}
  \label{fig:supply-formula-white}
\end{center}

Selvom denne formel er nyttig til at beskrive Bitcoins forsyning, er den faktisk
ingen steder at finde i koden. Udstedelse af nye bitcoins sker på en
algoritmisk kontrolleret måde ved at reducere belønningen, der betales til
minearbejdere hvert fjerde år~\cite{btcwiki:supply}. Formlen ovenfor bruges til at
sammenfatte hurtigt, hvad der sker under motorhjelmen. Hvad der virkelig sker, kan bedst
ses ved at se på ændringen i blokbelønningen, belønningen udbetalt til den, der
finder en gyldig blok, hvilket groft sker hvert 10. minut.

\begin{center}
  \includegraphics[width=\textwidth]{assets/images/you-are-here.png}
  \captionof{figure}{Bitcoins kontrollerede forsyning}
  \label{fig:you-are-here.png}
\end{center}

Formler, logaritmiske funktioner og eksponentialfunktioner er ikke nøjagtigt
intuitive at forstå. Konceptet \textit{lydighed} kan være lettere at
forstå, hvis det betragtes på en anden måde. Når vi først ved, hvor meget der er
af noget, og når vi ved, hvor svært det er at producere eller
få fat i denne ting, forstår vi straks dens værdi. Hvad der er sandt for
Picassos malerier, Elvis Presleys guitarer og Stradivarius-violiner,
er også sandt for fiatvaluta, guld og bitcoins.

Hårdheden af fiatvaluta afhænger af, hvem der har ansvaret for
de respektive trykkerier. Nogle regeringer er måske mere villige til at
trykke store mængder valuta end andre, hvilket resulterer i en svagere
valuta. Andre regeringer kan være mere restriktive i deres penges
trykning, hvilket resulterer i en hårdere valuta.

\begin{samepage}\begin{quotation}
\enquote{En vigtig aspect af denne nye virkelighed er, at institutioner som
Federal Reserve ikke kan gå konkurs. De kan printe enhver mængde penge, de
måtte have brug for, til næsten ingen omkostninger.}
\begin{flushright} -- Jörg Guido Hülsmann\footnote{Jörg Guido Hülsmann, \textit{The
Ethics of Money Production}~\cite{hulsmann2008ethics}}
\end{flushright}\end{quotation}\end{samepage}

Før vi havde fiatvalutaer, blev lydigheden af penge bestemt af
de naturlige egenskaber ved det, vi brugte som penge. Mængden
af guld på jorden er begrænset af fysikkens love. Guld er sjældent, fordi
supernovaer og kollisioner mellem neutronstjerner er sjældne. \enquote{Strømmen} af guld er
begrænset, fordi udvinding af det er en stor indsats. Som et tungt element
er det hovedsageligt begravet dybt under jorden.

Ophævelsen af guldfoden åbnede op for en ny virkelighed: tilføjelse af nye penge
kræver blot en dråbe blæk. I vores moderne verden kræver tilføjelse af et par
nuller til saldoen på en bankkonto endnu mindre indsats: at flippe et par bits i en
bankcomputer er nok.

Princippet beskrevet ovenfor kan udtrykkes mere generelt som
forholdet mellem \enquote{lager} og \enquote{strøm}. Ganske enkelt er \textit{lageret} hvor meget af
noget der i øjeblikket er til stede. For vores formål er lageret en måling
af den nuværende pengeforsyning. \textit{Strømmen} er, hvor meget der produceres
over en periode (f.eks. per år). Nøglen til at forstå lydighed
af penge ligger i forståelsen af dette lager-til-strøm-forhold.

Beregning af lager-til-strøm-forholdet for fiatvaluta er vanskelig, fordi hvor
mange penge der er, afhænger af, hvordan du ser på det.~\cite{wiki:money-supply} Du
kunne tælle kun pengesedler og mønter (M0), tilføje rejsechecks og checke
indskud (M1), tilføje opsparingskonti og investeringsfonde og nogle andre ting (M2),
og endda tilføje indlånscertifikater til alt dette (M3). Desuden varierer, hvordan alt
dette er defineret og målt, fra land til land, og da den amerikanske
centralbank stoppede med at offentliggøre \cite{web:fed-m3} tal for M3, må vi
nøjes med pengemængden M2. Jeg ville gerne verificere disse
tal, men jeg gætter på, at vi må stole på centralbanken for nu.

Guld, en af de sjældneste metaller på jorden, har det højeste lager-til-strøm
forhold. Ifølge US Geological Survey er lidt mere end 190.000 tons blevet udvundet.
I de sidste få år er der blevet udvundet omkring 3100 tons guld
årligt.~\cite{mineral-commodity-summaries}

Ved hjælp af disse tal kan vi nemt beregne lager-til-strøm-forholdet for
guld (se figur~\ref{fig:stock-to-flow-gold}).

\begin{center}
  \centering
  \begin{equation}
  \frac{190,000 t}{3,100 t} = ~ 61
  \end{equation}
  \captionof{figure}{Lager-til-flow-forholdet for guld}
  \label{fig:stock-to-flow-gold}
\end{center}

Intet har en højere lager-til-flow-forhold end guld. Dette er grunden til, at guld indtil nu
var den hårdeste, mest lydende valuta, der eksisterede. Det bliver ofte sagt, at alt det guld
der hidtil er udvundet, ville kunne være i to olympiske swimmingpools. Ifølge mine
beregninger\footnote{\url{https://bit.ly/gold-pools}}, ville vi have brug for fire. Så
måske skal dette opdateres, eller olympiske swimmingpools er blevet mindre.

Indtast Bitcoin. Som du sandsynligvis ved, var bitcoin-minedrift alting i
de sidste par år. Dette skyldes, at vi stadig er i de tidlige
faser af det, der kaldes \textit{belønningsæraen}, hvor minedriftsnoder
belønnes med \textit{meget} bitcoin for deres beregningsindsats. Vi er
i øjeblikket i belønningsæra nummer 3, som begyndte i 2016 og slutter i
tidlig 2020, sandsynligvis i maj. Mens bitcoin-udbuddet er forudbestemt,
tillader Bitcoin's indre funktioner kun omtrentlige datoer.
Ikke desto mindre kan vi med sikkerhed forudsige, hvor høj Bitcoins
lager-til-flow-forhold vil være. Advarsel: det vil være højt.

Hvor højt? Nå, det viser sig, at Bitcoin vil blive uendeligt svært (se
Figur~\ref{fig:stock-to-flow-white-cropped}).

\begin{center}
  \includegraphics[width=\textwidth]{assets/images/stock-to-flow-white-cropped.png}
  \captionof{figure}{Visualisering af lager og flow for USD, guld og Bitcoin}
  \label{fig:stock-to-flow-white-cropped}
\end{center}

\paragraph{}
På grund af en eksponentiel reduktion af minedriftsbelønningen vil tilstrømningen af nye
bitcoin aftage, hvilket resulterer i en himmelflugt i lager-til-flow-forholdet.
Det vil indhente guld i 2020, kun for at overgå det fire år senere ved at
fordoble dets soliditet igen. En sådan fordobling vil forekomme 64 gange i
alt. Takket være eksponentiernes kraft vil antallet af minedriftsbitcoin
per år falde under 100 bitcoin om 50 år og under 1 bitcoin om
75 år. Den globale hane, som er blokbelønningen, vil tørre ud
omkring år 2140 og stoppe effektivt produktionen af
bitcoin. Dette er et langt spil. Hvis du læser dette, er du stadig
tidligt på den.

\begin{center}
  \includegraphics[width=\textwidth]{assets/images/soundness-over-time.png}
  \captionof{figure}{Stigende lager-til-flow-forhold for bitcoin sammenlignet med guld}
  \label{fig:soundness-over-time}
\end{center}

Når bitcoin nærmer sig uendeligt lager-til-flow-forhold, vil det være den
lydende valuta, der nogensinde har eksisteret. Uendelig soliditet er svær at slå.

Set gennem økonomiens linse er Bitcoin's \textit{sværhedsjustering}
formentlig dens vigtigste komponent. Hvor svært det er at mine bitcoin afhænger
af, hvor hurtigt nye bitcoins mines.\footnote{Det afhænger faktisk af, hvor
hurtigt gyldige blokke findes, men til vores formål er dette det samme som
\enquote{at mine bitcoins} og vil være det i de næste 120 år.} Det er den dynamiske
justering af netværkets minedifficulty, der gør det muligt for os at forudsige dens
fremtidige udbud.

Simpliciteten af sværhedsjusteringsalgoritmen kan distrahere fra dens dybde,
men sværhedsjusteringen er virkelig en revolution af Einsteinianiske proportioner.
Den sikrer, at uanset hvor meget eller hvor lidt indsats der lægges i minedrift, vil Bitcoin's kontrollerede udbud ikke blive forstyrret. I modsætning til enhver anden ressource, uanset hvor meget
energi nogen vil investere i at mine bitcoin, vil den samlede belønning ikke
stige.

Præcis som $E=mc^2$ dikterer den universelle hastighedsgrænse i vores univers,
dikterer Bitcoin's sværhedsjustering \textbf{den universelle pengelimit}
i Bitcoin.

\paragraph{}
Hvis det ikke var for denne sværhedsjustering, ville alle bitcoins allerede være minedriftet.
Hvis det ikke var for denne sværhedsjustering, ville Bitcoin sandsynligvis ikke
have overlevet i sin barndom. Det er det, der sikrer netværket i dets belønningsæra.
Det er det, der sikrer en stabil og retfærdig fordeling\footnote{Dan Held,
\textit{Bitcoin's Distribution was Fair}~\cite{distribution-was-fair}} af nye
bitcoin. Det er termostaten, der regulerer Bitcoin's pengepolitik.

Einstein viste os noget nyt: uanset hvor hårdt du skubber til et
objekt, vil du på et vist tidspunkt ikke være i stand til at få mere hastighed ud af
det. Satoshi viste os også noget nyt: uanset hvor hårdt du graver
efter dette digitale guld, vil du på et vist tidspunkt ikke være i stand til at få mere
bitcoin ud af det. For første gang i menneskets historie har vi en
pengemæssig vare, som uanset hvor meget du prøver, ikke vil være i stand til
at producere mere af.

\paragraph{Bitcoin lærte mig, at lydende penge er essentielle.}

% ---
%
% #### Through the Looking-Glass
%
% - [Bitcoin's Energy Consumption: A Shift in Perspective][much energy]
%
% #### Down the Rabbit Hole
%
% - [The Ethics of Money Production][Jörg Guido Hülsmann] by Jörg Guido Hülsmann
% - [Mineral Commodity Summaries 2019][last few years] by the United States Geological Survey
% - [Bitcoin’s Distribution was Fair][fair distribution] by Dan Held
% - [Bitcoin's Controlled Supply][algorithmically controlled] on the Bitcoin Wiki
% - [Money Supply][how much money there is], [Speed of Light][universal speed limit] on Wikipedia
%
% <!-- Internal -->
% [much energy]: 
%
% [Fr. Bernard W. Dempsey, S.J.]: https://www.jstor.org/stable/29769582
% [Jörg Guido Hülsmann]: https://mises.org/sites/default/files/The%20Ethics%20of%20Money%20Production_2.pdf
% [stopped publishing]: https://www.federalreserve.gov/Releases/h6/discm3.htm
% [last few years]: https://minerals.usgs.gov/minerals/pubs/mcs/2018/mcs2018.pdf
% [my calculations]: https://www.wolframalpha.com/input/?i=volume+of+190000+metric+tons+gold+%2F+olympic+swimming+pool+volume
% [fair distribution]: https://blog.picks.co/bitcoins-distribution-was-fair-e2ef7bbbc892
%
% <!-- Bitcoin Wiki -->
% [algorithmically controlled]: https://en.bitcoin.it/wiki/Controlled_supply
%
% <!-- Wikipedia -->
% [how much money there is]: https://en.wikipedia.org/wiki/Money_supply
% [universal speed limit]: https://en.wikipedia.org/wiki/Speed_of_light#Upper_limit_on_speeds
% [alice]: https://en.wikipedia.org/wiki/Alice%27s_Adventures_in_Wonderland
% [carroll]: https://en.wikipedia.org/wiki/Lewis_Carroll

\part{Teknologi}
\label{ch:teknologi}
\chapter*{Teknologi}

\begin{chapquote}{Lewis Carroll, \textit{Alice i Eventyrland}}
\enquote{Nu skal jeg klare det bedre denne gang,} sagde hun til sig selv og 
begyndte med at tage den lille gyldne nøgle og låse op for døren, der førte ud 
i haven. 
\end{chapquote}

Gyldne nøgler, ure der kun virker af og til, kapløb for at løse
mærkelige gåder og byggere uden ansigter eller navne. Hvad lyder som
eventyr fra Eventyrland er dagligdag i Bitcoin-verdenen.

Som vi udforskede i kapitel~\ref{ch:økonomi}, er store dele af det nuværende 
finansielle system systematisk ødelagt. Ligesom Alice kan vi kun håbe på at 
klare det bedre denne gang. Men takket være en pseudonym opfinder har vi utrolig
avanceret teknologi til at støtte os denne gang: Bitcoin.

At løse problemer i en radikalt decentraliseret og fjendtlig miljø
kræver unikke løsninger. Hvad der ellers ville være trivielle problemer at løse
er alt andet end det i denne mærkelige verden af noder. Bitcoin er afhængig af 
stærk kryptografi for de fleste løsninger, i det mindste hvis man ser det gennem
teknologiens linse. Præcis hvor stærk denne kryptografi er, vil blive udforsket 
i en af de efterfølgende lektioner.

Kryptografi er det, Bitcoin bruger til at fjerne tillid til myndigheder.
I stedet for at stole på centraliserede institutioner, er systemet afhængigt 
af universets endelige autoritet: fysik. Dog er der stadig nogle korn af tillid 
tilbage. Vi vil undersøge disse korn i den anden lektion af dette kapitel.

~

\begin{samepage}
Del~\ref{ch:teknologi} -- Teknologi:

\begin{enumerate}
  \setcounter{enumi}{14}
  \item Styrke i antal
  \item Refleksioner over \enquote{Stol ikke, verificer}
  \item At fortælle tid kræver arbejde
  \item Bevæg dig langsomt og ødelæg ikke ting
  \item Privatlivet er ikke dødt
  \item Cypherpunks skriver kode
  \item Metaforer for Bitcoins fremtid
\end{enumerate}
\end{samepage}

De sidste par lektioner udforsker ethos inden for teknologisk udvikling i
Bitcoin, hvilket argumenteres for at være lige så vigtigt som teknologien selv. 
Bitcoin er ikke den næste glimrende app på din telefon. Det er grundlaget for 
en ny økonomisk virkelighed, hvorfor Bitcoin bør behandles som finansiel 
software på atomniveau.

Hvor er vi i denne finansielle, samfundsmæssige og teknologiske revolution? 
Netværk og teknologier fra fortiden kan tjene som metaforer for Bitcoins 
fremtid, hvilket udforskes i den sidste lektion af dette kapitel.

Endnu engang, spænd sikkerhedsselen og nyd turen. Som alle eksponentielle 
teknologier, er vi ved at gå parabolisk.
\chapter{Styrke i Tal}
\label{les:15}

\begin{chapquote}{Lewis Carroll, \textit{Alice i Eventyrland}}
\enquote{Lad mig se: fire gange fem er tolv, og fire gange seks er tretten, og fire gange syv er fjorten - åh nej! Jeg vil aldrig nå tyve på denne måde!}
\end{chapquote}

Tal er en essentiel del af vores hverdag. Store tal er dog ikke noget, de fleste af os er alt for fortrolige med. De største tal, vi måske støder på i hverdagen, ligger i størrelsesordenen af millioner, milliarder eller billioner. Vi kan læse om millioner af mennesker i fattigdom, milliarder af dollars brugt på bankredninger og billioner af national gæld. Selvom det er svært at forstå disse overskrifter, er vi på en måde komfortable med størrelsen af disse tal.

Selvom vi måske virker komfortable med milliarder og billioner, begynder vores intuition allerede at svigte med tal af denne størrelsesorden. Har du en fornemmelse af, hvor lang tid du ville skulle vente, før en million/milliard/billion sekunder passerer? Hvis du er som mig, er du tabt uden faktisk at knuse tallene.

Lad os tage et nærmere kig på dette eksempel: forskellen mellem hver er en stigning med tre størrelsesordener: $10^6$, $10^9$, $10^{12}$. At tænke i sekunder er ikke særlig nyttigt, så lad os oversætte dette til noget, vi kan forstå:

\begin{itemize}
  \item $10^6$: Ét million sekunder var $1 \frac{1}{2}$ uge siden.
  \item $10^9$: Ét milliard sekunder var næsten 32 år siden.
  \item $10^{12}$: Ét billion sekunder siden var Manhattan dækket af et tykt lag
  is.\footnote{Ét billion sekunder ($10^{12}$) var $31710$ år siden. Den Sidste Glaciale
  Maksimum var for $33,000$ år siden.~\cite{wiki:LGM}}
\end{itemize}

\begin{center}
  \includegraphics[width=\textwidth]{assets/images/xkcd-1225.png}
  \captionof{figure}{Ca. 1 billion sekunder siden. Kilde: xkcd 1225}
  \label{fig:xkcd-1225}
\end{center}

Så snart vi træder ind i den næsten astronomiske verden af moderne kryptografi, svigter vores intuition katastrofalt. Bitcoin er bygget omkring store tal og den virtuelle umulighed af at gætte dem. Disse tal er langt, langt større end noget, vi måske støder på i dagligdagen. Mange størrelsesordener større. At forstå, hvor store disse tal virkelig er, er afgørende for at forstå Bitcoin som helhed.

Lad os tage SHA-256\footnote{SHA-256 er en del af SHA-2-familien af kryptografiske hashfunktioner udviklet af NSA.~\cite{wiki:sha2}}, en af hashfunktionerne\footnote{Bitcoin bruger SHA-256 i sin blokhåndteringsalgoritme.~\cite{btcwiki:block-hashing}} brugt i Bitcoin, som et konkret eksempel. Det er kun naturligt at tænke på 256 bits som \enquote{to hundrede seksoghalvtreds,} hvilket slet ikke er et stort tal. Men tallet i SHA-256 handler om størrelsesordener - noget vores hjerner ikke er godt rustet til at håndtere.

Mens bitlængde er en praktisk metrik, går den sande betydning af 256-bit sikkerhed tabt i oversættelsen. På samme måde som millioner ($10^6$) og milliarder ($10^9$) ovenfor, er tallet i SHA-256 om størrelsesordener ($2^{256}$).

Så, hvor stærk er SHA-256 præcist?

\begin{quotation}\begin{samepage}
\enquote{SHA-256 er meget stærk. Det er ikke som det inkrementelle skridt fra MD5
til SHA1. Den kan vare adskillige årtier, medmindre der sker en massiv
gennembrudsangreb.}
\begin{flushright} -- Satoshi Nakamoto\footnote{Satoshi Nakamoto, i et svar på spørgsmål om SHA-256 kollisioner. \cite{satoshi-sha256}}
\end{flushright}\end{samepage}\end{quotation}

Lad os stave tingene ud. $2^{256}$ svarer til følgende tal:

\begin{quotation}\begin{samepage}
    115 quattuorvigintillion 792 trevigintillion 89 duovigintillion 237
    unvigintillion 316 vigintillion 195 novemdecillion 423 octodecillion 570
    septendecillion 985 sexdecillion 8 quindecillion 687 quattuordecillion 907
    tredecillion 853 duodecillion 269 undecillion 984 decillion 665 nonillion
    640 octillion 564 septillion 39 sextillion 457 quintillion 584 quadrillion 7
    trillion 913 billion 129 million 639 thousand 936.
\end{samepage}\end{quotation}

Det er mange nonillioner! At forstå dette tal er stort set umuligt. Der er intet i det fysiske univers at sammenligne det med. Det er langt større end antallet af atomer i det observerbare univers. Menneskehjernen er simpelthen ikke lavet til at forstå det.

\newpage

En af de bedste visualiseringer af den sande styrke af SHA-256 er en video af Grant Sanderson. Passende navngivet \textit{\enquote{Hvor sikkert er 256 bit sikkerhed?}}\footnote{Se videoen på \url{https://youtu.be/S9JGmA5_unY}}, viser den smukt, hvor stort et 256-bit rum er. Gør dig selv en tjeneste og brug fem minutter på at se den. Ligesom alle andre \textit{3Blue1Brown}-videoer er den ikke kun fascinerende, men også exceptionelt godt lavet. Advarsel: Du kan ende med at falde ned i et matematisk kaninhul.

\begin{center}
  \includegraphics[width=\textwidth]{assets/images/youtube-vid-inverted.png}
  \captionof{figure}{Illustration af SHA-256 sikkerhed. Oprindeligt grafik af Grant Sanderson alias 3Blue1Brown.}
  \label{fig:youtube-vid-inverted}
\end{center}

Bruce Schneier~\cite{web:schneier} brugte de fysiske grænser for beregning til at sætte dette tal i perspektiv: selv hvis vi kunne bygge en optimal computer, der ville bruge enhver tilført energi til at vende bits perfekt~\cite{wiki:landauer}, bygge en Dyson-sfære\footnote{En Dyson-sfære er en hypotetisk megakonstruktion, der fuldstændig omgiver en stjerne og fanger en stor procentdel af dens energiudgang.~\cite{wiki:dyson}} omkring vores sol og lade den køre i 100 billioner billioner år, ville vi stadig kun have en $25\%$ chance for at finde en nål i en 256-bit høstak.

\begin{quotation}\begin{samepage}
  \enquote{Disse tal har intet at gøre med teknologien i enhederne;
  de er maksimum, som termodynamik tillader. Og
  de antyder kraftigt, at brute-force angreb mod 256-bit nøgler vil være
  ufejlbare, indtil computere er bygget af noget andet end stof
  og besætter noget andet end rum.}
  \begin{flushright} -- Bruce Schneier\footnote{Bruce Schneier, \textit{Applied Cryptography} \cite{bruce-schneier}}
  \end{flushright}\end{samepage}\end{quotation}
  
  
  Det er svært at overvurdere dybden af dette. Stærk kryptografi
  vender magtbalancen af den fysiske verden, vi er så vant til.
  Uopløselige ting eksisterer ikke i den virkelige verden. Anvend tilstrækkelig kraft,
  og du vil kunne åbne enhver dør, kasse eller skatkiste.
  
  Bitcoin's skatkiste er meget anderledes. Den er sikret af stærk
  kryptografi, der ikke giver efter for brute force. Og så længe de
  underliggende matematiske antagelser holder, er brute force alt, hvad vi har.
  Selvfølgelig er der også muligheden for et globalt \$5 skruenøgleangreb (Figur~\ref{fig:xkcd-538}).
  Men tortur vil ikke fungere for alle bitcoin-adresser, og bitcoins kryptografiske
  mure vil besejre brute force-angreb. Selv hvis du kommer med kraften fra tusind soler. Bogstaveligt talt.

\begin{center}
  \centering
  \includegraphics[width=8cm]{assets/images/xkcd-538.png}
  \captionof{figure}{\$5 skruenøgleangreb. Kilde: xkcd 538}
  \label{fig:xkcd-538}
\end{center}

Denne kendsgerning og dens implikationer blev præcist opsummeret i opfordringen
til kryptografisk forsvar: \textit{\enquote{Intet beløb af tvang vil nogensinde løse
en matematisk opgave.}}

\begin{quotation}\begin{samepage}
\enquote{Det er ikke åbenlyst, at verden skulle fungere på denne måde. Men på en eller anden måde smiler universet til kryptering.}
\begin{flushright} -- Julian Assange\footnote{Julian Assange, \textit{A Call to Cryptographic Arms} \cite{call-to-cryptographic-arms}}
\end{flushright}\end{samepage}\end{quotation}

Ingen ved endnu med sikkerhed, om universets smil er ægte eller ej. Det
er muligt, at vores antagelse om matematiske asymmetrier er forkert, og
vi finder ud af, at P faktisk er lig med NP \cite{wiki:pnp}, eller vi finder overraskende hurtige
løsninger på specifikke problemer \cite{wiki:discrete-log}, som vi i øjeblikket antager er svære.
Hvis det skulle være tilfældet, vil kryptografi, som vi kender det, ophøre med at
eksistere, og implikationerne ville sandsynligvis ændre verden ud over
genkendelse.

\begin{quotation}\begin{samepage}
\enquote{Vires in Numeris} = \enquote{Styrke i Tal}\footnote{\textit{Vires in Numeris} blev først foreslået som en Bitcoin-motto af bitcointalk-brugeren \textit{epii}~\cite{epii}}
\end{samepage}\end{quotation}

\textit{Vires in numeris} er ikke kun en fængende motto brugt af bitcoin-entusiaster. Erkendelsen af, at der er en uudgrundelig styrke at finde i tal, er en dybdegående erkendelse. At forstå dette og den omvending af eksisterende magtbalance, det muliggør, har ændret mit syn på verden og den fremtid, der venter os.

Ét direkte resultat af dette er, at du ikke behøver at spørge nogen om tilladelse for at deltage i Bitcoin. Der er ingen side at tilmelde sig, ingen virksomhed ansvarlig, ingen regeringsinstans at sende ansøgningsformularer til. Bare generer et stort tal, og du er stort set klar til at gå i gang. Den centrale myndighed for kontoskabelse er matematik. Og kun Gud ved, hvem der har kontrol over det.

\begin{center}
  \includegraphics[width=\textwidth]{assets/images/elliptic-curve-examples.png}
  \captionof{figure}{Eksempler på elliptiske kurver. Grafik cc-by-sa Emmanuel Boutet.}
  \label{fig:elliptic-curve-examples}
\end{center}

Bitcoin er bygget på vores bedste forståelse af virkeligheden. Selvom der stadig er mange åbne problemer inden for fysik, datalogi og matematik, er vi ret sikre på nogle ting. At der er en asymmetri mellem at finde løsninger og validere korrektheden af disse løsninger er en sådan ting. At beregning kræver energi er en anden. Med andre ord: at finde en nål i en høstak er sværere end at tjekke, om det spidse objekt i din hånd faktisk er en nål eller ej. Og at finde nålen kræver arbejde.

Uendeligheden af ​​Bitcoins adresseområde er virkelig overvældende. Antallet af private nøgler endnu mere. Det er fascinerende, hvor meget af vores moderne verden reduceres til sandsynligheden for at finde en nål i en uudgrundeligt stor høstak. Jeg er nu mere opmærksom på denne kendsgerning end nogensinde.

\paragraph{Bitcoin lærte mig, at der er styrke i tal.}

% ---
%
% #### Down the Rabbit Hole
%
% - [How secure is 256 bit security?]["How secure is 256 bit security?"] by 3Blue1Brown
% - [Block Hashing Algorithm][hash functions] on the Bitcoin Wiki
% - [Last Glacial Maximum][thick layer of ice], [SHA-2][SHA-256], [Dyson Sphere][Dyson sphere], [Landauer's Principle][flip bits perfectly] [P versus NP][P actually equals NP], [Discrete Logarithm][specific problems] on Wikipedia
%
% [thick layer of ice]: https://en.wikipedia.org/wiki/Last_Glacial_Maximum
% [xkcd \#1125]: https://xkcd.com/1225/
% [SHA-256]: https://en.wikipedia.org/wiki/SHA-2
% [hash functions]: https://en.bitcoin.it/wiki/Block_hashing_algorithm
% ["How secure is 256 bit security?"]: https://www.youtube.com/watch?v=S9JGmA5_unY
% [Bruce Schneier]: https://www.schneier.com/
% [flip bits perfectly]: https://en.wikipedia.org/wiki/Landauer%27s_principle#Equation
% [Dyson sphere]: https://en.wikipedia.org/wiki/Dyson_sphere
% [2]: https://books.google.com/books?id=Ok0nDwAAQBAJ&pg=PT316&dq=%22These+numbers+have+nothing+to+do+with+the+technology+of+the+devices;%22&hl=en&sa=X&ved=0ahUKEwjXttWl8YLhAhUphOAKHZZOCcsQ6AEIKjAA#v=onepage&q&f=false
% [wrench attack]: https://xkcd.com/538/
% [call to cryptographic arms]: https://cryptome.org/2012/12/assange-crypto-arms.htm
% [P actually equals NP]: https://en.wikipedia.org/wiki/P_versus_NP_problem#P_=_NP
% [specific problems]: https://en.wikipedia.org/wiki/Discrete_logarithm#Cryptography
% [3Blue1Brown]: https://twitter.com/3blue1brown
%
% <!-- Wikipedia -->
% [alice]: https://en.wikipedia.org/wiki/Alice%27s_Adventures_in_Wonderland
% [carroll]: https://en.wikipedia.org/wiki/Lewis_Carroll

\chapter{Refleksioner om \enquote{Don't Trust, Verify}}
\label{les:16}

\begin{chapquote}{Lewis Carroll, \textit{Alice i Eventyrland}}
\enquote{Nu til beviset,} sagde Kongen, \enquote{og så dommen.}
\end{chapquote}

Bitcoin sigter mod at erstatte, eller i det mindste give et alternativ til,
konventionel valuta. Konventionel valuta er bundet til en centraliseret
myndighed, uanset om vi taler om lovligt betalingsmiddel som den amerikanske
dollar eller moderne monopolpenge som Fortnite's V-Bucks. I begge
eksempler er du bundet til at stole på den centrale myndighed for at udstede, 
administrere og cirkulere dine penge. Bitcoin løsner denne binding, og 
hovedproblemet som Bitcoin løser, er tillidsproblemet.

\begin{quotation}\begin{samepage}
\enquote{Det grundlæggende problem med konventionel valuta er al den tillid, 
der er nødvendig for at få det til at fungere. [...] Hvad der er nødvendigt, 
er et elektronisk betalingssystem baseret på kryptografisk bevis i stedet for 
tillid.} \begin{flushright} -- Satoshi Nakamoto\footnote{Satoshi Nakamoto,
  officiel Bitcoin-annoncering~\cite{bitcoin-announcement} og 
  whitepaper~\cite{whitepaper}}
\end{flushright}\end{samepage}\end{quotation}

Bitcoin løser tillidsproblemet ved at være fuldstændig decentraliseret,
uden central server eller betroede parter. Ikke engang betroede \textit{tredje}
parter, men betroede parter, punktum. Når der ikke er nogen central
myndighed, er der ganske enkelt ingen at stole på. Total decentralisering
er innovationen. Det er roden til Bitcoin's modstandsdygtighed, årsagen
til at det stadig er i live. Decentralisering er også grunden til, at vi har 
mining, noder, hardwarewallets, og ja, blockchain. Det eneste, du
skal \enquote{stole på}, er, at vores forståelse af matematik og fysik
ikke er helt ude af kurs, og at flertallet af minearbejdere handler ærligt 
(hvad de har incitament til at gøre).

Mens den almindelige verden opererer under antagelsen om \textit{\enquote{stol
på, men verificér,}} opererer Bitcoin under antagelsen om \textit{\enquote{stol
ikke på, verificér.}} Satoshi gjorde vigtigheden af at fjerne tillid meget 
tydelig både i introduktionen og konklusionen af Bitcoin whitepaper.

\begin{quotation}\begin{samepage}
\enquote{Konklusion: Vi har foreslået et system til elektroniske transaktioner
uden at stole på tillid.}
\begin{flushright} -- Satoshi Nakamoto\footnote{Satoshi Nakamoto, Bitcoin 
  whitepaper~\cite{whitepaper}}
\end{flushright}\end{samepage}\end{quotation}

Bemærk, at \textit{uden at stole på tillid} bruges i en meget specifik kontekst
her. Vi taler om betroede tredjeparter, dvs. andre enheder,
som du stoler på at producere, opbevare og behandle dine penge. Det antage 
f.eks., at du kan stole på din computer.

Som Ken Thompson viste i sin Turing Award-forelæsning, er tillid en
ekstremt vanskelig ting i den beregningsmæssige verden. Når du kører et
program, er du nødt til at stole på al slags software (og hardware), som
i teorien kunne ændre programmet, du forsøger at køre, på en ondsindet
måde. Som Thompson opsummerede i sin \textit{Refleksioner om at stole på 
tillid}:
\enquote{Moralen er åbenlys. Du kan ikke stole på kode, som du ikke har skabt
helt selv.}~\cite{trusting-trust}

\begin{figure}[htbp]
  \centering
  \includegraphics[width=\textwidth]{assets/images/ken-thompson-hack.png}
  \caption{Uddrag fra Ken Thompsons papir 'Refleksioner om at stole på tillid'}
  \label{fig:ken-thompson-hack}
\end{figure}

Thompson viste, at selv hvis du har adgang til kildekoden,
kan din kompilator --- eller enhver anden programbehandlings- eller
hardwareprogram --- være kompromitteret, og det ville være
meget vanskeligt at opdage denne bagdør. Således eksisterer der i praksis ikke
et virkelig \textit{trustless} system. Du ville være nødt til at skabe alt 
din software \textit{og} alt din hardware (assemblers, kompilatorer, linkers,
osv.) fra bunden, uden hjælp fra nogen ekstern software eller 
software-understøttet maskineri.

\begin{quotation}\begin{samepage}
\enquote{Hvis du ønsker at lave en æbletærte fra bunden, skal du først opfinde
universet.}
\begin{flushright} -- Carl Sagan\footnote{Carl Sagan, \textit{Cosmos} \cite{cosmos}}
\end{flushright}\end{samepage}\end{quotation}

Ken Thompson Hack er en særligt genial og svær at opdage bagdør,
så lad os hurtigt se på en svær at opdage bagdør, der fungerer uden at
modificere nogen software. Forskere fandt en måde at kompromittere 
sikkerhedskritisk hardware ved at ændre polariteten af 
siliciumforureninger.~\cite{becker2013stealthy} Ved blot at ændre de fysiske 
egenskaber af det materiale, som computerchips er lavet af, lykkedes det dem 
at kompromittere en kryptografisk sikker tilfældig talgenerator. Da denne
ændring ikke kan ses, kan bagdøren ikke opdages ved optisk inspektion, 
hvilket er en af de vigtigste metoder til at opdage manipulation af chips 
som disse.

\begin{figure}[htbp]
  \centering
  \includegraphics[width=\textwidth]{assets/images/stealthy-hardware-trojan.png}
  \caption{Stealthy Dopant-Level Hardware Trojans af Becker, Regazzoni, Paar, 
  Burleson}
  \label{fig:stealthy-hardware-trojan}
\end{figure}

Lyder det skræmmende? Nå, selv hvis du var i stand til at bygge alt fra
bunden, ville du stadig være nødt til at stole på den underliggende matematik. 
Du ville være nødt til at stole på, at \textit{secp256k1} er en elliptisk kurve 
uden bagdøre. Ja, ondsindede bagdøre kan indsættes i de matematiske
grundlag af kryptografiske funktioner, og man kan argumentere for, at dette 
allerede er sket mindst én gang.~\cite{wiki:Dual_EC_DRBG} Der er gode grunde 
til at være paranoid, og det faktum, at alt lige fra din hardware, til din 
software, til de elliptiske kurver, der bruges, kan have 
bagdøre~\cite{wiki:backdoors}, er nogle af dem.

\begin{quotation}\begin{samepage}
  \enquote{Don't trust. Verify.}
  \begin{flushright} -- Bitcoiners everywhere
\end{flushright}\end{samepage}\end{quotation}

De ovenstående eksempler burde illustrere, at \textit{trustless} computing er
utopisk. Bitcoin er sandsynligvis det system, der kommer tættest på denne
utopi, men det er stadig \textit{trust-minimized} --- med det formål at fjerne 
tillid hvor det er muligt. Man kan argumentere for, at kæden af tillid aldrig 
ender, da du også skal stole på, at beregning kræver energi, at P ikke er 
lig med NP, og at du rent faktisk er i virkeligheden og ikke
fanget i en simulering af ondsindede aktører.

Udviklere arbejder på værktøjer og procedurer for at minimere al resterende 
tillid endnu mere. For eksempel skabte Bitcoin-udviklere
Gitian\footnote{\url{https://gitian.org/}}, som er en metode til 
softwaredistribution for at skabe deterministiske builds. Ideen er, at hvis 
flere udviklere er i stand til at reproducere identiske binære filer, reduceres 
risikoen for ondsindet manipulation. Fancy bagdøre er ikke den eneste 
angrebsvektor. Enkle trusler om afpresning er også reelle trusler. Ligesom i 
hovedprotokollen bruges decentralisering til at minimere tillid.

Der gøres forskellige bestræbelser for at forbedre på hønen-og-ægget-problemet 
ved bootstrapping, som Ken Thompsons hack så brillant 
påpegede~\cite{web:bootstrapping}. En sådan indsats er Guix
\footnote{\url{https://guix.gnu.org}} (udtales \textit{geeks}), som bruger 
funktionelt deklareret pakkehåndtering, hvilket fører til bit-for-bit
reproducerbare builds efter design. Resultatet er, at du ikke længere 
behøver at stole på nogen software-leverende servere, da du kan verificere, at 
den serverede binære fil ikke er blevet manipuleret ved at genopbygge den fra 
bunden. For nylig blev der flet en pull-anmodning for at integrere Guix i 
Bitcoin-buildprocessen.\footnote{Se PR 15277 af 
\texttt{bitcoin-core}: \\ \url{https://github.com/bitcoin/bitcoin/pull/15277}}

\begin{figure}[htbp]
  \centering
  \includegraphics[width=\textwidth]
  {assets/images/guix-bootstrap-dependencies.png}
  \caption{Hvad kom først, hønen eller ægget?}
  \label{fig:guix-bootstrap-dependencies}
\end{figure}

heldigvis er Bitcoin ikke afhængig af en enkelt algoritme eller stykke
hardware. En effekt af Bitcoins radikale decentralisering er en
distribueret sikkerhedsmodel. Selvom bagdøre beskrevet ovenfor ikke skal 
tages let, er det usandsynligt, at hver softwarewallet,
hver hardwarewallet, hver kryptografiske bibliotek, hver noderealisering
og hver kompilator af hvert sprog er kompromitteret.
Muligt, men meget usandsynligt.

Bemærk, at du kan generere en privat nøgle uden at stole på nogen 
beregningsmæssig hardware eller software. Du kan kaste en 
mønt~\cite{antonopoulos2014mastering} et par gange, selvom afhængigt af din 
mønt og kastestil kan denne kilde af tilfældighed måske ikke være 
tilstrækkelig tilfældig. Der er en grund til, at lagringsprotokoller som 
Glacier\footnote{\url{https://glacierprotocol.org/}} råder til at bruge 
terninger af kasino-kvalitet som en af to kilder til entropi.

Bitcoin tvang mig til at reflektere over, hvad det egentlig indebærer ikke at 
stole på nogen. Det øgede min bevidsthed om bootstrapping-problemet og den 
implicitte kæde af tillid ved udvikling og kørsel af software. Det øgede også 
min bevidsthed om de mange måder, hvorpå software og hardware kan blive
kompromitteret.

\paragraph{Bitcoin lærte mig ikke at stole, men at verificere.}

% ---
%
% #### Down the Rabbit Hole
%
% - [The Bitcoin whitepaper][Nakamoto] by Satoshi Nakamoto
% - [Reflections on Trusting Trust][\textit{Reflections on Trusting Trust}] by Ken Thompson
% - [51% Attack][majority] on the Bitcoin Developer Guide
% - [Bootstrapping][bootstrapping], Guix Manual
% - [Secp256k1][secp256k1] on the Bitcoin Wiki
% - [ECC Backdoors][backdoors], [Dual EC DRBG][has already happened] on Wikipedia
%
% [Emmanuel Boutet]: https://commons.wikimedia.org/wiki/User:Emmanuel.boutet
% [\textit{Reflections on Trusting Trust}]: https://www.archive.ece.cmu.edu/~ganger/712.fall02/papers/p761-thompson.pdf
% [found a way]: https://scholar.google.com/scholar?hl=en&as_sdt=0%2C5&q=Stealthy+Dopant-Level+Hardware+Trojans&btnG=
% [Gitian]: https://gitian.org/
% [bootstrapping]: https://www.gnu.org/software/guix/manual/en/html_node/Bootstrapping.html
% [Guix]: https://www.gnu.org/software/guix/
% [pull-request]: https://github.com/bitcoin/bitcoin/pull/15277
% [flip a coin]: https://github.com/bitcoinbook/bitcoinbook/blob/develop/ch04.asciidoc#private-keys
% [Glacier]: https://glacierprotocol.org/
% [secp256k1]: https://en.bitcoin.it/wiki/Secp256k1
% [majority]: https://bitcoin.org/en/developer-guide#term-51-attack
%
% <!-- Wikipedia -->
% [backdoors]: https://en.wikipedia.org/wiki/Elliptic-curve_cryptography#Backdoors
% [has already happened]: https://en.wikipedia.org/wiki/Dual_EC_DRBG
% [Carl Sagan]: https://en.wikipedia.org/wiki/Cosmos_%28Carl_Sagan_book%29
% [alice]: https://en.wikipedia.org/wiki/Alice%27s_Adventures_in_Wonderland
% [carroll]: https://en.wikipedia.org/wiki/Lewis_Carroll

\chapter{At Fortælle Tid Kræver Arbejde}
\label{les:17}

\begin{chapquote}{Lewis Carroll, \textit{Alice i Eventyrland}}
\enquote{Kære, kære! Jeg kommer for sent!}
\end{chapquote}

Det bliver ofte sagt, at bitcoins udvindes, fordi tusinder af computere arbejder på at løse \textit{meget komplekse} matematiske problemer. Visse problemer skal løses, og hvis du beregner det rigtige svar, \enquote{producerer} du en bitcoin. Mens denne forenklede opfattelse af bitcoin-minedrift måske er lettere at formidle, går den lidt forbi pointen. Bitcoins bliver ikke produceret eller skabt, og hele affæren handler ikke rigtig om at løse bestemte matematiske problemer. Desuden er matematikken ikke særlig kompleks. Det komplekse er at \textit{fortælle tiden} i et decentraliseret system.

Som beskrevet i whitepaperet er proof-of-work-systemet (også kaldet minedrift) en måde at implementere en distribueret tidsstempelserver.

\begin{center}
  \includegraphics[width=\textwidth]{assets/images/bitcoin-whitepaper-timestamp-wide.png}
  \captionof{figure}{Uddrag fra whitepaperet. Sagde nogen tidskæde?}
  \label{fig:bitcoin-whitepaper-timestamp-wide}
\end{center}

Da jeg først lærte, hvordan Bitcoin fungerer, troede jeg også, at proof-of-work er ineffektivt og spild af ressourcer. Efter et stykke tid begyndte jeg dog at ændre min opfattelse af Bitcoin's energiforbrug~\cite{gigi:energy}. Det ser ud til, at proof-of-work stadig er bredt misforstået i dag, i året 10 AB (efter Bitcoin).

Da problemerne, der skal løses i proof-of-work, er opfundne, synes mange mennesker at tro, at det er \textit{unyttigt} arbejde. Hvis fokus udelukkende er på beregningen, er dette en forståelig konklusion. Men Bitcoin handler ikke om beregning. Det handler om \textit{uafhængigt at blive enige om rækkefølgen af tingene.}

Proof-of-work er et system, hvor alle kan validere, hvad der skete, og i hvilken rækkefølge det skete. Denne uafhængige validering er det, der fører til konsensus, en individuel enighed mellem flere parter om, hvem der ejer hvad.

I et radikalt decentraliseret miljø har vi ikke luksusen af absolut tid. Enhver ur ville introducere en betroet tredjepart, et centralt punkt i systemet, som man skulle stole på og kunne angribes. \enquote{Tid er rodproblemet,} som Grisha Trubetskoy påpeger~\cite{pow-clock}. Og Satoshi løste genialt dette problem ved at implementere et decentraliseret ur via en proof-of-work blockchain. Alle er enige på forhånd om, at kæden med det største kumulative arbejde er sandhedens kilde. Det er per definition, hvad der faktisk skete. Denne enighed er det, der nu er kendt som Nakamoto-konsensus.

\begin{quotation}\begin{samepage}
  \enquote{Netværket tidsstempler transaktioner ved at hashe dem ind i en løbende
  kæde, som fungerer som bevis på sekvensen af begivenheder, der er vidne til.}
  \begin{flushright} -- Satoshi Nakamoto\footnote{Satoshi Nakamoto, Bitcoin-whitepaperet~\cite{whitepaper}}
\end{flushright}\end{samepage}\end{quotation}
  
Uden en sammenhængende måde at fortælle tiden på er der ingen konsekvent måde at
skelne før fra efter. Pålidelig ordning er umulig. Som nævnt
ovenfor er Nakamoto-konsensus Bitcoins måde at konsekvent fortælle tiden på. Systemets incitamentsstruktur producerer et sandsynligt,
decentraliseret ur, ved at udnytte både grådighed og egeninteresse hos
konkurrerende deltagere. Det faktum, at dette ur er upræcist, er
ligegyldigt, fordi begivenhedernes rækkefølge er efterhånden entydig og kan
verificeres af alle.

Takket være proof-of-work er både arbejdet \textit{og} valideringen af arbejdet
radikalt decentraliseret. Alle kan deltage og forlade efter ønske, og
alle kan validere alt til enhver tid. Ikke kun det, men
alle kan validere systemets tilstand \textit{individuelt}, uden
at skulle stole på nogen andre for validering.

At forstå proof-of-work tager tid. Det er ofte modintuitivt,
og selvom reglerne er simple, fører de til ret komplekse fænomener.
For mig hjalp det at ændre min opfattelse af minedrift. Nyttig, ikke unyttig.
Validering, ikke beregning. Tid, ikke blokke.

\paragraph{Bitcoin lærte mig, at at fortælle tiden er svært, især hvis du er
decentraliseret.}

% ---
%
% #### Through the Looking-Glass
%
% - [Bitcoin's Energy Consumption: A shift in perspective][energy]
%
% #### Down the Rabbit Hole
%
% - [Blockchain Proof-of-Work Is a Decentralized Clock][points out] by Gregory Trubetskoy
% - [The Anatomy of Proof-of-Work][pow-anatomy] by Hugo Nguyen
% - [PoW is efficient][pow-efficient] by Dan Held
% - [Mining][bw-mining], [Controlled supply][bw-supply] on the Bitcoin Wiki
%
% [points out]: https://grisha.org/blog/2018/01/23/explaining-proof-of-work/
% [energy]: 
% [whitepaper]: https://bitcoin.org/bitcoin.pdf
%
% [pow-efficient]: https://blog.picks.co/pow-is-efficient-aa3d442754d3
% [pow-anatomy]: https://bitcointechtalk.com/the-anatomy-of-proof-of-work-98c85b6f6667
% [bw-mining]: https://en.bitcoin.it/wiki/Mining
% [bw-supply]: https://en.bitcoin.it/wiki/Controlled_supply
%
% <!-- Wikipedia -->
% [alice]: https://en.wikipedia.org/wiki/Alice%27s_Adventures_in_Wonderland
% [carroll]: https://en.wikipedia.org/wiki/Lewis_Carroll

\chapter{Bevæg Dig Langsomt og Ødelæg Ikke Ting}
\label{les:18}

\begin{chapquote}{Lewis Carroll, \textit{Alice i Eventyrland}}
Så båden snoede sig langsomt afsted, under den klare sommerdag, med dens muntre 
besætning og musik af stemmer og latter\ldots
\end{chapquote}

Det kan være en død mantra, men \enquote{bevæg dig hurtigt og ødelæg ting} er 
stadig, hvordan stor del af tech-verdenen opererer. Ideen om, at det ikke
betyder noget, hvis du får tingene rigtige første gang, er en grundlæggende 
søjle i \textit{fejl tidligt, fejl ofte} mentaliteten. Succes måles i vækst, 
så længe du vokser, er alt fint. Hvis noget ikke virker første gang, så
drejer og itererer du bare. Med andre ord: kast nok ting mod
væggen og se, hvad der bliver hængende.

Bitcoin er meget anderledes. Den er anderledes af design. Den er anderledes
af nødvendighed. Som Satoshi påpegede, er e-valuta blevet forsøgt
mange gange før, og alle tidligere forsøg er mislykkedes, fordi der
var et hoved, der kunne skæres af. Nyheden ved Bitcoin er, at det er
et uhyre uden hoveder.

\begin{quotation}\begin{samepage}
\enquote{Mange mennesker afskriver automatisk e-valuta som en tabt sag
på grund af alle virksomhederne, der er fejlet siden 1990'erne. Jeg håber, det 
er åbenlyst, at det kun var den centraliserede karakter af de systemer
der fordømte dem.}
\begin{flushright} -- Satoshi Nakamoto\footnote{Satoshi Nakamoto, i et svar 
    til Sepp Hasslberger \cite{satoshi-centralized-nature}}
\end{flushright}\end{samepage}\end{quotation}

En konsekvens af denne radikale decentralisering er en indbygget modstand 
mod forandring. \enquote{Bevæg dig hurtigt og ødelæg ting} fungerer ikke 
og vil aldrig fungere på Bitcoin's basale lag. Selvom det ville være 
ønskeligt, ville det ikke være muligt uden at overbevise \textit{alle} 
om at ændre deres måder. Det er distribueret konsensus. Det er naturen 
af Bitcoin.

\begin{quotation}\begin{samepage}
\enquote{Bitcoin's natur er sådan, at når version 0.1 blev frigivet, var
kerne designet hugget i sten for resten af dets levetid.}
\begin{flushright} -- Satoshi Nakamoto\footnote{Satoshi Nakamoto, i et svar 
    til Gavin Andresen \cite{satoshi-centralized-nature}}
\end{flushright}\end{samepage}\end{quotation}

Dette er en af de mange paradoksale egenskaber ved Bitcoin. Vi kom alle
til at tro, at alt, der er software, kan ændres let. Men
uhyrets natur gør det vanskeligt at ændre det.

Som Hasu smukt viser i Unpacking Bitcoin's Social
Contract~\cite{social-contract}, er det kun muligt at ændre Bitcoin's regler
ved at \textit{forslå} en ændring og dermed \textit{overtale} alle brugere
af Bitcoin til at acceptere denne ændring. Dette gør Bitcoin meget 
modstandsdygtig over for forandring,
selvom det er software.

Denne modstandsdygtighed er en af de vigtigste egenskaber ved Bitcoin.
Kritiske software-systemer skal være antifragile, hvilket er det
samspil, Bitcoin's sociale lag og dets tekniske lag garanterer.
Pengesystemer er af natur fjendtlige, og som vi har vidst i
tusinder af år, er solide grundlag afgørende i en fjendtlig
miljø.

\begin{quotation}\begin{samepage}
    \enquote{Regnen faldt, oversvømmelserne kom, og vinden blæste og slog på
    det hus; og det faldt ikke, for det var bygget på klippen.}
    \begin{flushright} -- Matthæus 7:24--27
\end{flushright}\end{samepage}\end{quotation}

Argumentabelt set er Bitcoin i denne lignelse om de kloge og tåbelige byggere
ikke huset. Det er klippen. Uforanderlig, uhældig, der giver grundlaget for et 
nyt økonomisk system.

Præcis som geologer, der ved, at klippeformationer altid bevæger sig
og udvikler sig, kan man se, at Bitcoin altid bevæger sig og udvikler sig også.
Du skal bare vide, hvor du skal kigge, og hvordan du skal se på det.

Introduktionen af pay to script hash\footnote{ Pay to script hash (P2SH)
transaktioner blev standardiseret i BIP 16. De tillader, at transaktioner 
sendes til en script hash (adresse der starter med 3) i stedet for en public 
key hash (adresser der starter med 1).~\cite{btcwiki:p2sh}} og segregate
witness\footnote{Segregated Witness (forkortet som SegWit) er en implementeret
protokolopgradering, der har til formål at beskytte mod 
transaktionsmalleabilitet og øge blokkapaciteten. SegWit adskiller 
\textit{vidnet} fra listen af inputs.~\cite{btcwiki:segwit}} er bevis på,
at Bitcoin's regler kan ændres, hvis tilstrækkeligt mange brugere er 
overbeviste om, at vedtagelse af denne ændring er til fordel for netværket. 
Den sidstnævnte muliggjorde udviklingen af lynnetværket 
\footnote{\url{https://lightning.network/}}, som er et af husene, der bliver 
bygget på Bitcoin's solide fundament. Fremtidige opgraderinger som Schnorr
signaturer~\cite{bip:schnorr} vil forbedre effektiviteten og privatlivet, 
såvel som scripts (læs: smarte kontrakter), der vil være uundskillelige fra 
almindelige transaktioner takket være Taproot~\cite{taproot}. Kloge byggere 
bygger virkelig på solide grundlag.

Satoshi var ikke kun en klog bygger teknologisk set. Han forstod også, at det 
ville være nødvendigt at træffe kloge beslutninger ideologisk set.

\begin{quotation}\begin{samepage}
    \enquote{At være åben kilde betyder, at enhver uafhængigt kan gennemgå 
    koden. Hvis den var lukket kilde, ville ingen kunne verificere 
    sikkerheden. Jeg synes, det er essentielt for et program af denne art at 
    være åben kilde.}
    \begin{flushright} -- Satoshi Nakamoto\footnote{Satoshi Nakamoto, i et 
        svar til SmokeTooMuch \cite{satoshi-open-source}}
\end{flushright}\end{samepage}\end{quotation}

Åbenhed er afgørende for sikkerhed og indlejret i åben kilde og
free software-bevægelsen. Som Satoshi påpegede, skal sikre protokoller og
den kode, der implementerer dem, være åbne --- der er ingen sikkerhed
gennem obskuritet. En anden fordel er igen relateret til decentralisering:
kode, der frit kan køres, studeres, ændres, kopieres og distribueres
sikrer, at den spredes vidt og bredt.

Bitcoin's radikalt decentraliserede karakter er det, der får det til at 
bevæge sig langsomt og med overvejelse. Et netværk af noder, hver kørt af 
en suveræn individ, er i sig selv modstandsdygtig over for ændringer --- 
ondsindede eller ej. Uden mulighed for at tvinge opdateringer på brugerne er 
den eneste måde at introducere ændringer på ved langsomt at overbevise hver 
eneste af disse individer om at acceptere en ændring. Denne ikke-centrale 
proces med at introducere og implementere ændringer er det, der gør nettet 
utroligt modstandsdygtigt over for ondsindede ændringer. Det er også det, 
der gør det sværere at rette ødelagte ting end i en centraliseret miljø, 
hvilket er grunden til, at alle prøver ikke at ødelægge ting i første omgang.

\paragraph{Bitcoin lærte mig, at bevæge sig langsomt er en af dets 
funktioner, ikke en fejl.}

% ---
%
% #### Through the Looking-Glass
%
% - [Lesson 1: Immutability and Change][lesson1]
%
% #### Down the Rabbit Hole
%
% - [Unpacking Bitcoin's Social Contract] by Hasu
% - [Schnorr signatures BIP][Schnorr signatures] by Pieter Wuille
% - [Taproot proposal][Taproot] by Gregory Maxwell
% - [P2SH][pay to script hash], [SegWit][segregated witness] on the Bitcoin Wiki
% - [Parable of the Wise and the Foolish Builders][Matthew 7:24--27] on Wikipedia
%
% <!-- Down the Rabbit Hole -->
% [lesson1]: {{ '/bitcoin/lessons/ch1-01-immutability-and-change' | absolute_url }}
%
% [Unpacking Bitcoin's Social Contract]: https://uncommoncore.co/unpacking-bitcoins-social-contract/
% [Matthew 7:24--27]: https://en.wikipedia.org/wiki/Parable_of_the_Wise_and_the_Foolish_Builders
% [pay to script hash]: https://en.bitcoin.it/wiki/Pay_to_script_hash
% [segregated witness]: https://en.bitcoin.it/wiki/Segregated_Witness
% [lightning network]: https://lightning.network/
% [Schnorr signatures]: https://github.com/sipa/bips/blob/bip-schnorr/bip-schnorr.mediawiki#cite_ref-6-0
% [Taproot]: https://lists.linuxfoundation.org/pipermail/bitcoin-dev/2018-January/015614.html
%
% <!-- Wikipedia -->
% [alice]: https://en.wikipedia.org/wiki/Alice%27s_Adventures_in_Wonderland
% [carroll]: https://en.wikipedia.org/wiki/Lewis_Carroll

\chapter{Privatliv er Ikke Dødt}
\label{les:19}

\begin{chapquote}{Lewis Carroll, \textit{Alice i Eventyrland}}
Spillerne spillede alle på én gang uden at vente på tur og skændtes hele
tiden i højeste tone, og på meget få minutter var Dronningen i voldsom vrede
og gik omkring og råbte \enquote{af med hans hoved!} eller \enquote{af med hendes
hoved!} cirka en gang i minuttet.
\end{chapquote}

Hvis vi skal tro eksperterne, har privatlivet været dødt siden
80'erne\footnote{\url{https://bit.ly/privacy-is-dead}}. Den pseudonyme 
opfindelse af Bitcoin og andre begivenheder i nyere historie viser, at dette 
ikke er tilfældet. Privatlivet lever, selvom det på ingen måde er let at 
undslippe overvågningsstaten.

Satoshi gik langt for at skjule sine spor og skjule
sin identitet. Ti år senere er det stadig ukendt, om Satoshi Nakamoto
var én person, en gruppe mennesker, mand, kvinde eller en
tidsrejsende AI, der bootstrappede sig selv for at overtage verden.
Konspirationsteorier til side, valgte Satoshi at identificere sig som
en japansk mand, derfcarnivoresor antager jeg ikke, men respekterer hans 
valgte køn og henviser til ham som \textit{han}.

\begin{figure}[htbp]
  \centering
  \includegraphics[width=\textwidth]{assets/images/nope.png}
  \caption{Jeg er ikke Dorian Nakamoto.}
  \label{fig:nope}
\end{figure}

Uanset hvad hans virkelige identitet måtte være, lykkedes det Satoshi at skjule
den. Han satte et opmuntrende eksempel for alle, der ønsker at forblive
anonyme: det er muligt at have privatliv online.

\begin{quotation}\begin{samepage}
\enquote{Kryptering virker. Ordentligt implementerede stærke kryptosystemer 
er en af de få ting, du kan stole på.}
\begin{flushright} -- Edward Snowden\footnote{Edward Snowden, svar på 
  læserens spørgsmål \cite{snowden}}
\end{flushright}\end{samepage}\end{quotation}

Satoshi var ikke den første pseudonyme eller anonyme opfinder, og han vil 
ikke være den sidste. Nogle har direkte efterlignet denne pseudonyme 
publiceringsstil, som Tom Elvis Yedusor fra 
MimbleWimble~\cite{mimblewimble-origin} berømmelse, mens andre har
offentliggjort avancerede matematiske beviser og forblevet helt
anonyme~\cite{4chan-math}.

Det er en mærkelig ny verden, vi lever i. En verden, hvor identitet er
valgfri, bidrag accepteres baseret på fortjeneste, og folk kan
samarbejde og handle frit. Det vil tage lidt tilvænning at blive
komfortabel med disse nye paradigmer, men jeg tror stærkt på, at alt dette 
har potentiale til at ændre verden til det bedre.

Vi bør alle huske på, at privatliv er en grundlæggende menneskeret
\footnote{Universal Declaration of Human Rights, 
\textit{Artikel 12}.~\cite{article12}}. Og så længe
mennesker udøver og forsvarer disse rettigheder, er kampen for privatlivet 
langt fra ovre.

\paragraph{Bitcoin lærte mig, at privatlivet ikke er dødt.}

% ---
%
% #### Down the Rabbit Hole
%
% - [Universal Declaration of Human Rights][fundamental human right] by the United Nations
% - [A lower bound on the length of the shortest superpattern][anonymous] by Anonymous 4chan Poster, Robin Houston, Jay Pantone, and Vince Vatter
%
% [since the 80ies]: https://books.google.com/ngrams/graph?content=privacy+is+dead&year_start=1970&year_end=2019&corpus=15&smoothing=3&share=&direct_url=t1%3B%2Cprivacy%20is%20dead%3B%2Cc0
% [time-traveling AI]: https://blockchain24-7.com/is-crypto-creator-a-time-travelling-ai/
% ["I am not Dorian Nakamoto."]: http://p2pfoundation.ning.com/forum/topics/bitcoin-open-source?commentId=2003008%3AComment%3A52186
% [MimbleWimble]: https://github.com/mimblewimble/docs/wiki/MimbleWimble-Origin
% [anonymous]: https://oeis.org/A180632/a180632.pdf
% [fundamental human right]: http://www.un.org/en/universal-declaration-human-rights/
%
% <!-- Wikipedia -->
% [alice]: https://en.wikipedia.org/wiki/Alice%27s_Adventures_in_Wonderland
% [carroll]: https://en.wikipedia.org/wiki/Lewis_Carroll

\chapter{Cypherpunks Skriver Kode}
\label{les:20}

\begin{chapquote}{Lewis Carroll, \textit{Alice i Eventyrland}}
\enquote{Jeg ser, du prøver at opfinde noget.}
\end{chapquote}

Som mange fantastiske idéer kom Bitcoin ikke ud af ingenting. Det blev
muliggjort ved at anvende og kombinere mange innovationer og opdagelser inden
for matematik, fysik, datalogi og andre områder. Mens
uundgåeligt en geni, ville Satoshi ikke have kunnet opfinde Bitcoin
uden de kæmper, hvis skuldre han stod på.

\begin{quotation}\begin{samepage}
\enquote{Den, der kun ønsker og håber, blander sig ikke aktivt i 
begivenhedernes gang og i formningen af sin egen skæbne.}
\begin{flushright} -- Ludwig von Mises\footnote{Ludwig von Mises, 
  \textit{Human Action} \cite{human-action}}
\end{flushright}\end{samepage}\end{quotation}
% > <cite>[Ludwig Von Mises]</cite>

En af disse kæmper er Eric Hughes, en af grundlæggerne af 
cypherpunk-bevægelsen og forfatter til \textit{A Cypherpunk's Manifesto}. 
Det er svært at forestille sig, at Satoshi ikke blev påvirket af dette 
manifest. Det taler om mange ting, som Bitcoin muliggør og udnytter, 
såsom direkte og private transaktioner, elektroniske penge og kontanter, 
anonyme systemer og forsvar af privatlivet med kryptografi og digitale 
signature.

\begin{quotation}\begin{samepage}
\enquote{Privatliv er nødvendigt for et åbent samfund i den elektroniske 
tidsalder. [...] Da vi ønsker privatliv, skal vi sikre, at hver part i en
transaktion kun har kendskab til det, der er direkte nødvendigt for den
transaktion. [...] Derfor kræver privatliv i et åbent samfund anonyme 
transaktionssystemer. Indtil nu har kontanter været det primære sådanne 
system. Et anonymt transaktionssystem er ikke et hemmeligt transaktionssystem.
[...]
Vi cypherpunks dedikerer os til at opbygge anonyme systemer. Vi forsvarer 
vores privatliv med kryptografi, med anonyme mail
videresendelsessystemer, med digitale signaturer og med elektroniske
penge.
Cypherpunks skriver kode.}
\begin{flushright} -- Eric Hughes\footnote{Eric Hughes, A Cypherpunk's 
  Manifesto \cite{cypherpunk-manifesto}}
\end{flushright}\end{samepage}\end{quotation}

Cypherpunks finder ikke trøst i håb og ønsker. De blander sig aktivt
i begivenhedernes gang og former deres egen skæbne.
Cypherpunks skriver kode.

Således, i sand cypherpunk-stil, satte Satoshi sig ned og begyndte at skrive
kode. Kode, der tog en abstrakt idé og beviste for verden, at det rent 
faktisk fungerede. Kode, der såede frøet til en ny økonomisk virkelighed.
Takket være denne kode kan alle verificere, at dette nye system rent faktisk
fungerer, og cirka hvert 10. minut beviser Bitcoin for verden, at det stadig 
er i live.

\begin{center}
  \includegraphics[width=\textwidth]{assets/images/bitcoin-code-white.png}
  \captionof{figure}{Kodeuddrag fra Bitcoin version 0.1}
  \label{fig:bitcoin-code-white}
\end{center}

For at sikre, at hans innovation transcenderer fantasi og bliver virkelighed,
skrev Satoshi kode for at implementere sin idé, før han skrev whitepaperet. 
Han sørgede også for ikke at udsætte\footnote{\enquote{Vi bør ikke udsætte 
det for evigt, indtil enhver mulig funktion er færdig.} -- 
Satoshi Nakamoto~\cite{satoshi-delay}} enhver udgivelse for evigt.
Alt i alt, \enquote{der vil altid være én mere ting at gøre.}

\begin{quotation}\begin{samepage}
\enquote{Jeg måtte skrive al koden, før jeg kunne overbevise mig selv om, at jeg
kunne løse hvert problem, derefter skrev jeg papiret.}
\begin{flushright} -- Satoshi Nakamoto\footnote{Satoshi Nakamoto, Sv: 
  Bitcoin P2P e-cash paper \cite{satoshi-mail-code-first}}
\end{flushright}\end{samepage}\end{quotation}

I nutidens verden af endeløse løfter og tvivlsom udførelse var der desperat 
behov for en øvelse i dedikeret opbygning. Vær omhyggelig, overbevis
dig selv om, at du rent faktisk kan løse problemerne, og implementer
løsningerne. Vi bør alle stræbe efter at være lidt mere cypherpunk.

\paragraph{Bitcoin lærte mig, at cypherpunks skriver kode.}

% ---
%
% #### Down the Rabbit Hole
%
% - [Bitcoin version 0.1.0 announcement][version 0.1.0] by Satoshi Nakamoto
% - [Bitcoin P2P e-cash paper announcement][mail-announcement] by Satoshi Nakamoto
%
% [mail-announcement]: http://www.metzdowd.com/pipermail/cryptography/2008-October/014810.html
% [Ludwig Von Mises]: https://mises.org/library/human-action-0/html/pp/613
% [version 0.1.0]: https://bitcointalk.org/index.php?topic=68121.0
% [not to delay]: https://bitcointalk.org/index.php?topic=199.msg1670#msg1670
% [6]: http://www.metzdowd.com/pipermail/cryptography/2008-November/014832.html
%
% <!-- Wikipedia -->
% [alice]: https://en.wikipedia.org/wiki/Alice%27s_Adventures_in_Wonderland
% [carroll]: https://en.wikipedia.org/wiki/Lewis_Carroll

\chapter{Metaforer for Bitcoins fremtid}
\label{les:21}

\begin{chapquote}{Lewis Carroll, \textit{Alice i Eventyrland}}
\enquote{Jeg ved, at der er noget interessant, der helt sikkert vil ske\ldots}
\end{chapquote}

I løbet af de sidste par årtier blev det tydeligt, at teknologisk innovation 
ikke følger en lineær trend. Uanset om du tror på den teknologiske singularitet 
eller ej, er det uundgåeligt, at fremskridt er eksponentielt i mange områder. 
Ikke kun det, men hastigheden, hvormed teknologier bliver adopteret, 
accelererer, og før du ved af det, er busken på skolegården væk, og dine 
børn bruger i stedet Snapchat. Eksponentielle kurver har en tendens til at 
slå dig i ansigtet længe før, du ser dem komme.

Bitcoin er en eksponentiel teknologi bygget på eksponentielle teknologier.
\textit{Our World in Data}\footnote{\url{https://ourworldindata.org/}} viser 
smukt den stigende hastighed af teknologisk adoption, der starter i 1903 med 
introduktionen af landlinjer (se figur~\ref{fig:tech-adoption}). Landlinjer, 
elektricitet, computere, internet, smartphones; følger alle eksponentielle 
tendenser inden for pris-ydelsesforhold og adoption. Det gør Bitcoin 
også~\cite{tech-adoption}.

\begin{center}
  \includegraphics[width=\textwidth]{assets/images/tech-adoption.png}
  \captionof{figure}{Bitcoin er bogstaveligt talt uden for diagrammet.}
  \label{fig:tech-adoption}
\end{center}

Bitcoin har ikke kun én, men flere netværkseffekter\footnote{Trace Mayer,
\textit{The Seven Network Effects of Bitcoin}~\cite{7-network-effects}}, alle
som resulterer i eksponentielle vækstmønstre inden for deres respektive 
områder: pris, brugere, sikkerhed, udviklere, markedsandel og global adoption 
som penge.

Efter at have overlevet sin spæde barndom vokser Bitcoin fortsat hver dag på
flere områder end ét. Teknologien er måske ikke nået fuld modenhed endnu.
Den kan være i sin adolescens. Men hvis teknologien er eksponentiel, er vejen
fra obskuritet til udbredelse kort.

\begin{center}
  \includegraphics[width=\textwidth]{assets/images/mobile-phone.png}
  \captionof{figure}{Mobiltelefon, ca. 1965 vs. 2019.}
  \label{fig:mobile-phone}
\end{center}

I sin TED-talk fra 2003 valgte Jeff Bezos at bruge elektricitet som en metafor
for websitets fremtid.\footnote{\url{http://bit.ly/bezos-web}} Alle tre 
fænomener --- elektricitet, internettet, Bitcoin --- er \textit{muliggørende} 
teknologier, netværk, der gør andre ting mulige. De er infrastruktur at bygge 
på, grundlæggende i naturen.

Elektricitet har været til stede i lang tid nu. Vi tager det for givet. 
Internettet er noget yngre, men de fleste mennesker tager det også for givet. 
Bitcoin er ti år gammel og er trådt ind i den offentlige bevidsthed under den 
seneste hype-cyklus. Kun de tidligste adoptører tager det for givet. Jo mere 
tid der går, desto flere mennesker vil genkende Bitcoin som noget, der bare 
er.\footnote{Dette er kendt som \textit{Lindy-effekten}. Lindy-effekten er 
en teori om, at den fremtidige forventede levetid for visse ikke-forgængelige 
ting som teknologi eller en idé er proportional med deres nuværende alder, 
således at hver ekstra overlevelsesperiode indebærer en længere forventet 
levetid.~\cite{wiki:lindy}}

I 1994 var internettet stadig forvirrende og uhåndterligt. At se denne gamle 
optagelse af \textit{Today Show}\footnote{\url{https://youtu.be/UlJku_CSyNg}} 
gør det tydeligt, at hvad der føles naturligt og intuitivt nu faktisk ikke 
var det dengang. Bitcoin er stadig forvirrende og fremmed for de fleste, 
men ligesom internettet er anden natur for digitale indfødte, vil at bruge 
og stable sats\footnote{\url{https://twitter.com/hashtag/stackingsats}} 
være anden natur for fremtidens Bitcoin-indfødte.

\begin{quotation}\begin{samepage}
\enquote{Fremtiden er allerede her --- den er bare ikke særlig ligeligt fordelt.}
\begin{flushright} -- William Gibson\footnote{William Gibson, 
  \textit{Videnskaben i science fiction} \cite{william-gibson}}
\end{flushright}\end{samepage}\end{quotation}

I 1995 brugte omkring $15\%$ af amerikanske voksne internettet. Historiske 
data fra Pew Research Center~\cite{pew-research} viser, hvordan internettet 
har indvævet sig i alle vores liv. Ifølge en forbrugerundersøgelse foretaget 
af Kaspersky Lab~\cite{web:kaspersky} har 13\% af respondenterne brugt 
Bitcoin og dets kloner til at betale for varer i 2018. Selvom betalinger 
ikke er den eneste anvendelse af bitcoin, er det en indikation af, hvor vi 
er i internettets tid: i starten til midten af 90'erne.

I 1997 erklærede Jeff Bezos i et brev til aktionærerne~\cite{bezos-letter}, 
at \enquote{dette er dag 1 for internettet,} idet han anerkendte det store 
uudnyttede potentiale for internettet og dermed hans virksomhed. Uanset 
hvilken dag det er for Bitcoin, er de enorme mængder uudnyttet potentiale 
tydelige for alle, undtagen den mest tilfældige observatør.

\begin{center}
  \includegraphics[width=\textwidth]
  {assets/images/internet-evolution-white-dates.png}
  \captionof{figure}{Internettet, 1982 vs. 2005. Kilde: cc-by Merit Network, 
  Inc. og Barrett Lyon, Opte Project}
  \label{fig:internet-evolution-white-dates}
\end{center}

Bitcoin's første node gik online i 2009, efter at Satoshi mined \textit{genesis
blokken}\footnote{Genesisblokken er den første blok i Bitcoin-blokkenkæden.
Moderne versioner af Bitcoin nummererer den som blok $0$, selvom meget tidlige 
versioner tællede den som blok $1$. Genesisblokken er normalt hårdkodet i
softwaren i applikationer, der bruger Bitcoin-blokkenkæden. Det er et
særligt tilfælde, fordi det ikke henviser til en tidligere blok og producerer en
udfoldelig subsidie. \textit{coinbase}-parameteren indeholder, sammen med
normal data, følgende tekst: \textit{\enquote{The Times 03/Jan/2009 Chancellor 
on brink of second bailout for banks}} \cite{btcwiki:genesis-block}} og udgav
softwaren i det vilde. Hans node var ikke alene i lang tid. Hal Finney var en
af de første til at opfatte ideen og slutte sig til netværket. Ti år
senere, ved skrivelsen af dette, kører mere end
$75.000$\footnote{\url{https://bit.ly/luke-nodecount}} noder Bitcoin.

\begin{center}
  \centering
  \includegraphics[width=8cm]{assets/images/running-bitcoin.png}
  \captionof{figure}{Hal Finney forfattede den første tweet, der nævnte 
  bitcoin i januar 2009.}
  \label{fig:running-bitcoin}
\end{center}

Protokollens basislag er ikke det eneste, der vokser eksponentielt.
Lightning-netværket, en teknologi på andet niveau, vokser endnu
hurtigere.

I januar 2018 havde lynnetværket $40$ noder og $60$
kanaler~\cite{web:lightning-nodes}. I april 2019 voksede netværket til mere
end $4000$ noder og omkring $40.000$ kanaler. Husk, at dette stadig er
eksperimentel teknologi, hvor tab af midler kan og sker. Alligevel er tendensen
tydelig: tusindvis af mennesker er uforsigtige og ivrige efter at bruge det.

\begin{center}
  \includegraphics[width=\textwidth]{assets/images/lnd-growth-lopp-white.png}
  \captionof{figure}{Lightning Network, januar 2018 vs december 2018. 
  Kilde: Jameson Lopp}
  \label{fig:lnd-growth-lopp-white.png}
\end{center}

For mig, der har oplevet den meteoriske stigning af internettet, er parallelle
mellem internettet og Bitcoin åbenlyse. Begge er netværk, begge
er eksponentielle teknologier, og begge muliggør nye muligheder, nye
industrier, nye livsformer. Ligesom elektricitet var den bedste
metafor for at forstå, hvor internettet er på vej hen, kan internettet
være den bedste metafor for at forstå, hvor Bitcoin er på vej hen. Eller, som
Andreas Antonopoulos siger det, er Bitcoin \textit{The Internet of Money}.
Disse metaforer er en fantastisk påmindelse om, at selvom historie ikke gentager
sig, rimer den ofte.

Eksponentielle teknologier er svære at forstå og ofte undervurderede.
Selvom jeg har stor interesse i sådanne teknologier, bliver jeg
konstant overrasket over tempoet af fremskridt og innovation. At se
Bitcoin-økosystemet vokse er som at se internettets opstigning i
hurtigafspilning. Det er opløftende.

Min søgen efter at forsøge at forstå Bitcoin har ført mig ned ad stierne
af historie på mere end én måde. At forstå gamle samfundsstrukturer, 
tidligere valutaer, og hvordan kommunikationsnetværk udviklede sig, var alle
en del af rejsen. Fra håndøksen til smartphones har teknologi
utvivlsomt ændret vores verden mange gange. Netværksteknologier
er især transformationelle: skrivning, veje, elektricitet, internettet.
Alle har ændret verden. Bitcoin har ændret min og
vil fortsætte med at ændre sind og hjerter hos dem, der tør bruge
det.

\paragraph{Bitcoin lærte mig, at forståelse af fortiden er afgørende for
at forstå dens fremtid. En fremtid, der kun lige er begyndt\ldots}

% ---
%
% #### Down the Rabbit Hole
%
% - [The Rising Speed of Technological Adoption][the rising speed of technological adoption] by Jeff Desjardins
% - [The 7 Network Effects of Bitcoin][multiple network effects] by Trace Mayer
% - [The Electricity Metaphor for the Web's Future][TED talk] by Jeff Bezos
% - [How the internet has woven itself into American life][data from the Pew Research Center] by Susannah Fox and Lee Rainie
% - [Genesis Block][genesis block] on the Bitcoin Wiki
% - [Lindy Effect][more time] on Wikipedia
%
% [Our World in Data]: https://ourworldindata.org/
% [the rising speed of technological adoption]: https://www.visualcapitalist.com/rising-speed-technological-adoption/
% [multiple network effects]: https://www.thrivenotes.com/the-7-network-effects-of-bitcoin/
% [TED talk]: https://www.ted.com/talks/jeff_bezos_on_the_next_web_innovation
% [recording of the Today Show]: https://www.youtube.com/watch?v=UlJku_CSyNg
% [William Gibson]: https://www.npr.org/2018/10/22/1067220/the-science-in-science-fiction
% [data from the Pew Research Center]: https://www.pewinternet.org/2014/02/27/part-1-how-the-internet-has-woven-itself-into-american-life/
% [consumer survey]: https://www.kaspersky.com/blog/money-report-2018/
% [letter to shareholders]: http://media.corporate-ir.net/media_files/irol/97/97664/reports/Shareholderletter97.pdf
% [running bitcoin]: https://twitter.com/halfin/status/1110302988?lang=en
% [40 nodes]: https://bitcoinist.com/bitcoin-lightning-network-mainnet-nodes/
% [reckless]: https://twitter.com/hashtag/reckless
% [Jameson Lopp]: https://twitter.com/lopp/status/1077200836072296449
% [\textit{The Internet of Money}]: https://theinternetofmoney.info/
% [stacking]: https://twitter.com/hashtag/stackingsats
%
% <!-- Bitcoin Wiki -->
% [genesis block]: https://en.bitcoin.it/wiki/Genesis_block
%
% <!-- Wikipedia -->
% [more time]: https://en.wikipedia.org/wiki/Lindy_effect
% [alice]: https://en.wikipedia.org/wiki/Alice%27s_Adventures_in_Wonderland
% [carroll]: https://en.wikipedia.org/wiki/Lewis_Carroll

\addpart{Afsluttende tanker}
\pdfbookmark{Konklusion}{konklusion}
\label{ch:konklusion}

\chapter*{Konklusion}

\begin{chapquote}{Lewis Carroll, \textit{Alice i Eventyrland}}
\enquote{Begynd ved begyndelsen,} sagde Kongen meget alvorligt, \enquote{og 
fortsæt, indtil du kommer til enden: så stop.}
\end{chapquote}

Som nævnt i begyndelsen mener jeg, at enhver besvarelse på spørgsmålet 
\textit{“Hvad har du lært af Bitcoin?”} altid vil være ufuldstændig. Symbiosen 
af det, der kan betragtes som flere levende systemer - Bitcoin, teknosfæren og 
økonomi - er for indviklet, emnerne for mange, og ting bevæger sig for hurtigt 
til nogensinde at blive fuldt forstået af én person.

Selv uden fuld forståelse og selv med alle dens særheder og tilsyneladende 
mangler fungerer Bitcoin utvivlsomt. Den fortsætter med at producere blokke 
cirka hvert tiende minut og gør det smukt. Jo længere Bitcoin fortsætter med at 
fungere, desto flere mennesker vil vælge at bruge det.

\begin{quotation}\begin{samepage}
\enquote{Det er sandt, at ting er smukke, når de virker. Kunst er funktion.}
\begin{flushright} -- Giannina Braschi\footnote{Giannina Braschi, \textit{Empire of Dreams} \cite{braschi2011empire}}
\end{flushright}\end{samepage}\end{quotation}

\paragraph{} Bitcoin er et barn af internettet. Det vokser eksponentielt, 
udvisker grænserne mellem discipliner. Det er ikke klart, for eksempel, hvor 
den rene teknologis rige slutter, og hvor en anden sfære begynder. Selvom 
Bitcoin kræver computere for at fungere effektivt, er datalogi ikke 
tilstrækkelig til at forstå det. Bitcoin er ikke kun grænseløs med hensyn til 
dets indre funktioner, men også grænseløs med hensyn til akademiske discipliner.

Økonomi, politik, spilteori, pengehistorie, netværksteori, finans, kryptografi, 
informationsteori, censur, lov og regulering, menneskelig organisation, 
psykologi - alle disse og mere er områder af ekspertise, der måske kan hjælpe 
med at forstå, hvordan Bitcoin fungerer og hvad Bitcoin er.

Ingen enkelt opfindelse er ansvarlig for dens succes. Det er kombinationen af 
flere, tidligere ikke-relaterede dele, limet sammen af spilteoretiske 
incitamenter, der udgør revolutionen, som er Bitcoin. Den smukke blanding af 
mange discipliner er det, der gør Satoshi til en geni.

\paragraph{} Ligegyldigt hvor komplekst et system er, skal Bitcoin træffe 
kompromisser med hensyn til effektivitet, omkostning, sikkerhed og mange andre 
egenskaber. Ligesom der ikke findes en perfekt løsning på at udlede en firkant 
fra en cirkel, vil enhver løsning på de problemer, som Bitcoin forsøger at 
løse, altid være ufuldkommen.

\begin{quotation}\begin{samepage}
\enquote{Jeg tror ikke, vi nogensinde vil have gode penge igen, før vi tager
tingen ud af regeringens hænder, det vil sige, vi kan ikke tage det med
vold ud af regeringens hænder, alt hvad vi kan gøre, er på en snedig omvej
introducere noget, de ikke kan stoppe.}
\begin{flushright} -- Friedrich Hayek\footnote{Friedrich Hayek om pengepolitik, 
    guldfoden, underskud, inflation og John Maynard Keynes 
    \url{https://youtu.be/EYhEDxFwFRU}}
\end{flushright}\end{samepage}\end{quotation}

Bitcoin er den snedige, omvejsfulde måde at genindføre gode penge i verden på. 
Den gør det ved at placere en suveræn individ bag hver node, præcis som Da 
Vinci forsøgte at løse det uløselige problem med at kvadrere en cirkel ved at 
placere Vitruvian Man i dens centrum. Noder fjerner effektivt enhver opfattelse 
af et centrum og skaber et system, der er forbløffende antifragilt og ekstremt 
svært at lukke ned. Bitcoin lever, og dens hjerteslag vil sandsynligvis 
overleve os alle.

Jeg håber, du har nydt disse enogtyve lektioner. Måske er den mest vigtige 
lektion, at Bitcoin bør undersøges holistisk, fra flere vinkler, hvis man gerne 
vil have noget, der nærmer sig et komplet billede. Ligesom at fjerne én del 
fra et komplekst system ødelægger det hele, synes at undersøge dele af Bitcoin 
isoleret at forurene forståelsen af det. Hvis blot én person fjerner 
\enquote{blockchain} fra sit ordforråd og erstatter det med 
\enquote{en kæde af blokke}, vil jeg dø en lykkelig mand.

Under alle omstændigheder fortsætter min rejse. Jeg planlægger at bevæge mig 
dybere ned i dette kaninhul, og jeg inviterer dig til at følge med
på turen.\footnote{\url{https://twitter.com/dergigi}}

% <!-- Twitter -->
% [dergigi]: https://twitter.com/dergigi
%
% <!-- Internal -->
% [sly roundabout way]: https://youtu.be/EYhEDxFwFRU?t=1124
% [Giannina Braschi]: https://en.wikipedia.org/wiki/Braschi%27s_Empire_of_Dreams


\cleardoublepage

\input{backmatter/acknowledgments}

\listoffigures

\input{backmatter/bibliography}
\printbibliography
\theendnotes

\end{document}
